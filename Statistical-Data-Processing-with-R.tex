\documentclass[runningheads,14pt,a4paper,openany]{book}

% Добавя възможност за сензитивни хипер-връзки в самия документ.
\usepackage[linktocpage=true]{hyperref}
\usepackage{nameref}
\usepackage[utf8x]{inputenc}
\usepackage[english,bulgarian]{babel}
\usepackage{url}
\usepackage{lipsum}
% Според изискванията на ИИКТ-БАН не бива да има номерация на страниците в ръкописа.
%\usepackage{nopageno}
\usepackage{shorttoc}
\usepackage[pdftex]{graphicx}
% Директория в която се намират изображенията.
\graphicspath{{images/}}
\usepackage{array,arydshln}
\usepackage{imakeidx}
\usepackage{placeins}
% Използва се за включване на кориците под формата на PDF файлове.
\usepackage{pdfpages}
% Служи за управление на заглавията.
\usepackage{fancyheadings}
\usepackage{listings}

% Добавени от доц. Вера Ангелова.
%\usepackage{amsmath,amssymb,amsthm}
%\usepackage{longtable}
%\usepackage{pifont}   
%\usepackage{epsfig}
%\usepackage{tocloft}
%\usepackage{etoc}
%\usepackage{wasysym}
%\usepackage{eurosym}
%\usepackage{slashbox}
%\usepackage{soul}
%\usepackage{enumitem}
%\usepackage{mathcomp}
%\usepackage{epstopdf}
%\usepackage{latexsym}
%\usepackage{eucal}
%\usepackage{mathrsfs}

\textheight 22.8cm
\textwidth 17cm
\oddsidemargin -0.54cm
\evensidemargin -0.54cm
\topmargin 1.5cm

\parskip=0.2cm
\parindent=20pt
\flushbottom

% Премахва подчертаващата линия в заглавните части.
\renewcommand{\headrulewidth}{0pt}

\selectlanguage{bulgarian}

\lhead[\thepage \quad Тодор Балабанов, Румен Кетипов, Зорница Атанасова \quad \hfill]{}
\chead{}
\rhead[]{\hfill Статистическа обработка на данни с R \quad \thepage }
\lfoot{}
\cfoot[\em Лекции по компютърни науки и технологии на ИИКТ - БАН, № *, 20**]{\em Лекции по компютърни науки и технологии на ИИКТ - БАН, № *, 20**} 
\rfoot{}

\onecolumn
\makeindex[columns=1, title=Азбучен указател, intoc]

% Подменя думата използван а за ноемрация на фрагментите програмен код.
\renewcommand{\lstlistingname}{Листинг}

% Определя цвета за фон на програмния код.
\lstset{backgroundcolor=\color{lightgray}}

\begin{document}

\def\ql{\textquotedblleft}\def\qr{\textquotedblright}

\includepdf[pages={1,2}]{images/front}
\thispagestyle{empty}

\voffset =-1truecm

% Използва се за номерация на страниците.
\renewcommand{\thepage}{\roman{page}}

% Смяна на названието за списъка от листингите
\renewcommand{\lstlistlistingname}{Списък на листингите}

\setcounter{page}{-1}
\thispagestyle{empty}
\pagestyle{empty}
\thispagestyle{empty}

% Тук стои таблицата със съдържанието, което се генерира от названието на главите.
\newpage
\thispagestyle{empty}
\pagestyle{empty}
\shorttoc{Теми}{0}
\thispagestyle{empty}
\pagestyle{empty}

% Тук стои таблицата със съдържанието, което се генерира от названието на главите и названието на секциите в тях.
\newpage
\thispagestyle{empty}
\pagestyle{empty}
\thispagestyle{empty}
\tableofcontents
\thispagestyle{empty}
\pagestyle{empty}

% Списък с фигурите.
\newpage
\listoffigures
\addcontentsline{toc}{chapter}{Списък на фигурите}

% Списък с таблиците. 
\newpage
\listoftables
\addcontentsline{toc}{chapter}{Списък на таблиците}

% Списък с листингите.
\newpage
\lstlistoflistings
\addcontentsline{toc}{chapter}{Списък на листингите}

\newpage
\addcontentsline{toc}{chapter}{Предговор}
\chapter*{Предговор}
\pagenumbering{arabic}
\setcounter{page}{1}
\pagestyle{fancyplain}

Това учебно помагало е предвидено за студенти и докторанти, които в своите магистърски или докторски тези се сблъскват с потребността да извършат определени експерименти, а след това да обработят статистически получените резултати. 

В съвременния живот нуждата от обработка на информация единствено нараства. В множество ситуации от ежедневния ни живот се налага да бъдат вземани решения. От своя страна, всяко решение е толкова по-успешно, колкото по-информирано е взето то. Статистическата обработка на събраната информация е една от основните за вземането на информирани решения. В областта на статистическата обработка съществуват множество софтуерни решения, като се започне от по-достъпните за хора без опит, като Microsoft Excel и се стигне до професионалните пакети, като SPSS, Matlab и Mathematica. 

Това учебно помагало представя програмния продукт R, който първоначално се разработва от Robert Gentleman и Ross Ihaka в University of Auckland през 1993 година. R е замислен като алтернатива на програмния продукт S, създаден от John Chambers, служител в Bell Labs. Първоначалният замисъл за R е инструмент, който да бъде използван в интерактивен режим, през командния ред. В последствие тази идея прераства в самостоятелен програмен език. Основното предназначение на R е обработка на данни, което включва въвеждане, пресмятане, визуализация на графики и отчети. 

Езикът получава значително по голяма популярност след 2000 година, като излиза от рамките на академичните среди и навлиза в финансовите среди, маркетинга, фармацията, социологията, психологията и в много други области. Най-често потребителите на R са хора с опит в програмни езици, като C/C++, Java, C\# или пък преди това са използвали други статистически пакети, като SAS, SPSS и дори Excel. Тези потребители дават значителен тласък в развитието на пакета R, добавяйки множество софтуерни приставки (add-ons). 

Въпреки че в някои случаи R се оказва стряскащ и дори смущаващ, особено за начинаещите потребители, с времето и с процеса по навлизане в материята овладяването му се улеснява и ускорява. Това учебно помагало представя информацията по един достъпен и олекотен начин за възприемане. Изложени са предимно най-важните аспекти от използването на пакета R, което от своя страна дава стабилна основа за бъдещо самостоятелно развитие на читателя. Материалът е съобразен със съдържанието на курса „Анализ на данни с R“, провеждан в „Център за обучение“ към „Българска академия на науките. Учебното помагало е организирано в следните глави.

Глава 1 - \nameref{chapter01}: Представя процеса по инсталиране и стартиране на програмния пакет.
\newpage
\chapter{Инсталация и стартиране}
\label{chapter01}

Тъй като програмният продукт R се разработва под формата на софтуер с отворен код, то употребата му с не търговска цел не изисква заплащане. За работа с R е достатъчно да се инсталира основният пакет, въпреки това съществува и интегрирана среда за разработка наречена R Studio. За нуждите на учебното помагало ще бъде използван само основният пакет. Всеки желаещ да разшири уменията си с използването на интегрираната среда за разработка, би могъл самостоятелно да разучи възможностите ѝ.

\section{Изтегляне на инсталационните файлове}

\begin{figure}[h!]
  \centering
  \includegraphics[width=1.0\linewidth]{pic0001}
  \caption{Начална уеб страница на продукта}
\label{fig:pic0001}
\end{figure}
\FloatBarrier

Както множество софтуерно продукти и R е достъпен за изтегляне от уеб страницата на продукта в Интернет (Фиг. \ref{fig:pic0001}) с адрес: http://www.r-project.org/ 

\begin{figure}[h]
  \centering
  \includegraphics[width=1.0\linewidth]{pic0002}
  \caption{Списък със сървъри за изтегляне}
\label{fig:pic0002}
\end{figure}
\FloatBarrier

В раздела за изтегляне са посочени множество активни връзки към различни географски локации (Фиг. \ref{fig:pic0002}). Обичайна практика, при софтуерните продукти с отворен код, е наличието на множество сървъри, разположени по цял свят, да предлагат изтегляне на файловете нужни за инсталацията. Тази практика се е наложила най-вече за ускоряване на изтеглянето, но също така и за намаляване на натоварването, което сървърите понасят при множество заявки. Не на последно място, много често за разпространението на инсталационните файлове се разчита на доброволческа информационна инфраструктура, за която не се заплаща. 

\begin{figure}[h]
  \centering
  \includegraphics[width=1.0\linewidth]{pic0003}
  \caption{Избор на подходящата за операционната система инсталация}
\label{fig:pic0003}
\end{figure}
\FloatBarrier

Често срещана практика е продуктите с отворен кода да се разпространяват за трите най-популярни операционни системи, този принцип е спазен и за продукта R (Фиг. \ref{fig:pic0003}). При комерсиалните софтуерни продукти много често се залага на една единствена операционна система, но при продуктите с отворен код идеологията е, че трябва да се достигне до възможно най-голям брой потребители и поради тази причина се полагат допълнителни усилия софтуерът да работи на възможно най-много платформи (платформа – комбинация между хардуер и операционна система). Тази стратегия за дълготрайно развитие залага и на следващата особеност в развитието на продуктите с отворен код, а именно, че една част от потребителите с времето се превръщат в хора добавящи програмен код към продуктите. Освен видът на операционната система, от значение е и размерът на машинната дума, която процесорът поддържа. Към настоящия момент, най-разпространени са изчислителните машини с 64 битова машинна дума, но тъй като все още има много техника, която работи на 32 битова машинна дума, продуктът R е достъпен и за двата варианта. 

\begin{figure}[h]
  \centering
  \includegraphics[width=1.0\linewidth]{pic0004}
  \caption{Избор на версия за изтегляне}
\label{fig:pic0004}
\end{figure}
\FloatBarrier

Добра практика е при работата със софтуерни продукти, които се разпространяват като отворен код, винаги да се използва най-новата стабилна версия. В случая, за операционната система Mac OS X (Фиг. \ref{fig:pic0004}), това е версията 3.5.1, която е налична под формата на инсталационне файл R-3.5.1.pkg (Фиг. \ref{fig:pic0003}).

\begin{figure}[h]
  \centering
  \includegraphics[width=1.0\linewidth]{pic0005}
  \caption{Запазване на инсталационния файл}
\label{fig:pic0005}
\end{figure}
\FloatBarrier

\section{Инсталация}

За всяка от операционните системи е достатъчно потребителят да следва инструкциите и инсталацията протича безпроблемно по указаните стъпки. 

\begin{figure}[h]
  \centering
  \includegraphics[width=1.0\linewidth]{pic0006}
  \caption{Активиране на инсталатора}
\label{fig:pic0006}
\end{figure}
\FloatBarrier

С двойно щракване на мишката се активира инсталатора (Фиг. \ref{fig:pic0006}). След което следва екран с подробности за самата инсталация (Фиг. \ref{fig:pic0007}).

\begin{figure}[h]
  \centering
  \includegraphics[width=1.0\linewidth]{pic0007}
  \caption{Подробности за инсталацията}
\label{fig:pic0007}
\end{figure}
\FloatBarrier

От съществено значение е потребителите на продукти с отворен код да са добре запознати с условията при които те получават продуктите, особено когато е без заплащане. Поради тази причина, потребителят трябва изрично да се съгласи с условията на лиценза под който се разпространява продуктът R (Фиг. \ref{fig:pic0008},\ref{fig:pic0009}).

\begin{figure}[h]
  \centering
  \includegraphics[width=1.0\linewidth]{pic0008}
  \caption{Лицензно споразумение за ползване}
\label{fig:pic0008}
\end{figure}
\FloatBarrier

Софтуерният продукт R се разпространява под отворен лиценз GPL2, който в най-общи линии очертава рамките на условията при които потребителите получават софтуера. Най-важните клаузи в лиценза са свързани с липса на гаранция и с изричното съгласие на потребителя, че производителят не носи никаква юридическа отговорност, произтекла от употребата на софтуерния продукт. Пълният текст на лицензът е достъпен в уеб страницата на фондацията, която го поддържа \cite{gpl2}.

\begin{figure}[h]
  \centering
  \includegraphics[width=1.0\linewidth]{pic0009}
  \caption{Изрично съгласие}
\label{fig:pic0009}
\end{figure}
\FloatBarrier

Съвременните операционни системи са от отворен тип и инсталацията на допълнителни софтуерни продукти се явяват своеобразно разширение на операционната система. Поради тази причина, всеки създател на операционна система е избрал правила и начини за добавяне на софтуерни продукти. Една от основните характеристики е определяне на директория във файловата система на операционната система, където новодобавеният софтуер ще бъде поставен. Някои операционни системи (примерно Linux базираните дистрибуции) използват специално организиран мениджър на пакетите, който има грижата за консистентността на добавяните софтуерни продукти. При Mac OS X също е налична възможността за автоматично управление на инсталациите, но е дадена и възможност потребителят да избира мястото на инсталация. В операционната система Microsoft Windows, потребителят решава къде да помести новоинсталирания софтуер. 

\begin{figure}[h]
  \centering
  \includegraphics[width=1.0\linewidth]{pic0010}
  \caption{Информация за директорията и използваното дисково пространство}
\label{fig:pic0010}
\end{figure}
\FloatBarrier

При желание е възможно да бъде подменена инсталационната директория (Фиг. \ref{fig:pic0010}).

\begin{figure}[h]
  \centering
  \includegraphics[width=1.0\linewidth]{pic0011}
  \caption{Успешно приключване на инсталацията}
\label{fig:pic0011}
\end{figure}
\FloatBarrier

Инсталацията приключва със съобщение за успешно изпълнение (Фиг. \ref{fig:pic0011}).

\section{Работа в режим на команди}

Инсталаторът създава икона за стартиране на R командния интерпретатор.

\begin{figure}[h]
  \centering
  \includegraphics[width=1.0\linewidth]{pic0012}
  \caption{Основен прозорец на продукта}
\label{fig:pic0012}
\end{figure}
\FloatBarrier

При успешна инсталация и стартиране, потребителят получава достъп до прозорец служещ като команден интерпретатор (Фиг. \ref{fig:pic0012}). Първоначалният замисъл за продуктът R е бил това да представлява интерпретатор на команди. В интерактивен режим потребителят въвежда команда и наблюдава получения резултат. Макар и това да е основният начин за работа, R позволява последователността от команди да бъде записана във файл и да се изпълняват като единен скрипт. За много потребители, особено такива с предишен опит в софтуерни продукти с силно застъпен графичен потребителски интерфейс (примерно статистически анализ на данни в Microsoft Excel), използването на терминал с команди първоначално е трудно и дори дразнещо, но с напредване на времето потребителите оценяват гъвкавостта която този начин на работа позволява. Използването на поредица от команди често се оказва значително по-бърз начин за работа в сравнение с подготовката на експеримент в софтуер с графичен интерфейс. Също така, наличието на серията команди дава възможност значително по-бързо експериментът да бъде преповторен. Често при по-голям обем данни софтуерните пакети с графичен потребителски интерфейс забавят работата си неприемливо много. И не на последно място, наличието на статистическия модел като скрипт позволява използването на системи за контрол на версиите (каквато примерно е Git и облачната услуга GitHub), нещо което търговските бинарни файлови формати (примерно XLSX, в Microsoft Excel) не позволяват. При работата с командния интерпретатор на R, клавишът „стрелка нагоре“ повтаря последната използвана команда. Списъкът с вече изпълнени команди може да бъде преминат със стрелките нагоре и надолу. Тъй като интерпретацията на всяка команда става в момента на нейното повикване е възможно да бъде стартиран код, който да отнема твърде дълго време или да изпадне в безкраен цикъл. При тази ситуация, натискането на клавиша Esc или клавишната комбинация Ctrl+C прекъсва текущо изпълняваната команда. 

\begin{figure}[h]
  \centering
  \includegraphics[width=1.0\linewidth]{pic0013}
  \caption{Изпълнение на команда за печат}
\label{fig:pic0013}
\end{figure}
\FloatBarrier

При правилно работеща инсталация изписването на командата за печат би дала резултата показан на Фиг. \ref{fig:pic0013}. Както показва първата примерна команда, има фундаментална разлика в начина по който работят компилаторите и интерпретаторите. Езици като C/C++ изискват компилация на програмния код до машинни инструкции, които процесорът на компютъра изпълнява. При интерпретивните езици, като PHP, Python, JavaScript и разбира се R, всяка стъпка от програмата се интерпретира в момента на повикването й. По-напредналите потребители на R са добре запознати с начина по който са изградени библиотеките на езика и знаят добре, че съществуват възможности код писан на C/C++ да бъде изпълняван в средата на R. Такава межуезикова връзка най-често се налага при голям обем данни за пресмятане и относително бавни алгоритми, които извършват пресмятането. 

\newpage
\chapter{Пакетна организация и основни операции в R}
\label{chapter02}

Най-голямата сила на продукта R се дължи на хилядите пакети\index{пакети} (софтуерни приставки), създадени от безброй потребители на продукта. Наличните пакети покриват цялата област на статистиката и статистическата обработка на данни. 

Под пакет се разбира софтуерна библиотека от предварително написан програмен кой, който има за цел да реши определена задача или група от задачи. Тъй като продуктът R е една отворена система е важно да се има предвид, че не всички пакети са с еднакво качество. Една част от пакетите са изключително професионално написани, устойчиви са на некоректно използване и имат добра база от поддържащи ги потребители. В същото време друга част от пакетите са създадени с голяма доза добри намерения, но работят бавно, дават дефекти или просто не вършат това за което са създадени. Голяма част от пакетите са написани от статистици за статистици и това може да доведе до някои странни въпроси при част от потребителите, особено при хора идващи от индустрията за производство на софтуер. 

Настоящото учебно помагало представя само най-основните пакети, достатъчни да бъде изложен материалът свързан с базовите познания по R. Опит да бъдат представени всички пакети е непосилен за едно издание, най-вече защото броят и видът на пакетите постоянно се променя. 

\section{Инсталиране на пакети}

Съществуват различни начини за инсталиране на пакети в R, но най-основният от тях е чрез команда в конзолата на пакета R. 

\begin{figure}[h!]
  \centering
  \includegraphics[width=1.0\linewidth]{pic0014}
  \caption{Команда за инсталиране на пакета coefplot}
\label{figure0014}
\end{figure}
\FloatBarrier

За да започне инсталирането на пакет (в случая coefplot) е достатъчно да се изпише командата от Фиг. \ref{figure0014}.

\begin{figure}[h!]
  \centering
  \includegraphics[width=1.0\linewidth]{pic0015}
  \caption{Избор на сървър за изтегляне на пакета}
\label{figure0015}
\end{figure}
\FloatBarrier

След което следва избор на сървър за изтегляне на пакета (Фиг. \ref{figure0015}). Разумна стратегия е да се избират сървъри, които териториално се намират в близост до мястото от което се работи. Това би осигурило малко по-голяма бързина на връзката в Глобалната мрежа. 

\begin{figure}[h!]
  \centering
  \includegraphics[width=1.0\linewidth]{pic0016}
  \caption{Зависимости между пакетите}
\label{figure0016}
\end{figure}
\FloatBarrier

Често срещан случай е един пакет да има функционална зависимост от други пакети (Фиг. \ref{figure0016}). В такава ситуация е необходимо всички нужни пакети също да бъдат инсталирани. Стратегията при разработка на пакети е те да бъдат предлагани в компилиран (бинарен) вид, но понякога най-новите версии са под формата на програмен код и тогава потребителят има възможност да избере между бинарната версия или версията с програмен код. 

\begin{figure}[h!]
  \centering
  \includegraphics[width=1.0\linewidth]{pic0017}
  \caption{Резултат от инсталацията на пекета}
\label{figure0017}
\end{figure}
\FloatBarrier

Инсталацията на пакета приключва с подробен листинг, съдържащ описание на извършените операции (Фиг. \ref{figure0017}). 

\begin{figure}[h!]
  \centering
  \includegraphics[width=1.0\linewidth]{pic0018}
  \caption{Зареждане на пакета coefplot}
\label{figure0018}
\end{figure}
\FloatBarrier

Дали пакетът е надлежно инсталиран може да се провери с командата require (Фиг. \ref{figure0018}), която зарежда пакета в паметта. 

Съществува възможност пакетите да се инсталират под формата на програмен код, директно от хранилищата за програмен код, но за тази цел са нужни подходящите компилатори (най-често C/C++ и Fortran), както и по-задълбочени умения по програмиране. В редки случаи се налага инсталиране на пакета от ZIP файл. При такава ситуация е важно предварително да бъдат инсталирани всички пакети от които инсталирания пакет зависи. 

Премахване на инсталирани пакети става с помощта на командата remove.packages на която се подава вектор с имената на пакетите, които трябва да бъдат премахнати.

\section{Зареждане на пакети}

За да бъдат използвани пакетите не е достатъчно те да бъдат инсталирани, но трябва с команда да бъдат включени в текущата сесия от изчисления. R предлага две команди за зареждане на пакети – library и require. И двете изпълняват едно и също нещо – зареждат пакета в общата памет. Разликата е, че require връща TRUE, ако зареждането е било успешно и FALSE при неуспех. Тази възможност е полезна в редките случаи, когато пакетът се зарежда от програмния текст на функция. Подобна практика не е препоръчителна, но R дава такава възможност. И двете функции получават като параметър името на пакета, със или без кавички. Пакетите се зареждат еднократно и остават налични през цялата сесия от изчисления или докато изрично не бъдат премахнати от общата памет. 

\begin{figure}[h!]
  \centering
  \includegraphics[width=1.0\linewidth]{pic0019}
  \caption{Премахване на пакета coefplot от общата памет}
\label{figure0019}
\end{figure}
\FloatBarrier

Премахването на пакет от общата памет става с командата detach (Фиг. \ref{figure0019}). Същественото при тази команда е, че преди името на пакета се записва думата package. 

Тъй като пакетите се разработват основно на доброволни начала не рядко се случва в различни пакети да има едноименни функции. При подобна колизия на имената решението е операцията за принадлежност – двойно двуеточие (::). Когато бъде използвана операцията за принадлежност дори може да не се зарежда пакетът към който принадлежи функцията. 

\section{Основни математически операции}

R позволява да се извършват сложни математически пресмятания, но също така може да се използва и за базови математически сметки. 

\begin{figure}[h!]
  \centering
  \includegraphics[width=1.0\linewidth]{pic0020}
  \caption{Примерни аритметични операции}
\label{figure0020}
\end{figure}
\FloatBarrier

Най-базовите математически операции са събирането, изваждането, умножението и делението. Тези операции се изпълняват в R както е показано на Фиг. \ref{figure0020}. Пресмятанията в R се състоят от операции и операнди. Когато няколко операции бъдат обединени, чрез операндите си, се получава математически израз. 

\begin{lstlisting}[caption=Събиране, label=listing0001]
1 + 1
\end{lstlisting}

В листинг \ref{listing0001} е демонстрирана операцията за събиране, която има два операнда. Когато става въпрос за математически операции, те имат серия свойства. Като най-съществена характеристика може да се отбележи броят на операндите. Събирането е класически пример за бинарна операция, тъй като има два операнда (ляв и десен). 

\begin{lstlisting}[caption=Унарен минус, label=listing0002]
-5
\end{lstlisting}

В гимназиалния курс по математика не се споменава наличието на унарен плюс, макар и да се учи за унарен минус (Листинг \ref{listing0002}). Унарният плюс и минус променят значението на операнда. В примера от листинг \ref{listing0002}, унарният минус променя значението на числото пето от положително към отрицателно. Тъй като унарният плюс не бил променил значението на операнда си, масовата практика е унарничт плюс да не се записва, нещо което не е възможно с унарния минус. Освен унарни и бинарни операции в някои езици (примерно C/C++, Java, C\#, PHP и други) съществува една единствена тернарна операция (?:), която има смисъла на условния оператор за преход if. 

Както бе споменато по-горе, комбинацията от няколко операции и техните операнди водят до съставянето на математически израз (Листинг \ref{listing0003}). 

\begin{lstlisting}[caption=Аритметичен израз с две събирания, label=listing0003]
2 + 2 + 2
\end{lstlisting}

Тъй като съвременните изчислителни машини са организирани по такъв начин, че процесорът да извършва само една математическа операция на един такт от пресмятането, става актуален въпросът коя от операциите ще бъде изпълнена първа и коя втора, при положение, че операциите са с еднакъв приоритет. Тъй като в англо-саксонската писмена система е прието да се пише и чете от ляво на дясно, то множество математически операции се изпълняват от ляво на дясно. Това се нарича лява асоциативност и събирането е точно от тази група операции.

\begin{lstlisting}[caption=Израз за каскадно присвояване, label=listing0004]
a = b = 2
\end{lstlisting}

В гимназиалната математика символът равно се използва за проверка на идентичността между двата операнда, но в компютърните езици символът за равенство има смисъл на операция за присвояване. Това означава, че десният операнд бива присвоен като стойност на левия операнд. При съставянето на математически израз с каскада от присвоявания няма друг вариант освен първо най-десният операнд да бъде изпълнен и едва накрая най-левият. Това се нарича дясна асоциативност и се използва при значително малък брой от математическите операции. 

\begin{lstlisting}[caption=Контекстна зависимост на опрациите, label=listing0005]
"abc" + "def"
2 + 2
\end{lstlisting}

Следващата важна характеристика на математическите операции е тяхната контекстна зависимост. В множество езици събирането на символни низове води до конкатенация (не и в R), докато събирането на числа води до резултат от числено събиране (Листинг \ref{listing0005}). 

\begin{lstlisting}[caption=Контекстна зависимост на опрацията за делене, label=listing0006]
5 / 3
5.0 / 3.0
\end{lstlisting}

В множество програмни езици (не и в R) операцията за делене е контекстно зависима (Листинг \ref{listing0006}). Когато и двата операнда са цели числа, резултатът е целочислено деление, а когато поне един от операндите е дробно число, то резултатът е дробно число. 

\begin{lstlisting}[caption=Приоритет на операциите, label=listing0007]
2 + 2 * 2
\end{lstlisting}

Когато в един математически израз участват повече от една операции с еднаква асоциативност от значение става приоритетът на всяка от тях. Най-често даваният пример е събирането и умножението (Листинг \ref{listing0007}). В случая първо се извършва умножението, тъй като е по-високо приоритетно, а едва след това събирането. 

\begin{lstlisting}[caption=Смяна на приоритета, label=listing0008]
(2 + 2) * 2
\end{lstlisting}

В компютърните езици кръглите скоби имат смисъла на операция за промяна на реда по който ще се извърши пресмятането с цел смяна на приоритета (Листинг \ref{listing0008}).

Последният съществен признак на операциите е в коя група попадат – аритметични, логически, побитови, за сравнение, за присвояване и други.

\section{Типове и променливи}

Повечето съвременни програмни езици организират работата с информация в групи от променливи. За разлика от строго типизираните езици в R не се задава тип на променливата. Типът на променливата неявно се определя от стойността, която е присвоена към нея. Това позволява да се присвояват дори обекти или функции. Това означава, че една и съща променлива може да съдържа данни от различни типове в различни моменти от времето. 

Променливата се появява в общата памет веднага след първата операция за присвояване, за което съществуват цяла група операции за присвояване (Листинг \ref{listing0009}). Променливите в R могат да съдържат в имената си латинските букви и арабските цифри, също символът точка (.) и подчертавка (\_). Имената на променливите не могат да започват с цифра или с подчертавка и са чувствителни към малки/големи букви. 

\begin{lstlisting}[caption=Операции за присвояване, label=listing0009]
a = 1
b <- 2
c = d = 3
e <- f <- 4
assign("g", 5)
a += 6
b -= 7
\end{lstlisting}

Стрелка на ляво (<-) служи за присвояване в R, но в повечето конвенционални програмни езици не присъства. 

\begin{lstlisting}[caption=Алтернативи за операцията присвояване, label=listing0010]
median(x = 1:10)
median(x <- 1:10)
\end{lstlisting}

Разликата между двете операции си проличава най-ясно при викането на функции с аргументи (Листинг \ref{listing0010}). В първия случай променливата x не остава в глобалната памет, а изчезва, докато при втория случай променливата x остава в глобалната памет, след извикването. 

Добра практика е за имената на променливите да се избират съществителни имена, а не еднобуквени имена или съкращения. 

\begin{lstlisting}[caption=Премахване на променливи от глобалната памет, label=listing0011]
rm( a )
rm( list=ls() )
\end{lstlisting}

Премахването на променлива от общата памет става с командата rm (Листинг \ref{listing0011}). За да се почисти цялата глобална памет се дава списък с всички променливи, налични в глобалната памет. Въпреки че R извиква Garbage Collector-а на определени интервал от време с командата gc() може да бъде отправена пряка заявка за освобождаване на ненужно заетата памет. 

\newpage
\addcontentsline{toc}{chapter}{Заключение}
\chapter*{Заключение}
\thispagestyle{empty}

Без да претендира за изчерпателност настоящото учебно помагало прави въведение в статистическата обработка на данни с помощта на един от най-популярните програмни продукти, а именно програмния пакет $R$. В практическата работа на ученици, студенти, докторанти и специалисти по статистика се срещат множество особености, които до голяма степен са засегнати в изложения материал. Макар и да съществуват множество алтернативни програмни продукти, като $SPSS$, $Matlab$ и $Mathematica$, програмният продукт $R$ се отличава с финансова ефективност и отворен модел за разширяване. В учебното помагало не са засегнати темите за напреднали, тъй като целите на авторите са основно да провокират широката аудитория. Темите за напреднали могат да бъдат открити в множество учебници и книги в чуждоезичната литература, както и в голям брой видео уроци. 



% Списък с използвана литература и източници на информация.
\newpage
\begin{thebibliography}{99}
\addcontentsline{toc}{chapter}{Библиография}

\bibitem{gpl2} GNU General Public License, version 2, Free Software Foundation, \\\texttt{https://www.gnu.org/licenses/old-licenses/gpl-2.0.html}

\end{thebibliography}

% Азбучен указател на използваните термини.
\newpage
\printindex

\includepdf[pages=-,height=320mm]{images/back}
\end{document}
