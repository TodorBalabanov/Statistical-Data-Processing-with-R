\documentclass[runningheads,14pt,a4paper,openany]{book}

% Добавя възможност за сензитивни хипер-връзки в самия документ.
\usepackage[pdftex, bookmarks, linktocpage]{hyperref}
% Команда с множество опции за настройка на поведението на пакета hyperref, с най-полезната опция - кирилизация на заглавията от Bookmarks в Acrobat.
\hypersetup{unicode=true, colorlinks=true, linkcolor=black, citecolor=black, urlcolor=black}
\usepackage{nameref}
\usepackage[utf8x]{inputenc}
% Използва се за цвят на заглавните редове в таблиците.
\usepackage[table]{xcolor}
\usepackage[english,bulgarian]{babel}
\usepackage{shorttoc}
\usepackage[pdftex]{graphicx}
\usepackage{imakeidx}
\usepackage{placeins}
% Използва се за включване на кориците под формата на PDF файлове.
\usepackage{pdfpages}
% Служи за управление на заглавията.
\usepackage{fancyhdr}
\usepackage{listings}
% Служи за списък на формулите.
\usepackage{tocloft}
\usepackage{amsmath,amssymb,amsthm}

% Добавени от доц. Вера Ангелова.
%\usepackage{longtable}
%\usepackage{pifont}   
%\usepackage{epsfig}
%\usepackage{etoc}
%\usepackage{wasysym}
%\usepackage{eurosym}
%\usepackage{slashbox}
%\usepackage{soul}
%\usepackage{enumitem}
%\usepackage{mathcomp}
%\usepackage{epstopdf}
%\usepackage{latexsym}
%\usepackage{eucal}
%\usepackage{mathrsfs}

\title{Статистическа обработка на данни с R}
\author{Тодор Балабанов, Зорница Атанасова, Румен Кетипов}

% Директория в която се намират изображенията.
\graphicspath{{images/}}

\textheight 22.8cm
\textwidth 17cm
\oddsidemargin -0.54cm
\evensidemargin -0.54cm
\topmargin 1.5cm

\parskip=0.2cm
\parindent=20pt
\flushbottom

% Добавя черти под хедъра и над футъра.

\selectlanguage{bulgarian}

% Оформление на хедъра и футъра.
\lhead[\thepage \quad Тодор Балабанов, Зорница Атанасова, Румен Кетипов \quad \hfill]{}
\rhead[]{\hfill Статистическа обработка на данни с R \quad \thepage }
\cfoot[\em Лекции по компютърни науки и технологии на ИИКТ - БАН, № 9, 2019]{\em Лекции по компютърни науки и технологии на ИИКТ - БАН, № 9, 2019} 
\renewcommand{\headrulewidth}{1pt}
\renewcommand{\footrulewidth}{1pt}

\onecolumn
\makeindex[columns=2, title=Азбучен указател, intoc]

% Подменя думата използван а за ноемрация на фрагментите програмен код.
\renewcommand{\lstlistingname}{Листинг}

% Определя характеристиките на листигните за програмния код.
\lstset{backgroundcolor=\color{gray!30}, breaklines=true, language=r, frame=single}

\begin{document}

\def\ql{\textquotedblleft}\def\qr{\textquotedblright}

\includepdf[pages={1,2}]{images/front}
\thispagestyle{empty}

\voffset =-1truecm

% Използва се за номерация на страниците.
\renewcommand{\thepage}{\roman{page}}

% Смяна на названието за списъка от листингите.
\renewcommand{\lstlistlistingname}{Списък на листингите}

% Смяна на названието за списъка от формули.
\newcommand{\listequationsname}{Списък на формулите}
\newlistof{listofequations}{equ}{\listequationsname}
\newcommand{\listofequations}[1] {
	\addcontentsline{equ}{listofequations}{\protect\numberline{\theequation}#1}\par
}

\setcounter{page}{-1}
\thispagestyle{empty}
\pagestyle{empty}
\thispagestyle{empty}

% Тук стои таблицата със съдържанието, което се генерира от названието на главите.
\newpage
\thispagestyle{empty}
\pagestyle{empty}
\shorttoc{Теми}{0}
\thispagestyle{empty}
\pagestyle{empty}

% Тук стои таблицата със съдържанието, което се генерира от названието на главите и названието на секциите в тях.
\newpage
\thispagestyle{empty}
\pagestyle{empty}
\thispagestyle{empty}
\tableofcontents
\thispagestyle{empty}
\pagestyle{empty}

% Списък с фигурите.
\newpage
\listoffigures
\addcontentsline{toc}{chapter}{Списък на фигурите}

% Списък с листингите.
\newpage
\lstlistoflistings
\addcontentsline{toc}{chapter}{Списък на листингите}

% Списък с формулите. 
\newpage
%\listofequations
\addcontentsline{toc}{chapter}{Списък на формулите}

% Списък с таблиците. 
\newpage
\listoftables
\addcontentsline{toc}{chapter}{Списък на таблиците}

\pagestyle{fancy}
\newpage
\addcontentsline{toc}{chapter}{Предговор}
\chapter*{Предговор}
\pagenumbering{arabic}
\setcounter{page}{1}
\pagestyle{fancyplain}

Това учебно помагало е предвидено за студенти и докторанти, които в своите магистърски или докторски тези се сблъскват с потребността да извършат определени експерименти, а след това да обработят статистически получените резултати. 

В съвременния живот нуждата от обработка на информация единствено нараства. В множество ситуации от ежедневния ни живот се налага да бъдат вземани решения. От своя страна, всяко решение е толкова по-успешно, колкото по-информирано е взето то. Статистическата обработка на събраната информация е една от основните за вземането на информирани решения. В областта на статистическата обработка съществуват множество софтуерни решения, като се започне от по-достъпните за хора без опит, като Microsoft Excel и се стигне до професионалните пакети, като SPSS, Matlab и Mathematica. 

Това учебно помагало представя програмния продукт R, който първоначално се разработва от Robert Gentleman и Ross Ihaka в University of Auckland през 1993 година. R е замислен като алтернатива на програмния продукт S, създаден от John Chambers, служител в Bell Labs. Първоначалният замисъл за R е инструмент, който да бъде използван в интерактивен режим, през командния ред. В последствие тази идея прераства в самостоятелен програмен език. Основното предназначение на R е обработка на данни, което включва въвеждане, пресмятане, визуализация на графики и отчети. 

Езикът получава значително по голяма популярност след 2000 година, като излиза от рамките на академичните среди и навлиза в финансовите среди, маркетинга, фармацията, социологията, психологията и в много други области. Най-често потребителите на R са хора с опит в програмни езици, като C/C++, Java, C\# или пък преди това са използвали други статистически пакети, като SAS, SPSS и дори Excel. Тези потребители дават значителен тласък в развитието на пакета R, добавяйки множество софтуерни приставки (add-ons). 

Въпреки че в някои случаи R се оказва стряскащ и дори смущаващ, особено за начинаещите потребители, с времето и с процеса по навлизане в материята овладяването му се улеснява и ускорява. Това учебно помагало представя информацията по един достъпен и олекотен начин за възприемане. Изложени са предимно най-важните аспекти от използването на пакета R, което от своя страна дава стабилна основа за бъдещо самостоятелно развитие на читателя. Материалът е съобразен със съдържанието на курса „Анализ на данни с R“, провеждан в „Център за обучение“ към „Българска академия на науките. Учебното помагало е организирано в следните глави.

Глава 1 - \nameref{chapter01}: Представя процеса по инсталиране и стартиране на програмния пакет.
\newpage
\chapter{Инсталация и стартиране}
\label{chapter01}

Тъй като програмният продукт R се разработва под формата на софтуер с отворен код, то употребата му с не търговска цел не изисква заплащане. За работа с R е достатъчно да се инсталира основният пакет, въпреки това съществува и интегрирана среда за разработка наречена R Studio. За нуждите на учебното помагало ще бъде използван само основният пакет. Всеки желаещ да разшири уменията си с използването на интегрираната среда за разработка, би могъл самостоятелно да разучи възможностите ѝ.

\section{Изтегляне на инсталационните файлове}

\begin{figure}[h!]
  \centering
  \includegraphics[width=1.0\linewidth]{pic0001}
  \caption{Начална уеб страница на продукта}
\label{fig:pic0001}
\end{figure}
\FloatBarrier

Както множество софтуерно продукти и R е достъпен за изтегляне от уеб страницата на продукта в Интернет (Фиг. \ref{fig:pic0001}) с адрес: http://www.r-project.org/ 

\begin{figure}[h]
  \centering
  \includegraphics[width=1.0\linewidth]{pic0002}
  \caption{Списък със сървъри за изтегляне}
\label{fig:pic0002}
\end{figure}
\FloatBarrier

В раздела за изтегляне са посочени множество активни връзки към различни географски локации (Фиг. \ref{fig:pic0002}). Обичайна практика, при софтуерните продукти с отворен код, е наличието на множество сървъри, разположени по цял свят, да предлагат изтегляне на файловете нужни за инсталацията. Тази практика се е наложила най-вече за ускоряване на изтеглянето, но също така и за намаляване на натоварването, което сървърите понасят при множество заявки. Не на последно място, много често за разпространението на инсталационните файлове се разчита на доброволческа информационна инфраструктура, за която не се заплаща. 

\begin{figure}[h]
  \centering
  \includegraphics[width=1.0\linewidth]{pic0003}
  \caption{Избор на подходящата за операционната система инсталация}
\label{fig:pic0003}
\end{figure}
\FloatBarrier

Често срещана практика е продуктите с отворен кода да се разпространяват за трите най-популярни операционни системи, този принцип е спазен и за продукта R (Фиг. \ref{fig:pic0003}). При комерсиалните софтуерни продукти много често се залага на една единствена операционна система, но при продуктите с отворен код идеологията е, че трябва да се достигне до възможно най-голям брой потребители и поради тази причина се полагат допълнителни усилия софтуерът да работи на възможно най-много платформи (платформа – комбинация между хардуер и операционна система). Тази стратегия за дълготрайно развитие залага и на следващата особеност в развитието на продуктите с отворен код, а именно, че една част от потребителите с времето се превръщат в хора добавящи програмен код към продуктите. Освен видът на операционната система, от значение е и размерът на машинната дума, която процесорът поддържа. Към настоящия момент, най-разпространени са изчислителните машини с 64 битова машинна дума, но тъй като все още има много техника, която работи на 32 битова машинна дума, продуктът R е достъпен и за двата варианта. 

\begin{figure}[h]
  \centering
  \includegraphics[width=1.0\linewidth]{pic0004}
  \caption{Избор на версия за изтегляне}
\label{fig:pic0004}
\end{figure}
\FloatBarrier

Добра практика е при работата със софтуерни продукти, които се разпространяват като отворен код, винаги да се използва най-новата стабилна версия. В случая, за операционната система Mac OS X (Фиг. \ref{fig:pic0004}), това е версията 3.5.1, която е налична под формата на инсталационне файл R-3.5.1.pkg (Фиг. \ref{fig:pic0003}).

\begin{figure}[h]
  \centering
  \includegraphics[width=1.0\linewidth]{pic0005}
  \caption{Запазване на инсталационния файл}
\label{fig:pic0005}
\end{figure}
\FloatBarrier

\section{Инсталация}

За всяка от операционните системи е достатъчно потребителят да следва инструкциите и инсталацията протича безпроблемно по указаните стъпки. 

\begin{figure}[h]
  \centering
  \includegraphics[width=1.0\linewidth]{pic0006}
  \caption{Активиране на инсталатора}
\label{fig:pic0006}
\end{figure}
\FloatBarrier

С двойно щракване на мишката се активира инсталатора (Фиг. \ref{fig:pic0006}). След което следва екран с подробности за самата инсталация (Фиг. \ref{fig:pic0007}).

\begin{figure}[h]
  \centering
  \includegraphics[width=1.0\linewidth]{pic0007}
  \caption{Подробности за инсталацията}
\label{fig:pic0007}
\end{figure}
\FloatBarrier

От съществено значение е потребителите на продукти с отворен код да са добре запознати с условията при които те получават продуктите, особено когато е без заплащане. Поради тази причина, потребителят трябва изрично да се съгласи с условията на лиценза под който се разпространява продуктът R (Фиг. \ref{fig:pic0008},\ref{fig:pic0009}).

\begin{figure}[h]
  \centering
  \includegraphics[width=1.0\linewidth]{pic0008}
  \caption{Лицензно споразумение за ползване}
\label{fig:pic0008}
\end{figure}
\FloatBarrier

Софтуерният продукт R се разпространява под отворен лиценз GPL2, който в най-общи линии очертава рамките на условията при които потребителите получават софтуера. Най-важните клаузи в лиценза са свързани с липса на гаранция и с изричното съгласие на потребителя, че производителят не носи никаква юридическа отговорност, произтекла от употребата на софтуерния продукт. Пълният текст на лицензът е достъпен в уеб страницата на фондацията, която го поддържа \cite{gpl2}.

\begin{figure}[h]
  \centering
  \includegraphics[width=1.0\linewidth]{pic0009}
  \caption{Изрично съгласие}
\label{fig:pic0009}
\end{figure}
\FloatBarrier

Съвременните операционни системи са от отворен тип и инсталацията на допълнителни софтуерни продукти се явяват своеобразно разширение на операционната система. Поради тази причина, всеки създател на операционна система е избрал правила и начини за добавяне на софтуерни продукти. Една от основните характеристики е определяне на директория във файловата система на операционната система, където новодобавеният софтуер ще бъде поставен. Някои операционни системи (примерно Linux базираните дистрибуции) използват специално организиран мениджър на пакетите, който има грижата за консистентността на добавяните софтуерни продукти. При Mac OS X също е налична възможността за автоматично управление на инсталациите, но е дадена и възможност потребителят да избира мястото на инсталация. В операционната система Microsoft Windows, потребителят решава къде да помести новоинсталирания софтуер. 

\begin{figure}[h]
  \centering
  \includegraphics[width=1.0\linewidth]{pic0010}
  \caption{Информация за директорията и използваното дисково пространство}
\label{fig:pic0010}
\end{figure}
\FloatBarrier

При желание е възможно да бъде подменена инсталационната директория (Фиг. \ref{fig:pic0010}).

\begin{figure}[h]
  \centering
  \includegraphics[width=1.0\linewidth]{pic0011}
  \caption{Успешно приключване на инсталацията}
\label{fig:pic0011}
\end{figure}
\FloatBarrier

Инсталацията приключва със съобщение за успешно изпълнение (Фиг. \ref{fig:pic0011}).

\section{Работа в режим на команди}

Инсталаторът създава икона за стартиране на R командния интерпретатор.

\begin{figure}[h]
  \centering
  \includegraphics[width=1.0\linewidth]{pic0012}
  \caption{Основен прозорец на продукта}
\label{fig:pic0012}
\end{figure}
\FloatBarrier

При успешна инсталация и стартиране, потребителят получава достъп до прозорец служещ като команден интерпретатор (Фиг. \ref{fig:pic0012}). Първоначалният замисъл за продуктът R е бил това да представлява интерпретатор на команди. В интерактивен режим потребителят въвежда команда и наблюдава получения резултат. Макар и това да е основният начин за работа, R позволява последователността от команди да бъде записана във файл и да се изпълняват като единен скрипт. За много потребители, особено такива с предишен опит в софтуерни продукти с силно застъпен графичен потребителски интерфейс (примерно статистически анализ на данни в Microsoft Excel), използването на терминал с команди първоначално е трудно и дори дразнещо, но с напредване на времето потребителите оценяват гъвкавостта която този начин на работа позволява. Използването на поредица от команди често се оказва значително по-бърз начин за работа в сравнение с подготовката на експеримент в софтуер с графичен интерфейс. Също така, наличието на серията команди дава възможност значително по-бързо експериментът да бъде преповторен. Често при по-голям обем данни софтуерните пакети с графичен потребителски интерфейс забавят работата си неприемливо много. И не на последно място, наличието на статистическия модел като скрипт позволява използването на системи за контрол на версиите (каквато примерно е Git и облачната услуга GitHub), нещо което търговските бинарни файлови формати (примерно XLSX, в Microsoft Excel) не позволяват. При работата с командния интерпретатор на R, клавишът „стрелка нагоре“ повтаря последната използвана команда. Списъкът с вече изпълнени команди може да бъде преминат със стрелките нагоре и надолу. Тъй като интерпретацията на всяка команда става в момента на нейното повикване е възможно да бъде стартиран код, който да отнема твърде дълго време или да изпадне в безкраен цикъл. При тази ситуация, натискането на клавиша Esc или клавишната комбинация Ctrl+C прекъсва текущо изпълняваната команда. 

\begin{figure}[h]
  \centering
  \includegraphics[width=1.0\linewidth]{pic0013}
  \caption{Изпълнение на команда за печат}
\label{fig:pic0013}
\end{figure}
\FloatBarrier

При правилно работеща инсталация изписването на командата за печат би дала резултата показан на Фиг. \ref{fig:pic0013}. Както показва първата примерна команда, има фундаментална разлика в начина по който работят компилаторите и интерпретаторите. Езици като C/C++ изискват компилация на програмния код до машинни инструкции, които процесорът на компютъра изпълнява. При интерпретивните езици, като PHP, Python, JavaScript и разбира се R, всяка стъпка от програмата се интерпретира в момента на повикването й. По-напредналите потребители на R са добре запознати с начина по който са изградени библиотеките на езика и знаят добре, че съществуват възможности код писан на C/C++ да бъде изпълняван в средата на R. Такава межуезикова връзка най-често се налага при голям обем данни за пресмятане и относително бавни алгоритми, които извършват пресмятането. 

\newpage
\chapter{Пакетна организация и основни операции в R}
\label{chapter02}

Най-голямата сила на продукта R се дължи на хилядите пакети\index{пакети} (софтуерни приставки), създадени от безброй потребители на продукта. Наличните пакети покриват цялата област на статистиката и статистическата обработка на данни. 

Под пакет се разбира софтуерна библиотека от предварително написан програмен кой, който има за цел да реши определена задача или група от задачи. Тъй като продуктът R е една отворена система е важно да се има предвид, че не всички пакети са с еднакво качество. Една част от пакетите са изключително професионално написани, устойчиви са на некоректно използване и имат добра база от поддържащи ги потребители. В същото време друга част от пакетите са създадени с голяма доза добри намерения, но работят бавно, дават дефекти или просто не вършат това за което са създадени. Голяма част от пакетите са написани от статистици за статистици и това може да доведе до някои странни въпроси при част от потребителите, особено при хора идващи от индустрията за производство на софтуер. 

Настоящото учебно помагало представя само най-основните пакети, достатъчни да бъде изложен материалът свързан с базовите познания по R. Опит да бъдат представени всички пакети е непосилен за едно издание, най-вече защото броят и видът на пакетите постоянно се променя. 

\section{Инсталиране на пакети}

Съществуват различни начини за инсталиране на пакети в R, но най-основният от тях е чрез команда в конзолата на пакета R. 

\begin{figure}[h!]
  \centering
  \includegraphics[width=1.0\linewidth]{pic0014}
  \caption{Команда за инсталиране на пакета coefplot}
\label{figure0014}
\end{figure}
\FloatBarrier

За да започне инсталирането на пакет (в случая coefplot) е достатъчно да се изпише командата от Фиг. \ref{figure0014}.

\begin{figure}[h!]
  \centering
  \includegraphics[width=1.0\linewidth]{pic0015}
  \caption{Избор на сървър за изтегляне на пакета}
\label{figure0015}
\end{figure}
\FloatBarrier

След което следва избор на сървър за изтегляне на пакета (Фиг. \ref{figure0015}). Разумна стратегия е да се избират сървъри, които териториално се намират в близост до мястото от което се работи. Това би осигурило малко по-голяма бързина на връзката в Глобалната мрежа. 

\begin{figure}[h!]
  \centering
  \includegraphics[width=1.0\linewidth]{pic0016}
  \caption{Зависимости между пакетите}
\label{figure0016}
\end{figure}
\FloatBarrier

Често срещан случай е един пакет да има функционална зависимост от други пакети (Фиг. \ref{figure0016}). В такава ситуация е необходимо всички нужни пакети също да бъдат инсталирани. Стратегията при разработка на пакети е те да бъдат предлагани в компилиран (бинарен) вид, но понякога най-новите версии са под формата на програмен код и тогава потребителят има възможност да избере между бинарната версия или версията с програмен код. 

\begin{figure}[h!]
  \centering
  \includegraphics[width=1.0\linewidth]{pic0017}
  \caption{Резултат от инсталацията на пекета}
\label{figure0017}
\end{figure}
\FloatBarrier

Инсталацията на пакета приключва с подробен листинг, съдържащ описание на извършените операции (Фиг. \ref{figure0017}). 

\begin{figure}[h!]
  \centering
  \includegraphics[width=1.0\linewidth]{pic0018}
  \caption{Зареждане на пакета coefplot}
\label{figure0018}
\end{figure}
\FloatBarrier

Дали пакетът е надлежно инсталиран може да се провери с командата require (Фиг. \ref{figure0018}), която зарежда пакета в паметта. 

Съществува възможност пакетите да се инсталират под формата на програмен код, директно от хранилищата за програмен код, но за тази цел са нужни подходящите компилатори (най-често C/C++ и Fortran), както и по-задълбочени умения по програмиране. В редки случаи се налага инсталиране на пакета от ZIP файл. При такава ситуация е важно предварително да бъдат инсталирани всички пакети от които инсталирания пакет зависи. 

Премахване на инсталирани пакети става с помощта на командата remove.packages на която се подава вектор с имената на пакетите, които трябва да бъдат премахнати.

\section{Зареждане на пакети}

За да бъдат използвани пакетите не е достатъчно те да бъдат инсталирани, но трябва с команда да бъдат включени в текущата сесия от изчисления. R предлага две команди за зареждане на пакети – library и require. И двете изпълняват едно и също нещо – зареждат пакета в общата памет. Разликата е, че require връща TRUE, ако зареждането е било успешно и FALSE при неуспех. Тази възможност е полезна в редките случаи, когато пакетът се зарежда от програмния текст на функция. Подобна практика не е препоръчителна, но R дава такава възможност. И двете функции получават като параметър името на пакета, със или без кавички. Пакетите се зареждат еднократно и остават налични през цялата сесия от изчисления или докато изрично не бъдат премахнати от общата памет. 

\begin{figure}[h!]
  \centering
  \includegraphics[width=1.0\linewidth]{pic0019}
  \caption{Премахване на пакета coefplot от общата памет}
\label{figure0019}
\end{figure}
\FloatBarrier

Премахването на пакет от общата памет става с командата detach (Фиг. \ref{figure0019}). Същественото при тази команда е, че преди името на пакета се записва думата package. 

Тъй като пакетите се разработват основно на доброволни начала не рядко се случва в различни пакети да има едноименни функции. При подобна колизия на имената решението е операцията за принадлежност – двойно двуеточие (::). Когато бъде използвана операцията за принадлежност дори може да не се зарежда пакетът към който принадлежи функцията. 

\section{Основни математически операции}

R позволява да се извършват сложни математически пресмятания, но също така може да се използва и за базови математически сметки. 

\begin{figure}[h!]
  \centering
  \includegraphics[width=1.0\linewidth]{pic0020}
  \caption{Примерни аритметични операции}
\label{figure0020}
\end{figure}
\FloatBarrier

Най-базовите математически операции са събирането, изваждането, умножението и делението. Тези операции се изпълняват в R както е показано на Фиг. \ref{figure0020}. Пресмятанията в R се състоят от операции и операнди. Когато няколко операции бъдат обединени, чрез операндите си, се получава математически израз. 

\begin{lstlisting}[caption=Събиране, label=listing0001]
1 + 1
\end{lstlisting}

В листинг \ref{listing0001} е демонстрирана операцията за събиране, която има два операнда. Когато става въпрос за математически операции, те имат серия свойства. Като най-съществена характеристика може да се отбележи броят на операндите. Събирането е класически пример за бинарна операция, тъй като има два операнда (ляв и десен). 

\begin{lstlisting}[caption=Унарен минус, label=listing0002]
-5
\end{lstlisting}

В гимназиалния курс по математика не се споменава наличието на унарен плюс, макар и да се учи за унарен минус (Листинг \ref{listing0002}). Унарният плюс и минус променят значението на операнда. В примера от листинг \ref{listing0002}, унарният минус променя значението на числото пето от положително към отрицателно. Тъй като унарният плюс не бил променил значението на операнда си, масовата практика е унарничт плюс да не се записва, нещо което не е възможно с унарния минус. Освен унарни и бинарни операции в някои езици (примерно C/C++, Java, C\#, PHP и други) съществува една единствена тернарна операция (?:), която има смисъла на условния оператор за преход if. 

Както бе споменато по-горе, комбинацията от няколко операции и техните операнди водят до съставянето на математически израз (Листинг \ref{listing0003}). 

\begin{lstlisting}[caption=Аритметичен израз с две събирания, label=listing0003]
2 + 2 + 2
\end{lstlisting}

Тъй като съвременните изчислителни машини са организирани по такъв начин, че процесорът да извършва само една математическа операция на един такт от пресмятането, става актуален въпросът коя от операциите ще бъде изпълнена първа и коя втора, при положение, че операциите са с еднакъв приоритет. Тъй като в англо-саксонската писмена система е прието да се пише и чете от ляво на дясно, то множество математически операции се изпълняват от ляво на дясно. Това се нарича лява асоциативност и събирането е точно от тази група операции.

\begin{lstlisting}[caption=Израз за каскадно присвояване, label=listing0004]
a = b = 2
\end{lstlisting}

В гимназиалната математика символът равно се използва за проверка на идентичността между двата операнда, но в компютърните езици символът за равенство има смисъл на операция за присвояване. Това означава, че десният операнд бива присвоен като стойност на левия операнд. При съставянето на математически израз с каскада от присвоявания няма друг вариант освен първо най-десният операнд да бъде изпълнен и едва накрая най-левият. Това се нарича дясна асоциативност и се използва при значително малък брой от математическите операции. 

\begin{lstlisting}[caption=Контекстна зависимост на опрациите, label=listing0005]
"abc" + "def"
2 + 2
\end{lstlisting}

Следващата важна характеристика на математическите операции е тяхната контекстна зависимост. В множество езици събирането на символни низове води до конкатенация (не и в R), докато събирането на числа води до резултат от числено събиране (Листинг \ref{listing0005}). 

\begin{lstlisting}[caption=Контекстна зависимост на опрацията за делене, label=listing0006]
5 / 3
5.0 / 3.0
\end{lstlisting}

В множество програмни езици (не и в R) операцията за делене е контекстно зависима (Листинг \ref{listing0006}). Когато и двата операнда са цели числа, резултатът е целочислено деление, а когато поне един от операндите е дробно число, то резултатът е дробно число. 

\begin{lstlisting}[caption=Приоритет на операциите, label=listing0007]
2 + 2 * 2
\end{lstlisting}

Когато в един математически израз участват повече от една операции с еднаква асоциативност от значение става приоритетът на всяка от тях. Най-често даваният пример е събирането и умножението (Листинг \ref{listing0007}). В случая първо се извършва умножението, тъй като е по-високо приоритетно, а едва след това събирането. 

\begin{lstlisting}[caption=Смяна на приоритета, label=listing0008]
(2 + 2) * 2
\end{lstlisting}

В компютърните езици кръглите скоби имат смисъла на операция за промяна на реда по който ще се извърши пресмятането с цел смяна на приоритета (Листинг \ref{listing0008}).

Последният съществен признак на операциите е в коя група попадат – аритметични, логически, побитови, за сравнение, за присвояване и други.

\section{Типове и променливи}

Повечето съвременни програмни езици организират работата с информация в групи от променливи. За разлика от строго типизираните езици в R не се задава тип на променливата. Типът на променливата неявно се определя от стойността, която е присвоена към нея. Това позволява да се присвояват дори обекти или функции. Това означава, че една и съща променлива може да съдържа данни от различни типове в различни моменти от времето. 

Променливата се появява в общата памет веднага след първата операция за присвояване, за което съществуват цяла група операции за присвояване (Листинг \ref{listing0009}). Променливите в R могат да съдържат в имената си латинските букви и арабските цифри, също символът точка (.) и подчертавка (\_). Имената на променливите не могат да започват с цифра или с подчертавка и са чувствителни към малки/големи букви. 

\begin{lstlisting}[caption=Операции за присвояване, label=listing0009]
a = 1
b <- 2
c = d = 3
e <- f <- 4
assign("g", 5)
a += 6
b -= 7
\end{lstlisting}

Стрелка на ляво (<-) служи за присвояване в R, но в повечето конвенционални програмни езици не присъства. 

\begin{lstlisting}[caption=Алтернативи за операцията присвояване, label=listing0010]
median(x = 1:10)
median(x <- 1:10)
\end{lstlisting}

Разликата между двете операции си проличава най-ясно при викането на функции с аргументи (Листинг \ref{listing0010}). В първия случай променливата x не остава в глобалната памет, а изчезва, докато при втория случай променливата x остава в глобалната памет, след извикването. 

Добра практика е за имената на променливите да се избират съществителни имена, а не еднобуквени имена или съкращения. 

\begin{lstlisting}[caption=Премахване на променливи от глобалната памет, label=listing0011]
rm( a )
rm( list=ls() )
\end{lstlisting}

Премахването на променлива от общата памет става с командата rm (Листинг \ref{listing0011}). За да се почисти цялата глобална памет се дава списък с всички променливи, налични в глобалната памет. Въпреки че R извиква Garbage Collector-а на определени интервал от време с командата gc() може да бъде отправена пряка заявка за освобождаване на ненужно заетата памет. 

\newpage
\chapter{Сложни структури от данни и викане на функции}
\label{chapter03}
\thispagestyle{empty}

\section{Извикване на функции}

Функциите са последователност от инструкции, обособени като едно цяло, така че да са подходящи за многократно извикване. Функциите приемат входящи параметри, могат да имат върната стойност, а символът диез (\#) се използва в началото на ред за коментар. Организацията на програмния текст във функции позволява лесна четимост и по-лесно откриване на програмни дефекти (бъгове). По своята същност командите в конзолата на R са функции, които потребителят извиква. Поради този факт е важно да се знаят възможностите за работа с функции\index{извикване на функции}. За разлика от масово наложените езици за обектно-ориентирано програмиране, в R функциите са по-съществени от обектите.

\begin{lstlisting}[caption=Извикване на функции, label=listing0029]
x <- c(1, 2, 3, 5, 6, 7, 8, 9)
x
[1] 1 2 3 5 6 7 8 9

mean( x )
[1] 5.125

median( x )
[1] 5.5

sd( x )
[1] 2.900123
\end{lstlisting}

Листинг \ref{listing0029} демонстрира извикването на три функции, които получават като единствен аргумент вектор от числени стойности. По-сложните функции може да имат повече аргументи\index{аргументи на функции} и те да се подават по различен начин.

Всяка функция, която е достъпна в R, има съпровождаща документация\index{документация на функции}, но качеството на тази документация може да варира според уменията на автора ѝ. Най-бързият начин за достъп до информацията е чрез поставяне на въпросителен (?) пред името на функцията (Листинг \ref{listing0030}).

\begin{lstlisting}[caption=Документация за функциите, label=listing0030]
? mean
?? mean
? median
?? median
? sd
?? sd
\end{lstlisting}

Един въпросителен отваря информацията в локален прозорец, а два въпросителни отварят уеб страницата, съдържаща документацията на функцията.

\begin{lstlisting}[caption=Документация за операции, label=listing0031]
? `+`
? `-`
? `*`
? `/`
\end{lstlisting}

За голяма част от операциите също може да се получи информация по сходен начин (Листинг \ref{listing0031}), но операцията\index{документация на операции} трябва да бъде оградена със символа апостроф (\`).

\begin{lstlisting}[caption=Частично търсене, label=listing0032]
apropos( "med" )
[1] "elNamed"        "elNamed<-"      "median"         "median.default"
[5] "medpolish"      "runmed"
\end{lstlisting}

Често потребителите имат идея каква функция търсят, но не се досещат за точното изписване на името ѝ. В такива ситуации е полезна възможността за частично търсене\index{частично търсене}, която предоставя функцията apropos (Листинг \ref{listing0032}).

\section{Вектори}

Векторът\index{вектори} е колекция от елементи, които са от един и същи тип (Листинг \ref{listing0021}). 

\begin{lstlisting}[caption=Вектор от числа и вектор от символни низове, label=listing0021]
v1 <- c(1, 3, 2, 1, 5)
v2 <- c("Peter", "Ivan", "Geroge")
\end{lstlisting}

Векторите в R имат значителна роля за езика, тъй като R е векторизиран език, което го прави различен от конвенционалните програмни езици, като C/C++, C\# или Java. Това означава, че всяка математическа операция се изпълнява върху целия вектор и не е нужно да се обикалят отделните елементи един по един (Листинг \ref{listing0022}). За разлика от математическата концепция, в R векторите не се делят на вектор-стълб или вектор-ред. При нужда от вектор-ред или вектор-стълб може да се използват матрици с единична стойност на един от размерите.

\begin{lstlisting}[caption=Базови операции над вектори, label=listing0022]
x <- c(1, 2, 3, 4, 5, 6, 7, 8, 9, 10)
x
[1]  1  2  3  4  5  6  7  8  9 10

x * 5
[1]  5 10 15 20 25 30 35 40 45 50

x + 3
[1]  4  5  6  7  8  9 10 11 12 13

x - 4
[1] -3 -2 -1  0  1  2  3  4  5  6

x / 5
[1] 0.2 0.4 0.6 0.8 1.0 1.2 1.4 1.6 1.8 2.0

x ^ 3
[1]    1    8   27   64  125  216  343  512  729 1000

sqrt( x )
[1] 1.000000 1.414214 1.732051 2.000000 2.236068 2.449490 2.645751 2.828427
[9] 3.000000 3.162278
\end{lstlisting}

Основният начин за създаване на вектор е чрез функцията c, като названието ѝ идва от combine (комбиниране на елементи), но също е възможно да се използва и алтернативен запис (Листинг \ref{listing0023}).

\begin{lstlisting}[caption=Алтернативен синтаксис за създаване на вектори, label=listing0023]
1:10
[1] 1 2 3 4 5 6 7 8 9 10

10:1
[1] 10 9 8 7 6 5 4 3 2 1

-2:3
[1] -2 -1 0 1 2 3

5:-7
[1] 5 4 3 2 1 0 -1 -2 -3 -4 -5 -6 -7
\end{lstlisting}

Когато двата операнда на операцията са вектори\index{операции с вектори} с еднакви дължини, то операцията се прилага на всичките елементи по двойки (Листинг \ref{listing0024}).

\begin{lstlisting}[caption=Операции между вектори с еднаква дължина, label=listing0024]
x <- 1:10
y <- -10:-1

nchar( x )
[1] 1 1 1 1 1 1 1 1 1 2

nchar( y )
[1] 3 2 2 2 2 2 2 2 2 2

x + y
[1] -9 -7 -5 -3 -1  1  3  5  7  9

x - y
[1] 11 11 11 11 11 11 11 11 11 11

x * y
[1] -10 -18 -24 -28 -30 -30 -28 -24 -18 -10

x / y
[1]  -0.1000000  -0.2222222  -0.3750000  -0.5714286  -0.8333333  -1.2000000
[7]  -1.7500000  -2.6666667  -4.5000000 -10.0000000

x ^ y
[1] 1.000000e+00 1.953125e-03 1.524158e-04 6.103516e-05 6.400000e-05
[6] 1.286008e-04 4.164931e-04 1.953125e-03 1.234568e-02 1.000000e-01

x > y
[1] TRUE TRUE TRUE TRUE TRUE TRUE TRUE TRUE TRUE TRUE
\end{lstlisting}

Когато векторите са с различна дължина, по-късият вектор се превърта и се започва от началото му. Функцията nchar показва колко символа са необходими за изписването на всеки от елементите.

\begin{lstlisting}[caption=Проверка дали някоя или всички стойности от вектора отговарят на определено условие, label=listing0025]
any(x+y < 0)
[1] TRUE

all(x+y < 0)
[1] FALSE
\end{lstlisting}

При група от проверки може да се установи дали всички елементи на вектора изпълняват определено условие или поне някои елементи го изпълняват (Листинг \ref{listing0025}).

\begin{lstlisting}[caption=Достъп до отделни елементи във вектор, label=listing0026]
x[ 1 ]
[1] 1

x[ 2:3 ]
[1] 2 3

x[ c(2,5,7) ]
[1] 2 5 7
\end{lstlisting}

Достъп до отделни елементи във вектор може да се осъществи по индекс или множество от индекси (Листинг \ref{listing0026}).

\begin{lstlisting}[caption=Имена на елементите във вектора, label=listing0027]
z <- c(One=1, Two=2, Three=3)
z
  One   Two Three 
    1     2     3 

names( z )
[1] "One"   "Two"   "Three"
\end{lstlisting}

R позволява на елементите във вектора да се поставят имена (Листинг \ref{listing0027}).

\begin{lstlisting}[caption=Трансформация на вектор във фактор, label=listing0028]
e <- c("High School", "College", "Masters", "Doctorate")
e

[1] "High School" "College"     "Masters"     "Doctorate"  
f1 <- as.factor( e )
f1
[1] High School College     Masters     Doctorate  
Levels: College Doctorate High School Masters

as.numeric( f1 )
[1] 3 1 4 2

f2 <- factor(c("High School", "College", "Masters", "Doctorate"), 
		 levels=c("High School", "College", "Masters", "Doctorate"),
		 ordered=TRUE)
f2
[1] High School College     Masters     Doctorate  
Levels: High School < College < Masters < Doctorate

as.numeric( f2 )
[1] 1 2 3 4
\end{lstlisting}

Вектор може да бъде трансформиран във фактор\index{фактори} с помощта на функции за трансформация (Листинг \ref{listing0028}). Факторът е тип данни, при който всяка стойност се среща само по един път. Някои множества са неподредени (както е f1) и при тях няма значение редът на елементите, докато при подредените множества (както е f2) редът на елементите има значение . Функцията factor позволява изрично да се зададе какъв е редът на елементите. Факторът е значително по-икономичен на памет от вектора, тъй като се запазват единствено числените стойности на отделните елементи, но пък използването му може да доведе до трудни за откриване логически грешки.

\begin{lstlisting}[caption=Вектори с латинските букви, label=listing0051]
letters
 [1] "a" "b" "c" "d" "e" "f" "g" "h" "i" "j" "k" "l" "m" "n" "o" "p" "q" "r"
[19] "s" "t" "u" "v" "w" "x" "y" "z"

LETTERS
 [1] "A" "B" "C" "D" "E" "F" "G" "H" "I" "J" "K" "L" "M" "N" "O" "P" "Q" "R"
[19] "S" "T" "U" "V" "W" "X" "Y" "Z"
\end{lstlisting}

В R има два специално предефинирани вектора, които съдържат буквите от латинската азбука (Листинг \ref{listing0051}).

\section{Липсващи стойности}

Липсващи стойности\index{липсващи стойности} в данните е ежедневен проблем за хората обработващи статистическа информация. Причините за липсващите данни могат да произлизат от различни обстоятелства, например пропуснато измерване или дефектирал датчик. R дава две възможности за обозначаване на липсващи стойности в данните (NA и NULL). Макар да имат сходно значение, тези две стойности водят до различни резултати при различните пресмятания.

Когато има липсващи стойности в данните съществуват множество начини този факт да бъде отразен. В някои комплекти данни се записва недопустима числена стойност или се използва някаква символна комбинация. В езика R е възприето липсващите стойности да се обозначават с NA (Листинг \ref{listing0033}).

\begin{lstlisting}[caption=Липсващи стойности, label=listing0033]
x <- c(1, NA, 3, NA, 5)
x
[1]  1 NA  3 NA  5

is.na( x )
[1] FALSE  TRUE FALSE  TRUE FALSE

mean( x )
[1] NA

median( x )
[1] NA

sd( x )
[1] NA
\end{lstlisting}

Значението на NULL е липса, а не изпусната стойност, поради тази причина векторът се редуцира с толкова елементи, колкото NULL стойности има в него (Листинг \ref{listing0034}).

\begin{lstlisting}[caption=Липсващи стойности, label=listing0034]
y <- c(1, NULL, 3, NULL, 5)
y
[1] 1 3 5

is.na( y )
[1] FALSE FALSE FALSE

mean( y )
[1] 3

median( y )
[1] 3

sd( y )
[1] 2

is.null( y )
[1] FALSE
\end{lstlisting}

Тъй като на практика векторът се редуцира с броя на NULL стойностите си, то функцията is.null не е векторизирана, а се отнася за целия обект.

\section{Рамкирани данни}

Рамкираните данни\index{рамкирани данни} (data.frame) са една от най-полезните структури от данни в езика R. Най-интуитивната аналогия за рамкирани данни е един лист (data sheet) в Microsft Excel, състоящ се от колони и редове. В термините на статистиката, всяка колона е наблюдавана променлива, а всеки ред е едно конкретно наблюдение (измерване). В термините на R, всяка колона е вектор, а дължината на всичките вектори е една и съща. По този начин всяка колона може да съдържа различни типове данни. Също така, в рамките на една колона всички елементи са от един и същи тип.

Съществуват множество начини да се създадат рамкирани данни, но най-лесният е с функцията data.frame.

\begin{lstlisting}[caption=Създаване на рамкирани данни, label=listing0035]
x <- sample(1:5)
x
[1] 4 1 3 2 5

y <- sample(-2:2)
y
[1]  0 -2  1 -1  2

q <- c("Football", "Basketball", "Volleyball", "Handball", "Rugby")
q
[1] "Football"   "Basketball" "Volleyball" "Handball"   "Rugby"

df1 <- data.frame(x, y, q)
df1
  x  y          q
1 4  0   Football
2 1 -2 Basketball
3 3  1 Volleyball
4 2 -1   Handball
5 5  2      Rugby
\end{lstlisting}

Листинг \ref{listing0035} демонстрира създаването на рамкирани данни от два вектора с числа (sample служи за разбъркване на стойностите по случаен начин) и един вектор със символни низове. Така получената структура е с размери 5x3 и се състои от три вектора. Имената на колоните се вземат служебно, но е възможно те да бъдат определени при създаването на самата структура (Листинг \ref{listing0036}).

\begin{lstlisting}[caption=Създаване на рамкирани данни с имена на колоните, label=listing0036]
df2 <- data.frame(First=x, Second=y, Sport=q)
rownames( df2 ) <- c("One", "Two", "Three", "Four", "Five")
df2
      First Second      Sport
One       4      0   Football
Two       1     -2 Basketball
Three     3      1 Volleyball
Four      2     -1   Handball
Five      5      2      Rugby
\end{lstlisting}

Рамкираните данни имат множество атрибути, като най-съществените са броя редове и броя колони (Листинг \ref{listing0037}). Атрибутите имат съществено значение при прилагането на различните алгоритми за статистически анализ.

\begin{lstlisting}[caption=Атрибути на рамкираните данни, label=listing0037]
nrow( df1 )
[1] 5

ncol( df1 )
[1] 3

dim( df1 )
[1] 5 3

names( df2 )
[1] "First"  "Second" "Sport"

rownames( df2 )
[1] "One"   "Two"   "Three" "Four"  "Five" 

head(df1, n=3)
  x  y          q
1 4  0   Football
2 1 -2 Basketball
3 3  1 Volleyball

tail(df1, n=3)
  x  y          q
3 3  1 Volleyball
4 2 -1   Handball
5 5  2      Rugby

class( df1 )
[1] "data.frame"
\end{lstlisting}

Също така, може да се проверят имената на колоните и имената на редовете, с функцията head за първите няколко реда, а с функцията tail за последните няколко реда.

\begin{lstlisting}[caption=Фактори в рамковите данни, label=listing0038]
df2[1, 2]
[1] 0

df2[3, 2:3]
      Second      Sport
Three      1 Volleyball

df2$Sport
[1] Football   Basketball Volleyball Handball   Rugby     
Levels: Basketball Football Handball Rugby Volleyball

class( df2$Sport )
[1] "factor"

df2$Sport[1:2]
[1] Football   Basketball
Levels: Basketball Football Handball Rugby Volleyball

df2[3, ]
      First Second      Sport
Three     3      1 Volleyball

df2[, c("First", "Sport")]
      First      Sport
One       4   Football
Two       1 Basketball
Three     3 Volleyball
Four      2   Handball
Five      5      Rugby
\end{lstlisting}

Рамкираните данни позволяват достъп до елементите като индекси\index{достъп по индекс} на двумерен масив или директно с адресиране на конкретна колона (Листинг \ref{listing0038}). Достъпът до цял ред става без указване на колона. За достъп до колоните по име се съставя вектор с имената на колоните.

\begin{lstlisting}[caption=Вътрешно представяне на факторите, label=listing0039]
f1 <- factor( c("Sofia", "Plovdiv", "Varna", "Burgas", "Ruse") )
f1
[1] Sofia   Plovdiv Varna   Burgas  Ruse   
Levels: Burgas Plovdiv Ruse Sofia Varna

model.matrix(~f1 - 1)
  f1Burgas f1Plovdiv f1Ruse f1Sofia f1Varna
1        0         0      0       1       0
2        0         1      0       0       0
3        0         0      0       0       1
4        1         0      0       0       0
5        0         0      1       0       0
attr(,"assign")
[1] 1 1 1 1 1
attr(,"contrasts")
attr(,"contrasts")$f1
[1] "contr.treatment"
\end{lstlisting}

Факторите са малко по-различни от векторите, за да се проследи вътрешното им представяне в рамкираните данни може да се приложи функцията model.matrix (Листинг \ref{listing0039}).

\section{Списъци}

В някои ситуации е нужно да се ползва контейнер, който да съдържа обекти от различни типове. В R това се постига със списъчните структури. Този тип структури могат да съдържат голям брой и различни по тип елементи. Списъците се създават с функцията list (Листинг \ref{listing0040}).

\begin{lstlisting}[caption=Създаване на списък, label=listing0040]
l1 <- list(1, 2, 3, 4, 5)
l1
[[1]]
[1] 1

[[2]]
[1] 2

[[3]]
[1] 3

[[4]]
[1] 4

[[5]]
[1] 5
\end{lstlisting}

Всеки елемент в списъка е самостоятелен (Листинг \ref{listing0040}), но е възможно да има и списък с единствен елемент, който е вектор (Листинг \ref{listing0041}).

\begin{lstlisting}[caption=Вектор в списък, label=listing0041]
l2 <- list( c(1, 2, 3, 4, 5) )
l2
[[1]]
[1] 1 2 3 4 5
\end{lstlisting}

Разнородни елементи на списък са показани в Листинг \ref{listing0042}.

\begin{lstlisting}[caption=Списък с разнородни данни, label=listing0042]
l3 <- list( df2, 1:5, l1 )
l3
[[1]]
      First Second      Sport
One       4      0   Football
Two       1     -2 Basketball
Three     3      1 Volleyball
Four      2     -1   Handball
Five      5      2      Rugby

[[2]]
[1] 1 2 3 4 5

[[3]]
[[3]][[1]]
[1] 1

[[3]][[2]]
[1] 2

[[3]][[3]]
[1] 3

[[3]][[4]]
[1] 4

[[3]][[5]]
[1] 5
\end{lstlisting}

По подобие на рамкираните данни, списъците също могат да съдържат наименования на елементите си (Листинг \ref{listing0043}).

\begin{lstlisting}[caption=Названия на елементите в списъка, label=listing0043]
names( l3 ) <- c("Frame", "Vector", "Element")
l3
$Frame
      First Second      Sport
One       4      0   Football
Two       1     -2 Basketball
Three     3      1 Volleyball
Four      2     -1   Handball
Five      5      2      Rugby

$Vector
[1] 1 2 3 4 5

$Element
$Element[[1]]
[1] 1

$Element[[2]]
[1] 2

$Element[[3]]
[1] 3

$Element[[4]]
[1] 4

$Element[[5]]
[1] 5
\end{lstlisting}

Достъпът до елементите на списъка може да стане по индекс\index{достъп по индекс} или по название на елемента (Листинг \ref{listing0044}).

\begin{lstlisting}[caption=Достъп до елементите на списъка, label=listing0044]
l3[ 1 ]
$Frame
      First Second      Sport
One       4      0   Football
Two       1     -2 Basketball
Three     3      1 Volleyball
Four      2     -1   Handball
Five      5      2      Rugby

l3[ "Frame" ]
$Frame
      First Second      Sport
One       4      0   Football
Two       1     -2 Basketball
Three     3      1 Volleyball
Four      2     -1   Handball
Five      5      2      Rugby
\end{lstlisting}

Чрез вложено позоваване може да се достъпи конкретен елемент (Листинг \ref{listing0045}). Тъй като сложните структури от данни могат да съдържат на свой ред сложни структури от данни, вложеното позоваване може да добие твърде неприветлив вид.

\begin{lstlisting}[caption=Вложено позоваване, label=listing0045]
l3[[1]]
      First Second      Sport
One       4      0   Football
Two       1     -2 Basketball
Three     3      1 Volleyball
Four      2     -1   Handball
Five      5      2      Rugby

l3[["Frame"]]$Sport
[1] Football   Basketball Volleyball Handball   Rugby     
Levels: Basketball Football Handball Rugby Volleyball
\end{lstlisting}

Добавянето на елемент към списъка става с директно позоваване към елемента на който индекс трябва да попадне новият елемент, дори и това място да не е предварително предвидено (Листинг \ref{listing0046}).

\begin{lstlisting}[caption=Добавяне на елемент, label=listing0046]
l3[ 4 ] <- "Games"
l3
$Frame
      First Second      Sport
One       4      0   Football
Two       1     -2 Basketball
Three     3      1 Volleyball
Four      2     -1   Handball
Five      5      2      Rugby

$Vector
[1] 1 2 3 4 5

$Element
$Element[[1]]
[1] 1

$Element[[2]]
[1] 2

$Element[[3]]
[1] 3

$Element[[4]]
[1] 4

$Element[[5]]
[1] 5

[[4]]
[1] "Games"

length( l3 )
[1] 4
\end{lstlisting}

\section{Матрици}

Една от най-важните структури в математиката и статистиката е матрицата. Матриците\index{матрици} в R много наподобяват рамкираните данни, тъй като се състоят от колони и редове с разликата, че всички елементи на матрицата са еднотипни. По аналогия с векторите, матриците също се обработват с матрична аритметика, а не с обикаляне на елементите един по един.

\begin{lstlisting}[caption=Създаване на матрици, label=listing0047]
m1 <- matrix(1:6, nrow=3)
m1
     [,1] [,2]
[1,]    1    4
[2,]    2    5
[3,]    3    6

m2 <- matrix(7:12, nrow=3)
m2
     [,1] [,2]
[1,]    7   10
[2,]    8   11
[3,]    9   12

m3 <- matrix(7:18, nrow=2)
m3
     [,1] [,2] [,3] [,4] [,5] [,6]
[1,]    7    9   11   13   15   17
[2,]    8   10   12   14   16   18
\end{lstlisting}

Създаването на матрици става с функцията matrix (Листинг \ref{listing0047}). От съществено значение е размерът на матрицата, както и попълването на елементите, което се случва колона по колона.

\begin{lstlisting}[caption=Операции с матрици, label=listing0048]
nrow( m1 )
[1] 3

ncol( m1 )
[1] 2

dim( m1 )
[1] 3 2

m1 + m2
     [,1] [,2]
[1,]    8   14
[2,]   10   16
[3,]   12   18
 
m1 * m2
     [,1] [,2]
[1,]    7   40
[2,]   16   55
[3,]   27   72

m1 == m2
      [,1]  [,2]
[1,] FALSE FALSE
[2,] FALSE FALSE
[3,] FALSE FALSE
\end{lstlisting}

Повечето матрични операции се изпълняват елемент за елемент (Листинг \ref{listing0048}), но матричното умножение е малко по-особено тъй като изисква съчетаване на размерите по колони и редове (Листинг \ref{listing0049}).

\begin{lstlisting}[caption=Матрично умножение, label=listing0049]
m1 %*% t(m2)
     [,1] [,2] [,3]
[1,]   47   52   57
[2,]   64   71   78
[3,]   81   90   99
\end{lstlisting}

Както при рамкираните данни, така и при матриците може да има имена на колоните и редовете (Листинг \ref{listing0050}). Тази възможност значително подобрява визуализирането на данните след извършването на математическите пресмятания.

\begin{lstlisting}[caption=Имена на колоните и редовете, label=listing0050]
colnames( m1 ) <- c("First", "Second")
rownames( m1 ) <- c("One", "Two", "Three")
m1
      First Second
One       1      4
Two       2      5
Three     3      6

colnames( m2 ) <- c("Left", "Right")
rownames( m2 ) <- c("1st", "2nd", "3rd")
m2
    Left Right
1st    7    10
2nd    8    11
3rd    9    12
\end{lstlisting}

Функцията t служи за транспониране на матрица. Транспонирането най-често се налага при матричното умножение (Листинг \ref{listing0049}). За да бъде транспонирана една матрица, елементите ѝ се разменят симетрично, спрямо главния диагонал. Не е нужно матрицата да бъде квадратна за да бъде транспонирана. Транспонирането е валидно и за правоъгълни матрици.

\section{Масиви}

Масивът\index{масиви} по своята същност е многомерен вектор. Елементите на масива са еднотипни и достъпът до тях също се осъществява по индекс с квадратни скоби. Първият индекс е за ред, а вторият за колона и така нататък за по-високите размерности.

\begin{lstlisting}[caption=Работа с масиви, label=listing0052]
a1 <- array(1:12, dim = c(2, 3, 2))
a1
, , 1

     [,1] [,2] [,3]
[1,]    1    3    5
[2,]    2    4    6

, , 2

     [,1] [,2] [,3]
[1,]    7    9   11
[2,]    8   10   12

a1[1, , ]
     [,1] [,2]
[1,]    1    7
[2,]    3    9
[3,]    5   11
 
a1[1, , 1]
[1] 1 3 5
 
a1[, , 1]
     [,1] [,2] [,3]
[1,]    1    3    5
[2,]    2    4    6
\end{lstlisting}

Основната разлика между матриците и масивите е, че матриците са ограничени до две размерности, докато масивите могат да имат много измерения.

\section*{Заключение}

В настоящата глава са представени възможностите за извикване на функции в R. Разгледани са начините за извикване на документация за функциите. Представени са начини за работа с липсващи данни и са демонстрирани някои от по-сложните типове данни.


\newpage
\chapter{Въвеждане на данни и извеждане на графики}
\label{chapter04}

Статистическата обработка на данни в R започва с въвеждането на събраната информация\index{въвеждане на информация} и завършва с визуализация на резултатите\index{визуализация на резултатите} от анализа. Тези две фази от етапа на статистическата обработка имат своята важност, тъй като входящите данни силно определят надеждността на извършвания анализ, а правилно визуализираните резултати определят степента на разбиране, която ще постигне аудиторията пред която анализът се представя. 

\section{Въвеждане на данни от външни източници}

Данните в примерите до тази глава бяха фиксирани и се въвеждаха ръчно от конзолата, в интерактивен режим. Този начин на работа не е най-рационалния, когато се правят модели и с данните за модела се провеждат многократни експерименти. Обичайната практика е командите за съставянето на модела да бъдат написани в обикновен текстов файл, с разширение „.r“, а данните да бъдат зареждани от външен файл\index{четене от файл}. Този начин на работа позволява да бъдат създадени множество модели, които често се различават по нещо дребно, и да бъдат зареждани различни входни данни, примерно за различни периоди на измерване. 

Продуктът R позволява множество различни начини за въвеждане на данни в системата, но най-достъпният начин е през CSV (Comma Separated Values) файлове. CSV файловият формат е текстов файлов формат, който позволява таблично представяне на данни (колони и редове). CSV може да бъде четен и редактиран с обикновен текстов редактор, като Notepad под Microsoft Windows, TextEdit под Mac OS X или Nano под Linux. CSV комфортно се визуализира и обработва от продуктите Microsoft Excel, OpenOffice Calc и Libre Calc. 

\begin{lstlisting}[caption=Зареждане на данни от CSV файл, label=listing0053]
data <- read.table(file="http://raw.githubusercontent.com/TodorBalabanov/Statistical-Data-Processing-with-R/master/data/tomato.csv", header=TRUE, sep=",")

head( data )
  Round             Tomato Price      Source Sweet Acid Color Texture Overall
1     1         Simpson SM  3.99 Whole Foods   2.8  2.8   3.7     3.4     3.4
2     1  Tuttorosso (blue)  2.99     Pioneer   3.3  2.8   3.4     3.0     2.9
3     1 Tuttorosso (green)  0.99     Pioneer   2.8  2.6   3.3     2.8     2.9
4     1     La Fede SM DOP  3.99   Shop Rite   2.6  2.8   3.0     2.3     2.8
5     2       Cento SM DOP  5.49  D Agostino   3.3  3.1   2.9     2.8     3.1
6     2      Cento Organic  4.99  D Agostino   3.2  2.9   2.9     3.1     2.9
  Avg.of.Totals Total.of.Avg
1          16.1         16.1
2          15.3         15.3
3          14.3         14.3
4          13.4         13.4
5          14.4         15.2
6          15.5         15.1

tail( data )
   Round                   Tomato Price      Source Sweet Acid Color Texture
11     3       Scotts Backyard SM  0.00  Home Grown   1.6  2.9   3.1     2.4
12     3 Di Casa Barone (organic) 12.80      Eataly   1.7  3.6   3.8     2.3
13     4         Trader Joes Plum  1.49 Trader Joes   3.4  3.3   4.0     3.6
14     4          365 Whole Foods  1.49 Whole Foods   2.8  2.7   3.4     3.1
15     4        Muir Glen Organic  3.19 Whole Foods   2.9  2.8   2.7     3.2
16     4        Bionature Organic  3.39 Whole Foods   2.4  3.3   3.4     3.2
   Overall Avg.of.Totals Total.of.Avg
11     1.9          11.9         11.9
12     1.4          12.7         12.7
13     3.9          17.8         18.2
14     3.1          14.8         15.2
15     3.1          14.8         14.7
16     2.8          15.1         15.2
\end{lstlisting}

Зареждането на CSV в R най-ефективно се постига с функцията read.table (Листинг \ref{listing0053}). Резултатът от четенето е обект от тип рамкирани данни. При викането на функцията параметрите се подават с явно изписване на имената им. Точният адрес на файла се подава в кавички, а когато първият ред от данните е заглавен ред се подава флаг за заглавен ред. Третият аргумент е за указване на разделителя в редовете, тъй като не винаги този разделител е запетая. Често софтуерните продукти за електронни таблици поставят символа за табулация, като разделител между данните на един ред. Символът табулация попада в групата на белите символи (white spaces), тъй като не се изобразява видимо, а с празно пространство. Когато трябва да бъде подаден като аргумент за разделител се използва комбинацията от обратна наклонена черта и буквата t (\textbackslash t).

\section*{Заключение}

Коректното въвеждане на данните в системата е от изключителна важност за осъществяването на коректен анализ и постигането на приемливи статистически резултати. В другия край на процеса е самото визуализиране на получените резултати и максималната експресивност, която може да се постигне за представянето пред широка аудитория. Тези две стъпки от процеса по статистически анализ са свързани с въвеждането на информацията и графичното визуализиране на получените резултати. 


\newpage
\chapter{Оператори за контрол на изпълнението и потребителски функции}
\label{chapter05}
\thispagestyle{empty}

С процеса на усвояване на софтуерния продукт R всеки потребител установява, че някои команди биват повтаряни многократно. Това навежда на мисълта, че тези команди може да бъдат групирани и съхранени като програмен скрипт\index{програмен скрипт} (модел на експеримента). Основното предимство на такава организация е възможността един и същ модел да бъде използван за множество експерименти. Второто предимство е, че моделът\index{експериментален модел} може да бъде разменян между различни потребители, които работят над същата задача или сходни задачи. Езикът R е в групата на интерпретативните програмни езици, където се намира и програмният език JavaScript. За потребителите познаващи JavaScript, някои конструкции в R биха били изключително познати.

Скриптовете на R се записват в обикновени текстови файлове с .r разширение, както примерния файл на адрес:

\begin{lstlisting}[caption=Адрес на примерен R скрипт, label=listing0074]
https://raw.githubusercontent.com/TodorBalabanov/Statistical-Data-Processing-with-R/master/code/example0001.r
\end{lstlisting}

Може да се пише с текстов редактор по избор на потребителя, но може да се използва и текстовият редактор към продукта (Фиг. \ref{figure0025}).

\begin{figure}[h!]
  \centering
  \includegraphics[width=1.0\linewidth]{pic0025}
  \caption{Текстов редактор към продукта R}
\label{figure0025}
\end{figure}
\FloatBarrier

Най-бързият начин за стартиране на скрипта\index{изпълнение на скрипт} е чрез менюто на вградения текстов редактор Edit->Execute (Фиг. \ref{figure0026}).

\begin{figure}[h!]
  \centering
  \includegraphics[width=1.0\linewidth]{pic0026}
  \caption{Стартиране на R скрипт}
\label{figure0026}
\end{figure}
\FloatBarrier

Съществено е целият текст на скрипта да бъде маркиран, тъй като командният интерпретатор е оптимизиран в режим за изпълнение на команда по команда. Когато целият скрипт е маркиран се изпълняват, една след друга, всички команди.

\begin{figure}[h!]
  \centering
  \includegraphics[width=1.0\linewidth]{pic0027}
  \caption{Резултат от изпълнението на R скрипт}
\label{figure0027}
\end{figure}
\FloatBarrier

Резултатът от изпълнението на R скрипта се наблюдава в командния интерпретатор на продукта (Фиг. \ref{figure0027}) и изглежда точно както би трябвало командите да се въведат, на ръка, ако не бяха заредени от скриптов файл.

Алтернативна възможност за стартиране на R скриптове е конзолата на операционната система. При този вариант в конзолата на операционната система се извиква приложението R, а като параметри на приложението се подава файлът, съдържащ скрипта и параметър дали сесията от изпълнението на скрипта да бъде съхранена (Листинг \ref{listing0085}).

\begin{lstlisting}[caption=Изпълнение на R скрипт от конзолата на операционната система, label=listing0085]
r < ./Statistical-Data-Processing-with-R/code/example0001.r --no-save
\end{lstlisting}

При такова изпълнение се спестява зареждането на целия програмен продукт R за постоянно в оперативната памет (Фиг. \ref{figure0028}).

\begin{figure}[h!]
  \centering
  \includegraphics[width=1.0\linewidth]{pic0028}
  \caption{Резултат от изпълнението на R скрипт в конзолата на операционната система}
\label{figure0028}
\end{figure}
\FloatBarrier

Програмните скриптове се състоят от последователни инструкции, но с подходящи оператори за преход или повторение изпълнението на командите може да протече в различен от линейния ред. Тази група оператори се нарича оператори за контрол на изпълнението. Общата конструкция на операторите е заглавна част (ключова дума и условие) и тяло.

\section{Оператори за преход}

Операторите за условен преход\index{оператори за преход} променят изпълнението на скрипта в зависимост от логическо или числено условие. От там идва и названието им. В тази група оператори попадат if, else и switch.

В заглавните части на операторите за преход може да се проверява само едно условие или да се проверяват цяла група от условия (логически изрази)\index{логически операции}. За тази цел в R има логически операции като „И“ (операции \& и \&\&) и „ИЛИ“ (операции | и ||). При двойната форма на операциите се сравнява само по една стойност от двете страни (не са векторизирани), докато при единичната форма се сравняват елемент по елемент в множества от елементи от двете страни. Поради тази причина двойната форма е полезна при if оператора, а единичната форма се изисква при ifelse конструкцията. Друга много важна разлика между единичната и двойната форма е в начина по който се изчисляват изразите от двете страни на операцията. При единичната форма задължително се изчисляват и двата операнда, независимо дали това действително е нужно. При двойната форма се изчисляват само тези операнди, които са достатъчни за да определят финалният резултат от пресмятането на логическия израз. Тази разлика в използването на логическите операции може да се окаже изключително съществена, когато операндите са функции връщащи логическа стойност. В случаите когато е ненужно някой от операндите да се изчисли, тъй като другите вече са определили резултата, то част от функциите няма да бъдат извикани. С помощта на логическите операции могат да се построят много сложни логически изрази, при които важат общите правила за приоритет на операциите, както и възможността приоритетът да се променя, чрез подходящо поставяне на скоби.

\subsection{Оператор за условен преход}

Дори чисто исторически в програмните езици един от първите оператори за преход е оператора за условен преход (if оператор)\index{оператор за условен преход}.

\begin{lstlisting}[caption=Оператор за условен преход if, label=listing0075]
sayHello <- sample(c(TRUE,FALSE), 1, TRUE);

if(sayHello == TRUE) {
	print( "Hello!" );
}

print( "Bye!" );
\end{lstlisting}

Операторът за условен преход използва ключовата дума if (Листинг \ref{listing0075}), а в заглавната му част се записва израз пресмятан до логическа стойност TRUE или FALSE\index{логически стойности}. Смисълът на оператора if е, че тялото му бива изпълнено единствено ако изчислението на израза в заглавната част доведе до стойност TRUE. Ако изразът в заглавната част бъде изчислено до стойност FALSE, тялото на оператора се пропуска и изпълнението на програмата продължава след него.

В примера от листинг \ref{listing0075} променливата sayHello получава една случайна логическа стойност (TRUE или FALSE), като двете възможности са равно вероятни. На следващия ред операторът if изписва "Hello!" или го пропуска и изписва "Bye!". Скриптът трябва да се стартира няколко пъти, за да се наблюдава ефектът от случайния избор на стойност за променливата sayHello.

\subsection{Алтернатива при условен преход}

В множество ситуации, освен основна алтернатива за оператора if, е необходимо да има и допълнителна алтернатива\index{алтернатива при условен преход}, която да се изпълни при резултат от логическия израз FALSE. За тази цел конструкцията на оператора if може да се разшири с добавяне на else конструкция (Листинг \ref{listing0076}).

\begin{lstlisting}[caption=Оператор за условен преход if-else, label=listing0076]
sayHello <- sample(c(TRUE,FALSE), 1, TRUE);

if(sayHello == TRUE) {
	print( "Hello!" );
} else {
	print( "Hi!" );
}

print( "Bye!" );
\end{lstlisting}
Конструкцията else е контекстно зависима и поради тази причина може да се използва единствено в комбинация с конструкцията на оператора if. В примерния код от листинг \ref{listing0076} в половината от случаите на конзолата ще се изпише "Hello!", а в другата половина "Hi!".

\subsection{Каскада от условни преходи}

Условният преход ограничава до две възможности, но практиката понякога налага да се избират повече алтернативи. В такава ситуация може да се използва каскада от if-else конструкции\index{каскада от оператори за условен преход} (Листинг \ref{listing0077}).

\begin{lstlisting}[caption=Каскада от if-else, label=listing0077]
sayHello <- sample(c(0,1,2), 1, TRUE);

if(sayHello == 0) {
	print( "Hello!" );
} else if(sayHello == 1) {
	print( "Hi!" );
} else if(sayHello == 2) {
	print( "Yoo!" );
} else {
	print( "Error!" );
}

print( "Bye!" );
\end{lstlisting}

Недостатък на каскадните проверки е, че всяко условие трябва да бъде проверявано по отделно. Каскадата може да завършва с else конструкция, но тя не е задължителна.

R предлага ifelse оператор, който много прилича на if конструкцията в Microsoft Excel (Листинг \ref{listing0078}) и сериозно се различава от if-else оператора. Една от най-силните страни на ifelse конструкцията е, че тя е векторизирана и може да се прилага над група елементи едновременно.

\begin{lstlisting}[caption=Функцията ifelse, label=listing0078]
ifelse(sample(c(FALSE,TRUE), 1, TRUE), "Yes", "No")

ifelse(c(1,1,0,1,0,1)==1, "Yes", "No")
\end{lstlisting}

\subsection{Оператор за многовариантен избор}

Писането на каскадни конструкции от типа if-else може да бъде твърде неудобно и поради тази причина съществува switch конструкцията\index{оператор за многовариантен избор} (Листинг \ref{listing0079}).

\begin{lstlisting}[caption=Конструкция за многовариантен избор switch, label=listing0079]
switch(sample(c("a","b","c","d","e"),1,TRUE), "a"="one", "b"="two", "c"="three", "d"="four", "other")
\end{lstlisting}

Първият аргумент на switch е стойността която ще се проверява, а след това са изброени алтернативните възможности. Последната стойност, ако не й е зададена стойност за проверка, служи за отговор, когато нито една от алтернативите не е била определена. В примерния код на случаен принцип се избира една буква от пет възможни (конструкцията sample), след което се проверяват четири алтернативи и последна опция за else условие.

\section{Оператори за цикъл}

В конвенционалните програмни езици е обичайна практика елементите на масивите и контейнерите за данни (списъци, стекове, опашки и други) да бъдат обхождани един по един с помощта на цикли. В R целта е да бъдат прилагани векторизирани операции и максимално да се избягва използването на цикли за обхождане на елементи в контейнер за данни. Въпреки това, в някои ситуации се налага използването на цикли\index{оператори за цикъл} и поради тази причина R поддържа циклите for и while.

\subsection{Цикъл за обхождане}

Операторът for\index{цикъл за обхождане} в R представлява цикъл за обхождане на елементи във вектор, като елементът от текущата итерация е достъпен за използване в тялото на цикъла (Листинг \ref{listing0080}).

\begin{lstlisting}[caption=Оператор за цикъл for, label=listing0080]
for(number in 1:10) { print(number); }
[1] 1
[1] 2
[1] 3
[1] 4
[1] 5
[1] 6
[1] 7
[1] 8
[1] 9
[1] 10
\end{lstlisting}

Заглавната част се състои от променлива (в случая number), ключовата дума in и вектор от възможни стойности (в случая числата от едно до десет). Векторите могат да бъдат с различен тип на елементите, примерно символни низове (Листинг \ref{listing0081}).

\begin{lstlisting}[caption=Обхождане на вектор от символни низове, label=listing0081]
for(f in c("orange", "lemon", "kiwi", "cherry")) { print(f); }
[1] "orange"
[1] "lemon"
[1] "kiwi"
[1] "cherry"
\end{lstlisting}

\subsection{Цикъл с условие за край}

Когато множество от елементи няма да бъде обхождано, е по-удачно да се използва цикълът while\index{цикъл с условие за край}, който в заглавната си част съдържа логически израз, определящ условието за край на цикъла. Цикълът се върти, докато логическият израз в заглавната му част се пресмята до стойност TRUE (Листинг \ref{listing0082}).

\begin{lstlisting}[caption=Цикъл с условие за край, label=listing0082]
counter <- 1;
while(counter <= 5) { 
	print( counter ); 
	counter <- counter + 1;
}
[1] 1
[1] 2
[1] 3
[1] 4
[1] 5
\end{lstlisting}

При цикъла с условие за край променливата, която определя условията за приключване на итерациите трябва да се определи преди началото на цикъла. За да не бъде цикълът безкраен, е нужно тази променлива да бъде променена в тялото на цикъла, по такъв начин, че той да приключи изпълнението си.

\subsection{Прекъсване на циклите}

Понякога се налага определена итерация на цикъла да бъде прекъсната\index{прекъсване на цикли}. За тази цел R предлага ключовата дума next, която прекъсва текущата итерация и преминава към следващата (Листинг \ref{listing0083}).

\begin{lstlisting}[caption=Прекъсване на итерация, label=listing0083]
for(number in 1:10) { 
	if(number == 7) {
		next;
	}

	print(number);
}
[1] 1
[1] 2
[1] 3
[1] 4
[1] 5
[1] 6
[1] 8
[1] 9
[1] 10
\end{lstlisting}

В този случай пропуснатото число е седем, тъй като при седмата итерация е било изпълнено условието на оператора if и неговото тяло съдържа ключовата дума next.

В други ситуации се налага цикълът да бъде спрян изцяло и тогава се използва ключовата дума break (Листинг \ref{listing0084}).

\begin{lstlisting}[caption=Прекъсване на цикъла, label=listing0084]
for(number in 1:10) { 
	if(number == 3) {
		break;
	}

	print(number);
}
[1] 1
[1] 2
\end{lstlisting}

Двете ключови думи (next и break) са контекстно зависими конструкции и могат да се използват единствено в телата на циклите for и while.

\section{Потребителски функции}

При писането на програмен код, в съвременните програмни езици, е изключително добра практика, кодът да се групира в серия инструкции и те да се оформят като самостоятелна единица. За тази цел, R предлага възможността кодът да се оформя в потребителски написани функции\index{потребителски функции}.

Оформянето на скриптовете в добре организирани и малки по размер функции е основен похват в съвременното програмиране. Такъв стил на работа позволява по-лесна поддръжка, по-лесна проверка на резултатите създавани от функцията и по-лесна преизползваемост на кода.

Писането на функции в R малко се различава от повечето конвенционални програмни езици, но много прилича на функциите писани в JavaScript. Самата функция представлява обект, който бива присвоен на идентификатор (Листинг \ref{listing0086}).

\begin{lstlisting}[caption=Примерна потребителска функция, label=listing0086]
say.hello <- function() {
	print("Hello, World!");
}

say.hello();
[1] "Hello, World!"
\end{lstlisting}

Важно е да се отбележи, че символът точка (.) е валиден символ за съставяне на име на функцията. Това е фундаментална разлика спрямо повечето конвенционални програмни езици. Въпреки това, имената не бива да започват само с точка, тъй като обекти именувани по този начин имат по-специфична употреба. Функцията се създава с ключовата дума function\index{дефиниция на функция}, а тялото на функцията се огражда в къдрави скоби. Много добра практика е да се спазват стриктни правила за подравняване на различните конструкции, използвани като команди в тялото на функцията. Такъв стил на писане подобрява възможностите за четене на програмния текст и възможностите за откриване на евентуални грешки в него.

\subsection{Аргументи на функция}

Тъй като функциите са самостоятелно обособени групи от инструкции, то понякога е нужно към групата да бъдат подавани параметри, под формата на аргументи на функцията\index{аргументи на функция} (Листинг \ref{listing0087}).

\begin{lstlisting}[caption=Извикване на функция с аргумент, label=listing0087]
hello.person <- function( name ) {
	print( sprintf("Hello, %s!",name) );
}

hello.person("Dessislava");
[1] "Hello, Dessislava!"
\end{lstlisting}

Променливата name е достъпна само в тялото на функцията и не може да бъде използвана извън него.

\begin{lstlisting}[caption=Извикване на функция с повече аргументи, label=listing0088]
hello.person <- function(first, last) {
	print( sprintf("Hello, %s %s!",first,last) );
}

hello.person("Dessislava", "Gruncharova");
[1] "Hello, Dessislava Gruncharova!"

hello.person(last="Mladevnova", first="Vyara");
[1] "Hello, Vyara Mladevnova!"
\end{lstlisting}

Функциите в R позволяват викане с изброяване на параметрите по позиции или чрез явно указване кой аргумент, каква стойност да получи (Листинг \ref{listing0088}).

\subsection{Аргументи с подразбираща се стойност}

В някои ситуации е удачно някои от аргументите да имат зададена стойност по подразбиране. В много от съвременните езици това се постига с едноименни функции (overloading), но в R този ефект е възможен, чрез задаване на подразбираща се стойност\index{аргументи на функция с подразбираща се стойност} (Листинг \ref{listing0089}).

\begin{lstlisting}[caption=Извикване на функция с подразбиращи се аргументи, label=listing0089]
hello.person <- function(first, last, title="") {
	print( sprintf("Hello, %s %s %s!",title,first,last) );
}

hello.person("Zornitsa", "Radeva", "Miss");
[1] "Hello, Miss Zornitsa Radeva!"

hello.person("Todor", "Balabanov");
[1] "Hello,  Todor Balabanov!"
\end{lstlisting}

\subsection{Променлив брой аргументи}

По аналогия с програмния език C, в R са възможни функции с променлив брой аргументи, което се постига с операцията триеточие (...) в заглавната част на функцията (Листинг \ref{listing0090}). Механизмът с променлив брой аргументи\index{функции с променлив брой аргументи} е особено полезен, но той също така води и до много сериозни програмни грешки.

\begin{lstlisting}[caption=Функция с променлив брой аргументи, label=listing0090]
sum.up <- function(a, b, ...) {
	print( a+b );
}

sum.up(1, 2, 3, 4);
\end{lstlisting}

\subsection{Върната стойност}

Потребителските функции трябва да се пишат по такъв начин, че да функционират на принципа на черната кутия – аргументи на входа с резултат на изхода и капсулирано изчисление в тялото на функцията. За да бъде постигната тази цел, функциите в R могат да връщат стойност\index{върната стойност от функция}. По аналогия с JavaScript, в R може да се връща стойност от различен тип с ключовата дума return (Листинг \ref{listing0091}).

\begin{lstlisting}[caption=Връщане на стойност от функция, label=listing0091]
sum.up <- function(a, b, ...) {
	return(a + b);
}

print( sum.up(1,2,3,4) );
\end{lstlisting}

\subsection{Предаване на функция като аргумент}

В относително редки случаи се налага върху входните данни да бъдат извикани различни функции. В такава ситуация е важно към самата функция да бъде подаден обект от тип функция\index{функционални обекти} (Листинг \ref{listing0092}).

\begin{lstlisting}[caption=Избор на функция за извикване по време на изпълнение, label=listing0092]
do.stat <- function(values, calculation) {
	do.call(calculation, args=list(values));
}

print( do.stat(1:10,mean) );
[1] 5.5

print( do.stat(1:10,median) );
[1] 5.5

print( do.stat(1:10,sd) );
[1] 3.02765
\end{lstlisting}

В други програмни езици този ефект се постига с функционални указатели (езикът C) или функционални обекти (езикът Java).

\section*{Заключение}

Операторите за контрол на изпълнението позволяват създаването на по-сложни експериментални модели и извършването на по-задълбочени статистически изследвания. За ефективна и надеждна работа, езикът R позволява командите и операторите за контрол на изпълнението да бъдат организирани в отделни потребителски функции и файлове, съдържащи програмните скриптове.


\newpage
\chapter{Групиране на данни}
\label{chapter06}

\begin{equation}
OBP = \frac{H + BB + HBP}{AB + BB + HBP + SF}
\label{equation0001}
\end{equation}
\listofequations{On Base Percentage (OBP) статистика}

\section*{Заключение}

\newpage
\chapter{Реорганизация на данните и обработка на символни низове}
\label{chapter07}

Освен манипулацията на данните, често се налага реорганизиране на начина по който те са структурирани. Понякога се налага транспониране или пък обединяване на няколко множества данни в едно общо. При обединяване на данни от различни източници не винаги данните имат една и съща структура, което допълнително усложнява задачата по преструктурирането им. 

\section{Обединяване на множества от данни}

Най-елементарният случай на обединение е при наличието на две множества от данни, които имат идентични колони или колоните съвпадат по брой и имена.

\begin{lstlisting}[caption=Обединяване на множества от данни, label=listing0132]
ds1 <- cbind(TV=c("BNT","bTV","Nova"),Channel=c(1,2,3),Rating=c(0.1,0.3,0.2))

ds2 <- data.frame(TV=c("HBO","VH1","MTV"),Channel=c(4,5,6),Rating=c(0.4,0.5,0.6),stringsAsFactors=FALSE)

ds <- rbind(ds1, ds2)
\end{lstlisting}

С такава ситуация се използват функциите cbind и rbind (Листинг \ref{listing0132}). С функцията cbind (свързване на колони) се формира мтарица, което изисква броя редове в съставляващите я списъци да е еднакъв. Функцията rbind (свързване на редове) обединява две множества при които броят редове може да се различава. 

В реалната практика, данните рядко биват събрани в подходяща за обединяване структура. В такива случаи често се налага използването на сливане по ключ, което е добре познато на хората работещи с езика SQL. За демонстрация на възможните сливания е използвано множеството данни предоставено от USAID Open Government инициативата (Листинг \ref{listing0133}). 

\begin{lstlisting}[caption=USAID множество от данни, label=listing0133]
setwd("~/Desktop")

download.file(url="https://github.com/TodorBalabanov/Statistical-Data-Processing-with-R/raw/master/data/aid.zip", destfile="aid.zip")

unzip("aid.zip", exdir="./")
\end{lstlisting}

След като бъде свален архивният файл, той трябва да бъде разархивиран. 

\begin{lstlisting}[caption=Зареждане USAID данните в R, label=listing0134]
library(stringr)

for(file in dir("./",pattern="\\.csv")) {
	name <- str_sub(string=file, start=12, end=18)

	data <- read.table(file=file.path(".", file), header=TRUE, sep=",", stringsAsFactors=FALSE)

	assign(x=name, value=data)
}
\end{lstlisting}

Тъй като данните са разпръснати в множество CVS файлове, на които имената са съставени по определен шаблон, то е удачно зареждането на информацията да бъде автоматизирано, като се прегледа цялата директория и бъдат прочетени всички налични в нея CSV файлове (Листинг \ref{listing0134}). Тъй като информацията от всеки прочетен файл трябва да се присвои на променлива, то е важно да се подберат подходящи имена за променливите. R е език в който малките и големите букви имат значение и поради тази причина трябва да се внимава с изписването на променливите. Един от вариантите за избор на имена е частична информация от основното име на файла. Чрез подходящо отрязване на символите (преди 12 и след 18) от името на файла се формира достатъчно разпознаваемо име за променлива. Следва прочитане на информацията и присвояването й на съответната променлива. 

\subsection{Функция merge}

\begin{lstlisting}[caption=Сливане на данни с merge, label=listing0135]
head(merge(x=Aid_90s, y=Aid_00s, by.x=c("Country.Name", "Program.Name"), by.y=c("Country.Name", "Program.Name")))
\end{lstlisting}

При сливане на данни в две data.frame структури може да се използва функцията merge (Листинг \ref{listing0135}). Чрез by.x се определя ключът в левия data.frame, а чрез by.y се определя ключът в десния data.frame. Определянето на различни колони, като ключ е най-значимата възможност на функцията merge. Трябва да се има предвид, че функцията merge е в базовия пакет на R и съответно може да бъде относително бавна при изпълнението си, в сравнение с други алтернативни функции. 

\begin{lstlisting}[caption=Сливане на данни при data.table, label=listing0136]
library(data.table)

dt90 <- data.table(Aid_90s, key=c("Country.Name", "Program.Name"))
dt00 <- data.table(Aid_00s, key=c("Country.Name", "Program.Name"))

dt0090 <- dt90[dt00]
\end{lstlisting}

Сливането на данни в пакета data.table използва малко по-различен синтаксис от този при функцията merge (Листинг \ref{listing0136}).

\subsection{Функция join}

Функцията join, от пакета plyr, работи по аналогичен начин, както функцията merge, но е с по-добро бързодействие. Недостатък на тази функция е, че всички колони, участващи в ключа, трябва да имат идентични имена (Листинг \ref{listing0137}). 

\begin{lstlisting}[caption=Сливане на данни с join, label=listing0137]
library(plyr)

head(join(x=Aid_90s, y=Aid_00s, by=c("Country.Name", "Program.Name")))
\end{lstlisting}

Функцията join има аргумент с който може да се окажат различните видове сливане (ляво, дясно, вътрешно и външно). За да се получи едно общо множество, чрез функцията Reduce може да се изпълнят множество сливания по двойки. 

\subsection{Транспониране на данните}

В практиката често се налага размяна на редовете с колоните и обратното. Въпреки че програмни пакети, като Microsoft Excel предлагат такава функционалност, понякога техните ограничения могат да създадат значителни затруднения. Примерно в Microsoft Excel е възприето, че редовете по брой значително превъзхождат възможностите за брой колони. При транспониране на големи обеми от данни е съвсем възможно броя колони да не достигнат и това да доведе до грешка при трансформацията. 

\section*{Заключение}


\newpage
\chapter{Разширени графични възможности и вероятностни разпределения}
\label{chapter08}
\thispagestyle{empty}

\section{Разширени графични възможности}

Към базовите графични възможности на R, пакетите $ggplot2$ и $lattice$ добавят множество допълнителни функционалности. Синтаксисът на извикванията леко се различава от този на функциите в базовите възможности, но това създава затруднения само в началния етап от употребата на двата пакета. 

За визуализация, чрез $ggplot2$ като основа се използва функцията $ggplot$. В общия случай тази функция получава, като входен параметър данните за визуализиране и понякога някои допълнителни параметри. Резултатът от изпълнението на $ggplot$ е обект, който в последствие може да бъде допълнително променян, чрез добавяне на възможности с помощта на операцията събиране (+). 

\subsection{Хистограми и плътности}

Хистограмата\index{хистограма} служи за групиране на стойностите и изброяване на това колко стойности попадат по определените групи. Размерът на всяка група (bin) определя ширината на стълба. 

\begin{lstlisting}[caption=Хистограма и плътност, label=listing0150]
library(ggplot2)

ggplot(data=diamonds) + geom_histogram(aes(x=carat))

ggplot(data=diamonds) + geom_density(aes(x=carat),fill="grey50")
\end{lstlisting}

Както и в предходния пример за хистограма, тук също е представено разпределението на диамантите по карати (Листинг \ref{listing0150}).

\begin{figure}[h!]
  \centering
  \includegraphics[width=1.0\linewidth]{pic0030}
  \caption{Хистограма при 30 групи}
\label{figure0030}
\end{figure}
\FloatBarrier

Функцията $aes$ определя кои данни да бъдат използвани за разполагане по осите. В примера с диамантите, това е характеристиката за тегло (карат). 

\begin{figure}[h!]
  \centering
  \includegraphics[width=1.0\linewidth]{pic0031}
  \caption{Хистограма при 100 групи}
\label{figure0031}
\end{figure}
\FloatBarrier

Размерът на групите в хистограмата може да варира (Фиг. \ref{figure0030},\ref{figure0031}). За да се изчертае плътностна функция\index{плътностна функция} е достатъчно графичният обект, генериран от $ggplot$, да бъде декориран с функцията $geom\_density$ (Фиг. \ref{figure0032}), вместо с функцията $geom\_histogram$. 


\begin{figure}[h!]
  \centering
  \includegraphics[width=1.0\linewidth]{pic0032}
  \caption{Плътностна функция}
\label{figure0032}
\end{figure}
\FloatBarrier

Хистограмата показва броене по групи, докато плътностната функция задава вероятността определен камък да попадне в предварително определен интервал. Макар и много да си приличат, хистограмата и плътностната функция са подходящи в два различни случая. Хистограмите са полезни при дискретни случайни величини, докато плътностните функции намират повече употреба в непрекъснатите случайни величини. 

\subsection{Диаграми на разсейване}

Пакетът $ggplot2$ разширява възможностите за визуализация на диаграми на разпръскване\index{диаграма на разпръскване}, които базовата функционалност на R предлага (Листинг \ref{listing0151}). 

\begin{lstlisting}[caption=Диаграма на разпръскване с ggplot2, label=listing0151]
library(ggplot2)
ggplot(diamonds, aes(x=carat, y=price)) + geom_point()
\end{lstlisting}

Функцията $aes$ определя кои колони от множеството данни да се използват при визуализацията (Фиг. \ref{figure0033}). Пакетът $ggplot2$ дава и друго много съществено предимство, графиката която ще се изчертава да бъде съхранена в отделен обект, на който обект след това да се добавят различни визуални декорации (обектът $c2p$ от примера).

\begin{figure}[h!]
  \centering
  \includegraphics[width=1.0\linewidth]{pic0033}
  \caption{Диаграма на разпръскване с ggplot2}
\label{figure0033}
\end{figure}
\FloatBarrier

Пакетът дава възможности и за подреждането на група от диаграми на разсейването. Това се постига с някоя от функциите $facet\_wrap$ или $facet\_grid$ (Листинг \ref{listing0152}). 

\begin{lstlisting}[caption=Диаграма на разпръскване групирани по признак, label=listing0152]
c2p <- ggplot(diamonds, aes(x=carat, y=price))

c2p + geom_point(aes(color=color)) + facet_wrap(~color)

c2p + geom_point(aes(color=color)) + facet_grid(clarity~cut)
\end{lstlisting}

Функцията $facet\_wrap$ разделя множеството от данните на групи, според зададения признак (в примера това е цветът на диамантите) и след това формира диаграма на разпръскване за всяка от групите (Фиг. \ref{figure0034}).

\begin{figure}[h!]
  \centering
  \includegraphics[width=1.0\linewidth]{pic0034}
  \caption{Диаграма на разпръскване по групи за цвят на диамантите}
\label{figure0034}
\end{figure}
\FloatBarrier

Функцията $facet\_grid$ действа по сходен начин, но всички стойности на признака за групиране се отразяват на осите за всяка от графиките (Фиг. \ref{figure0035}). Важно е да се забележи начина по който са ориентирани осите на всяка от подграфиките. Ориентацията пряко зависи дали признакът за чистота е от ляво, а признакът за качество на среда от дясно или обратното. 

\begin{figure}[h!]
  \centering
  \includegraphics[width=1.0\linewidth]{pic0035}
  \caption{Визуализация с групиране по два признака}
\label{figure0035}
\end{figure}
\FloatBarrier

Организацията на графики по групи е възможна с различни графични представяния, като пример е представянето на хистограма\index{хистограма} в групи по качество на сряза (Фиг. \ref{figure0036}).

\begin{figure}[h!]
  \centering
  \includegraphics[width=1.0\linewidth]{pic0036}
  \caption{Визуализация на хистограми с групиране}
\label{figure0036}
\end{figure}
\FloatBarrier

\subsection{Графики тип кутия и цигулка}

Пакетът $ggplot2$ дава възможност за визуализация на графики тип кутия\index{графика тик кутия} (Листинг \ref{listing0153}).

\begin{lstlisting}[caption=Визуализация тип кутия, label=listing0153]
ggplot(diamonds, aes(y=depth)) + geom_boxplot()

ggplot(diamonds, aes(y=depth, x=cut)) + geom_boxplot()

ggplot(diamonds, aes(y=depth, x=cut)) + geom_boxplot() + geom_violin()

ggplot(diamonds, aes(y=depth, x=cut))+ geom_point() + geom_violin()
\end{lstlisting}

При обща визуализация на данните, без да се търси групиране по признан за абсцисната ос може да не се подава стойност (Фиг. \ref{figure0037}). 

\begin{figure}[h!]
  \centering
  \includegraphics[width=1.0\linewidth]{pic0037}
  \caption{Визуализация на характеристиката за дълбочина на диамантите}
\label{figure0037}
\end{figure}
\FloatBarrier

Визуализацията на графики от тип кутия, с групиране по признак се реализира чрез подаване на колоната, по която да се групира, като параметър за абцисна ос, на функцията aes (Фиг. \ref{figure0038}).

\begin{figure}[h!]
  \centering
  \includegraphics[width=1.0\linewidth]{pic0038}
  \caption{Dълбочина на диамантите в групи според сряза}
\label{figure0038}
\end{figure}
\FloatBarrier

От графика тип кутии много лесно се преминава към графика от тип цигулки\index{графика тип цигулка}, чрез подмяна на декориращата функция (Фиг. \ref{figure0039}).

\begin{figure}[h!]
  \centering
  \includegraphics[width=1.0\linewidth]{pic0039}
  \caption{Графика тип цигулки}
\label{figure0039}
\end{figure}
\FloatBarrier

Графиките тип кутия и тип цигулка си приличат, като основната разлика е, че цигулките имат повече смисъла на плътностна функция и носят повече информация, отколкото правите ръбове на кутиите. 

\begin{figure}[h!]
  \centering
  \includegraphics[width=1.0\linewidth]{pic0040}
  \caption{Добавяне на декорация с точки}
\label{figure0040}
\end{figure}
\FloatBarrier

Декорациите за визуализация на данните може да се наслагват една върху друга, като от съществено значение е редът на изчертаването им (Фиг. \ref{figure0040}). Ако декорацията с точките бъде добавена след декорацията с цигулките, точки ще се появят и върху самите цигулки. 

\subsection{Линейни графики}

В определени случаи най-удачно е визуалното представяне на информацията да бъде извършено с линейни графики\index{линейна графика}. Линейната графика е удачна примерно при представянето на тренд (Листинг \ref{listing0154}).

\begin{lstlisting}[caption=Линейни графики, label=listing0154]
library(lubridate)

ggplot(economics, aes(x=date, y=pop)) + geom_line()

economics$year <- year(economics$date)
economics$month <- month(economics$date, label=TRUE)

library(scales)

ggplot(economics, aes(x=month, y=pop)) + geom_line(aes(color=factor(year), group=year)) + scale_color_discrete(name="Year") + scale_y_continuous(labels=comma)+ labs(title="Population Growth", x="Month", y="Population")
\end{lstlisting}

В примерните данни за икономическо развитие, нарастването на популацията във времето е представена под формата на линейна графика (Фиг. \ref{figure0041}).

\begin{figure}[h!]
  \centering
  \includegraphics[width=1.0\linewidth]{pic0041}
  \caption{Нарастване на популацията във времето}
\label{figure0041}
\end{figure}
\FloatBarrier

При анализирането на ръст в популацията понякога е интересно тази информация да се организира по години и да се представи в обща графика. Данните за всяка година могат да се представят с различен цвят, така че да бъде ясно кои линии за кои периоди от време се отнасят. С помощта на библиотеката $lubridate$ в $economics$ данните се добавят две допълнителни колони за година и за месец. С помощта на библиотеката $scales$ се постига по-добро оформление на информацията по осите. 

\begin{figure}[h!]
  \centering
  \includegraphics[width=1.0\linewidth]{pic0042}
  \caption{Визуализиране на приръста по години}
\label{figure0042}
\end{figure}
\FloatBarrier

Важно е да се отбележи, че информацията за годината е от тип $factor$, така че да се използва за определяне на цветовете. Също така на ординатната ос се добавя и запетая, като разделител за хилядите. Като последна декорация е подмяната на текстовете за двете оси. 

\subsection{Тематично оформление}

При генерирането на графики е от съществено значение медията на която тези графики ще бъдат представяни. При визуализация на проектор или монитор може да се използват тематично тъмни\index{графични теми} цветове, а при разпечатване на хартия е по-разумно да се използват светли цветове, така че да се намалява разхода на мастило (Листинг \ref{listing0155}). Каквито и да са нуждите за представяне, в пакетът $ggthemes$ са добавени възможности за цялостно преобразяване на получените графики, чрез избор на теми (Фиг. \ref{figure0043}-\ref{figure0046}).

\begin{lstlisting}[caption=Избор на теми за визуално представяне, label=listing0155]
library(ggthemes)

ggplot(diamonds, aes(x=carat, y=price)) + geom_point(aes(color=color)) + theme_wsj()

ggplot(diamonds, aes(x=carat, y=price)) + geom_point(aes(color=color)) + theme_tufte()

ggplot(diamonds, aes(x=carat, y=price)) + geom_point(aes(color=color)) + theme_excel()

ggplot(diamonds, aes(x=carat, y=price)) + geom_point(aes(color=color)) +  theme_economist() + scale_colour_economist()
\end{lstlisting}

\begin{figure}[h!]
  \centering
  \includegraphics[width=1.0\linewidth]{pic0043}
  \caption{Тема Wall Street Journal}
\label{figure0043}
\end{figure}
\FloatBarrier

\begin{figure}[h!]
  \centering
  \includegraphics[width=1.0\linewidth]{pic0044}
  \caption{Тема Edward Tufte}
\label{figure0044}
\end{figure}
\FloatBarrier

\begin{figure}[h!]
  \centering
  \includegraphics[width=1.0\linewidth]{pic0045}
  \caption{Тема в стил Microsoft Excel}
\label{figure0045}
\end{figure}
\FloatBarrier

\begin{figure}[h!]
  \centering
  \includegraphics[width=1.0\linewidth]{pic0046}
  \caption{Тема Economist}
\label{figure0046}
\end{figure}
\FloatBarrier

\section{Изследване на случайни величини}

Когато се работи с данни за които няма предварителна информация е от съществено значение да се определи какви са параметрите на вероятностното разпределение. С помощта на пакета $Rdice$ е възможно да се генерират експерименти с различни зарове. Чрез генерирането и статистическото изследване на множество случайни събития се осъществява изследване наречено „Монте Карло метод“\index{метод Монте-Карло}.

\begin{lstlisting}[caption=Случайни величини със зарове, label=listing0156]
library(ggplot2)
library(Rdice)

x <- dice.roll(faces=6, dice=1, rolls=100000)

ggplot(data=x$results) + geom_histogram(aes(x=values)) + ggtitle("Single die rolled 100K times.") + xlab("Die Side") + ylab("Outcomes") + scale_x_continuous(breaks=round(seq(min(x$results$values),max(x$results$values),by=0.5)))

x <- dice.roll(faces=6, dice=2, rolls=100000)

ggplot(data=x$results) + geom_histogram(aes(x=(die_1+die_2))) + ggtitle("Two dice rolled 100K times.") + xlab("Dice Sides") + ylab("Outcomes") + scale_x_continuous(breaks=round(seq(min((x$results$die_1+x$results$die_2)),max((x$results$die_1+x$results$die_2)),by=0.5)))

x <- dice.roll(faces=6, dice=6, rolls=100000)
x$results$values = rowSums( x$results[,1:6] )

ggplot(data=x$results) + geom_histogram(aes(x=values)) + ggtitle("Six dice rolled 100K times.") + xlab("Dice Sides") + ylab("Outcomes") + scale_x_continuous(breaks=round(seq(min(x$results$values),max(x$results$values),by=0.5)))

x <- dice.roll(faces=6, dice=10, rolls=100000)
x$results$values = rowSums( x$results[,1:10] )

ggplot(data=x$results) + geom_density(aes(x=values)) + ggtitle("Ten dice rolled 100K times.") + xlab("Dice Sides") + ylab("Outcomes") + scale_x_continuous(breaks=round(seq(min(x$results$values),max(x$results$values),by=0.5)))
\end{lstlisting}

Ако се приеме, че променливата $x$ е резултатът от многократното хвърляне на един математически честен зар, със шест страни, то чрез изчертаване на хистограмата\index{хистограма} може да се добие представа за характера на случайната величина (Листинг \ref{listing0156}). От изчертаната хистограма ясно се вижда, че случайната величина е дискретна и равномерно разпределена (Фиг. \ref{figure0047}).

\begin{figure}[h!]
  \centering
  \includegraphics[width=1.0\linewidth]{pic0047}
  \caption{Хистограма на хвърлянията за един зар}
\label{figure0047}
\end{figure}
\FloatBarrier

Когато същият експеримент бъде повторен, но вместо един зар се използват два зара, ясно се различава, че някои събития стават по-малко вероятни от други и разпределението се превръща в триъгълно (Фиг. \ref{figure0048}).

\begin{figure}[h!]
  \centering
  \includegraphics[width=1.0\linewidth]{pic0048}
  \caption{Хистограма на хвърлянията за два зара}
\label{figure0048}
\end{figure}
\FloatBarrier

При шест зара ясно започва да се различава формата на нормалното разпределение (Фиг. \ref{figure0049}).

\begin{figure}[h!]
  \centering
  \includegraphics[width=1.0\linewidth]{pic0049}
  \caption{Хистограма на хвърлянията за шест зара}
\label{figure0049}
\end{figure}
\FloatBarrier

При десет зара и изчертаване на плътностна диаграма формата на нормалното вероятностно разпределение е ясно забележима (Фиг. \ref{figure0050}).

\begin{figure}[h!]
  \centering
  \includegraphics[width=1.0\linewidth]{pic0050}
  \caption{Плътностна диаграма на хвърлянията за десет зара}
\label{figure0050}
\end{figure}
\FloatBarrier

За изследването на една случайна величина, хистограмата и плътностната диаграма носят първоначална ориентировъчна информация за характеристиките на величината.

\section{Вероятностни разпределения}

В реалната практика от статистическия анализ се наблюдават множество случайни величини, които винаги се подчиняват на някакво вероятностно разпределение\index{вероятностно разпределение}. Определянето на вероятностното разпределение, към което принадлежи случайната величина има изключително важна роля в адекватното и надеждно извършване на статистическия анализ. След определяне на вероятностното разпределение от съществена важност е и определянето на параметрите, които характеризират разпределението.  

\subsection{Нормално разпределение}

Най-често срещаното в природата и най-използваното в статистическия анализ е нормалното разпределение. Също така, познато е и под названието на Гаусово разпределение\index{нормално разпределение} (Формула \ref{equation0002}). 

\begin{equation}
pdf(x) = \frac{1}{{\sigma \sqrt {2\pi } }}e^{{{ - \left( {x - \mu } \right)^2 } \mathord{\left/ {\vphantom {{ - \left( {x - \mu } \right)^2 } {2\sigma ^2 }}} \right. \kern-\nulldelimiterspace} {2\sigma ^2 }}}
\label{equation0002}
\end{equation}
\listofequations{Вероятностна функция на нормално разпределение}

Нормалното разпределение се характеризира с два параметъра - средната стойност $\mu$ и стандартно отклонение $\sigma$. Формата на графиката с която се изобразява нормалното разпределение е като камбана. Средната стойност задава къде се намира върхът на камбаната, по оста $X$, а стандартното отклонение определя широчината на камбаната. 

\begin{lstlisting}[caption=Нормално разпределение, label=listing0157]
library(ggplot2)

values <- rnorm(n=30000, mean=0, sd=0.85)
density <- dnorm( values )
cumulative <- pnorm( values )
quantile <- qnorm( cumulative )

ggplot(data.frame(x=values, y=density)) + aes(x=x, y=y) + geom_line() + labs(x="Normally Distributed Random Values ", y="Density")

ggplot(data.frame(x=values, y=cumulative)) + aes(x=x, y=y) + geom_line() + labs(x="Normally Distributed Random Values ", y="Cumulative Probability")

ggplot(data.frame(x=values, y=quantile)) + aes(x=x, y=y) + geom_line() + labs(x="Normally Distributed Random Values ", y="Quantile")
\end{lstlisting}

В езика R генерирането на нормално разпределени случайни числа става чрез функцията $rnorm$, която получава параметър за брой числа, средна стойност и стандартно отклонение. Подразбиращата се средна стойност е нула, а подразбиращото се стандартно отклонение е единица. Вероятността дадена стойност да бъде генериране се изчислява с функцията $dnorm$, която е полезна при изчертаването на плътностната функция (Фиг. \ref{figure0051}).

\begin{figure}[h!]
  \centering
  \includegraphics[width=1.0\linewidth]{pic0051}
  \caption{Плътностна функция на нормално разпределение}
\label{figure0051}
\end{figure}
\FloatBarrier

От плътностната функция\index{плътностна функция}, чрез интегриране, се получава кумулативната функция\index{кумулативна функция} на разпределението (Формула \ref{equation0003}).

\begin{equation}
cdf(x) = \int_{-\infty}^{a} \frac{1}{{\sigma \sqrt {2\pi } }}e^{{{ - \left( {x - \mu } \right)^2 } \mathord{\left/ {\vphantom {{ - \left( {x - \mu } \right)^2 } {2\sigma ^2 }}} \right. \kern-\nulldelimiterspace} {2\sigma ^2 }}} dx
\label{equation0003}
\end{equation}
\listofequations{Кумулативна функция на нормално разпределение}

Смисълът на кумулативната функция е, че определя вероятността да се падне число по-малко от зададеното в интервала (Фиг. \ref{figure0052}).

\begin{figure}[h!]
  \centering
  \includegraphics[width=1.0\linewidth]{pic0052}
  \caption{Комулативна функция на нормално разпределение}
\label{figure0052}
\end{figure}
\FloatBarrier

Обратната функция на $pnorm$ е $qnorm$ и тя изчислява квантила, ако е известна кумулативната вероятност.

\subsection{Биномно разпределение}

Биномното разпределение\index{биномно разпределение} (Формула \ref{equation0004}) е добре представено в R с помощта на серия функции по аналогия с функциите за нормално разпределение. 

\begin{equation}
pdf(x) = \binom{n}{x}p^{x}(1-p)^{n-x}
\label{equation0004}
\end{equation}
\listofequations{Вероятностна функция на биномно разпределение}

Където първият множител е биномен коефициент\index{биномни коефициенти} изчисляван по Формула \ref{equation0005}.

\begin{equation}
\binom{n}{x} = \frac{n!}{x!(n-x)!}
\label{equation0005}
\end{equation}
\listofequations{Биномен коефициент}

Параметърът $n$ определя броя опити, а параметърът $p$ задава вероятността за успех при единичен опит. Средната на биномното разпределение е $np$, а вариацията $np(1-p)$. Когато параметърът $n$ има стойност единица биномното разпределение се трансформира в бернулиево разпределение\index{бернулиево разпределение}. 

\begin{lstlisting}[caption=Биномно разпределение, label=listing0158]
library(ggplot2)

values <- rbinom(n=10000, size=7, prob=0.35)
density <- dbinom(x=2, size=7, prob=0.35)
cumulative <- pbinom(q=2, size=7, prob=0.35)
quantile <- qbinom(p=0.15, size=7, prob=0.35)

ggplot(data.frame(x=values)) + aes(x=x) + geom_histogram() + labs(x="Binomial Distributed Random Values ", y="Count")

density
cumulative
quantile
\end{lstlisting}

При биномното разпределение не просто се генерират случайни числа, а се генерират броя на успешните независими бернулиеви експеримента. В езика R за да се генерират биномно разпределени нормални числа се използва функцията $rbinom$ (Листинг \ref{listing0158}). Биномното разпределение е дискретно вероятностно разпределени, тъй като отразява определен брой успешни изходи от експеримент с предварително зададена вероятност за успех. Поради тази причина визуализацията става с помощта на хистограма (Фиг. \ref{figure0053}).

\begin{figure}[h!]
  \centering
  \includegraphics[width=1.0\linewidth]{pic0053}
  \caption{Хистограма на биномно разпределение}
\label{figure0053}
\end{figure}
\FloatBarrier

Освен броя случайни числа които трябва да се генерират към функцията $rbinom$ се подава параметър за броя експерименти и параметър за вероятността на успех при единичен експеримент. При значително увеличаване на броя експерименти биномното разпределение започва да клони към нормално разпределение. 

\begin{equation}
cdf(x) = \sum_{i=0}^{a}\binom{n}{i}p^{i}(1-p)^{n-i}
\label{equation0006}
\end{equation}
\listofequations{Кумулативна функция на биномно разпределение}

С функцията dbinom може да се провери плътността за определена стойност (Формула \ref{equation0006}), а с pbinom кумулативната стойност. И двете функции могат да се използват с вектор от стойности. 

\subsection{Поасоново разпределение}

Популярно в практиката е също и разпределението на Поасон\index{поасоново разпределение}. То има само един параметър $\lambda$, който отразява едновременно средната стойност и дисперсията (Формула \ref{equation0007}). 

\begin{equation}
pdf(x) = \frac{\lambda^{x}e^{-\lambda}}{x!}
\label{equation0007}
\end{equation}
\listofequations{Вероятностна функция на поасоново разпределение}

Разпределението е дискретно и намира приложение при случайни променливи, които отразяват случването на брой случайни събития в зададен интервал от време. 

\begin{equation}
cdf(x) = \sum_{i=0}^{a}\frac{\lambda^{i}e^{-\lambda}}{i!}
\label{equation0008}
\end{equation}
\listofequations{Кумулативна функция на поасоново разпределение}

Както и при повечето други вероятностни разпределения, това също започва да клони към нормалното разпределени когато параметърът $\lambda$ нарасне до големи стойности. 

\begin{lstlisting}[caption=Разпределение на Поасон, label=listing0159]
library(ggplot2)

values <- rpois(n=1000, lambda=0.8)
density <- dpois(x=2, lambda=0.8)
cumulative <- ppois(q=2, lambda=0.8)
quantile <- qpois(p=0.75, lambda=0.8)

ggplot(data.frame(x=values)) + aes(x=x) + geom_histogram() + labs(x="Poisson Distributed Random Values ", y="Count")

density
cumulative
quantile
\end{lstlisting}

Визуализацията на поасоновото разпределение става с помощта на хистограма\index{хистограма} (Фиг. \ref{figure0054}).

\begin{figure}[h!]
  \centering
  \includegraphics[width=1.0\linewidth]{pic0054}
  \caption{Хистограма на поасоново разпределение}
\label{figure0054}
\end{figure}
\FloatBarrier

С функцията $dpois$ може да се провери плътността за определена стойност (Формула \ref{equation0008}), а с $ppois$ кумулативната стойност. И двете функции могат да се използват с вектор от стойности. 

\subsection{Други разпределения}

Пакетът R предлага богат набор от вероятностни разпределения (Фиг. \ref{figure0055}). Една част от тези разпределения са широко използвани, други не чак толкова.

\begin{figure}[h!]
  \centering
  \includegraphics[width=1.0\linewidth]{pic0055}
  \caption{Списък с най-използваните вероятностни разпределения}
\label{figure0055}
\end{figure}
\FloatBarrier

\section*{Заключение}

Визуалното представяне на числените данни повишава степента за възприемане на представяната информация. Колкото по-големи са възможностите на пакетите за визуално представяне, толкова повече възможности биха имали хората работещи с информация и нейното представяне пред широка аудитория. При анализа на случайни величини, визуализацията с хистограма и плътностна функция може да даде най-груба първоначална представа за параметрите на случайната величина. Информация, която може да бъде изключително полезна при избора на по-сложни методи за статистически анализ. От своя страна, визуализацията на случайностите най-удачно се представя, чрез някои от най-използваните вероятностни разпределения. 


\newpage
\chapter{Статистическа обработка на данните}
\label{chapter09}

Статистическата обработка на данни се състои основно от два вида статистика – описателна статистика и сравнителна статистика. При описателната статистика се изчисляват определени параметри описващи характеристики на събраните данни, докато при сравнителната статистика се извършва сравнение между някои от описателните параметри на данните. 

\section{Описателна статистика}

Параметрите при описателната статистика основно са свързани с някакво централно групиране на данните и някакво разпръскване (дисперсия) около централното групиране. Параметри за централно групиране са средната стойност, медианата и модата, а параметри за разпръскване са дисперсията и стандартното отклонение.

\begin{lstlisting}[caption=Генериране на извадка от случайни числа, label=listing0160]
v1 <- round( rnorm(100, mean=62, sd=72) )

v2 <- v1

v2[sample(x=1:100, size=15, replace=FALSE)] <- NA

w1 <- 1 / sample(x=1:100, size=100, replace=TRUE)
\end{lstlisting}

За да бъдат илюстрирани възможностите на R за пресмятане на описателни статистики се използва извадка от 100 нормално разпределени случайни числа (Листинг \ref{listing0160}). Често в реалната практика данните съдържат липсващи измервания. При такава ситуация трябва да се вземат допълнителни мерки за пресмятане на описателните статистики. Функцията sample в R позволява на случаен принцип част от стойностите в определен вектор да бъдат избрани на случаен принцип, при равномерно вероятностно разпределение. Тази възможност се използва за замяна на 15\% от генерираните данни с липсваща стойност (Листинг \ref{listing0160}). Параметърът replace указва дали определено число може да бъде избрано повторно или не.

\subsection{Средна стойност}

Средната стойност е най-често използваната статистика и представлява сумата от стойностите разделена на общия брой стойности (Листинг. \ref{listing0161}).

\begin{lstlisting}[caption=Средна стойност, label=listing0161]
sum(v1) / length(v1)

mean( v1 )

mean( v2 )

mean(v2, na.rm=TRUE)
\end{lstlisting}

Когато липсват стойности при проведените измервания е невъзможно да се изчисли средната стойност без да се вземе решение как да се обработят липсващите числа. Има различни подходи за обработката на липсите, като интерполация или премахване. Независимо кой подход бъде избран, то той неизбежно води до внасяне на допълнителна грешка при пресмятанията. Ако бъде извършена интерполация, то съседните измервания биха внесли грешка в липсващите стойности. Ако бъдат премахнати липсващите измервания, то размерът на извадката намалява, а от там се увеличава и неточността на последващите пресмятания. 

В някои ситуации отделните измерени стойности имат различна тежест и се налага те да участват с различен коефициент при пресмятането на средната стойност (Листинг \ref{listing0162}). Такава е ситуацията когато се пресмята бал за прием в учебно заведение. Различните компоненти формиращи бал на всеки кандидат участват с различна тежест.

\begin{lstlisting}[caption=Претеглена средна стойност, label=listing0162]
weighted.mean(x=v1, w=w1)
\end{lstlisting}

\subsection{Минимална стойност, максимална стойност, медиана и мода}

В практиката често от значение е диапазонът в който се разпростират данните. В програмния пакет R този диапазон лесно се установява с функциите min и max (Листинг \ref{listing0163}).

\begin{lstlisting}[caption={Минимум, максимум, медиана и мода}, label=listing0163]
min( v1 )

max( v1 )

median( v1 )

# Mode calculation.
unique(v1)[ which.max( tabulate(match(v1,unique(v1)) ) ) ]
\end{lstlisting}

Един от основните недостатъци на средната стойност е, че при наличието на екстремално различни отделни измервания силно могат да повлияят върху общото пресмятане на средната стойност. Поради тази причина в практиката много често се използва медианата, а не средната стойност (Листинг \ref{listing0163}). За да се пресметне медианата данните се сортират. При нечетен брой стойности медианата е стойността на средния елемент. При четен брой стойности медианата е средно аритметично между двата елемента в средата. 

Модата е параметър, който отразява най-често срещаната стойност в множеството от данни. Освен числена стойност модата може да има и символна стойност, според характера на самите данни. В програмния пакет R няма функция, която да изчислява модата, но тя лесно може да се пресметне с комбинация от извикването на други функции (Листинг \ref{listing0163}).

\subsection{Дисперсия и стандартно отклонение}

След като бъде установено някакво централно групиране в данните е от съществено значение по какъв начин данните са разпръснати около това централно групиране. Изчисляването на дисперсията спомага за изследване на разпръскването (Листинг \ref{listing0164}). Дисперсията се изчислява като сума от квадрата на разликите между всяка стойност и средната, разделена на броя стойности минус едно. 

\begin{lstlisting}[caption=Дисперсия и стандартно отклонение, label=listing0164]
sum( (v1-mean(v1))^2 ) / (length(v1) - 1)

var( v1 )

sqrt( var(v1) )

sd( v1 )
\end{lstlisting}

Основен недостатък на дисперсията е, че се изчислява с повдигане на втора степен и полученият резултат е несравним с оригиналните измервания. Примерно при серия измервания в метри резултатът от изчислението на дисперсията би имал смисъла на квадратни метри. За да бъдат сравними стойностите е достатъчно дисперсията да се подложи на корен квадратен, което коригира резултата получен с повдигане на втора степен. Резултатът от квадратния корен на дисперсията се нарича стандартно отклонение (Листинг \ref{listing0164}).

\subsection{Квантили и обобщение}

Квантилите определят каква част от измерените стойности попадат в определен процент от вероятностното разпределение (Листинг \ref{listing0164}).

\begin{lstlisting}[caption=Квантили и обобщение, label=listing0165]
quantile(v1, probs=c(0.1, 0.2, 0.3, 0.4, 0.5, 0.6, 0.7, 0.8, 0.9))
#  10%   20%   30%   40%   50%   60%   70%   80%   90% 
#-28.0  -0.4  19.0  41.6  66.5  84.4 110.6 121.8 150.2 

summary( v1 )
\end{lstlisting}

Програмният пакет R предоставя и обобщаваща функция за описателните статистики наречена summary и тя визуализира стойностите за минимална, максимална, медиана, средна, първи и трети квантил.

\section{Сравнителна статистика}



\newpage
\chapter{Приближени пресмятания - подходи, методи, алгоритми}
\label{chapter10}
\thispagestyle{empty}


\newpage
\chapter{Предпечатна подготовка за представяне на ресултатите}
\label{chapter11}
\thispagestyle{empty}


\newpage
\addcontentsline{toc}{chapter}{Заключение}
\chapter*{Заключение}
\thispagestyle{empty}

Без да претендира за изчерпателност настоящото учебно помагало прави въведение в статистическата обработка на данни с помощта на един от най-популярните програмни продукти, а именно програмния пакет $R$. В практическата работа на ученици, студенти, докторанти и специалисти по статистика се срещат множество особености, които до голяма степен са засегнати в изложения материал. Макар и да съществуват множество алтернативни програмни продукти, като $SPSS$, $Matlab$ и $Mathematica$, програмният продукт $R$ се отличава с финансова ефективност и отворен модел за разширяване. В учебното помагало не са засегнати темите за напреднали, тъй като целите на авторите са основно да провокират широката аудитория. Темите за напреднали могат да бъдат открити в множество учебници и книги в чуждоезичната литература, както и в голям брой видео уроци. 



% Списък с използвана литература и източници на информация.
\newpage
\begin{thebibliography}{99}
\addcontentsline{toc}{chapter}{Библиография}

\bibitem{gpl2} GNU General Public License, version 2, Free Software Foundation, \\\texttt{http://www.gnu.org/licenses/old-licenses/gpl-2.0.html}

\bibitem{hnot} Hungarian notation, Wikimedia Foundation, Inc., \\\texttt{http://en.wikipedia.org/wiki/Hungarian\_notation}

\bibitem{spss} Zagumny, M.: The SPSS Boo: A Student Guide to the Statistical Package for the Social Sciences, iUniverse, 2001.

\bibitem{matlab} Higham. D., Higham, N.: MATLAB Guide 2nd Edition, SIAM: Society for Industrial and Applied Mathematics, 2005.

\bibitem{mathematica} Wolfram, S.: The MATHEMATICA Book, Version 4 4th Edition, Cambridge University Press, 1999.

\bibitem{sas} Cody, R., Smith, J: Applied Statistics and the SAS Programming Language 5th Edition, Pearson, 2005.

\bibitem{cpp} Jamsa, K., Klander, L.: Jamsa's C/C++ Programmer's Bible 1st Edition, Cengage Learning, 1997.

\bibitem{java} Arnold, K.,  Gosling, J.: The Java Programming Language (Java Series), Addison-Wesley, 1997.

\bibitem{csharp} Collingbourne, H.: The Little Book Of C\# Programming: Learn To Program C-Sharp For Beginners, Dark Neon, 2019.

\bibitem{excel} Salkind, N.: Excel Statistics: A Quick Guide Second 2nd Edition, SAGE Publications, Inc, 2012.

\bibitem{latex} Lamport, L.: LaTeX A Document Preparation System 2nd Edition, Addison-Wesley Professional, 1994.

\bibitem{php} Tatroe, K., MacIntyre, P., Lerdorf, R.: Programming PHP, Creating Dynamic Web Pages 3d Edition, O'Reilly Media, 2013.

\bibitem{python} Harvard, C.: Python Programming A Smarter And Faster Way To Learn Python In 7 Days: With Practical Exercises, Interview Questions, Tips And Tricks, Independently published, 2019.

\bibitem{javascript} Raasch, J.: JavaScript Programming Pushing the Limits 1st Edition, Wiley, 2013.

\bibitem{github} Guthals, S., Haack, P.: GitHub For Dummies (For Dummies (Computer/Tech)) 1st Edition, For Dummies, 2019.

\bibitem{json} iCode Academy: Json for Beginners Your Guide to Easily Learn Json In 7 Days, Independently published, 2017.

\bibitem{anova} Goos, P., Meintrup, D.: Statistics with JMP: Hypothesis Tests, ANOVA and Regression 1st Edition, Wiley, 2016. 

\bibitem{rstudio} Verzani, J.: Getting Started with RStudio An Integrated Development Environment for R 1st Edition,  O'Reilly Media, 2011.

\bibitem{markdown} Alam, K.: Learn Markdown The Complete Guide on Markdown Formatting, Creative Content Media, 2018.

\bibitem{html} Duckett, J.: HTML and CSS Design and Build Websites 1st Edition, John Wiley \& Sons, 2011.

\bibitem{sql} Turner, R.: SQL The Ultimate Beginner\'s Guide to Learn SQL Programming Step by Step, Amazon Digital Services LLC, 2019.

\bibitem{ann} Jones, H.: Neural Networks An Essential Beginners Guide to Artificial Neural Networks and their Role in Machine Learning and Artificial Intelligence, Amazon Digital Services LLC, 2018.

\bibitem{ga} Goldberg, D.: Genetic Algorithms in Search, Optimization, and Machine Learning 1st Edition, Addison-Wesley Professional, 1989.

\bibitem{montecarlo} Rubinstein, R., Kroese, D.: Simulation and the Monte Carlo Method 2nd Edition, Wiley-Interscience, 2007.

\end{thebibliography}

% Азбучен указател на използваните термини.
\newpage
\printindex

\includepdf[pages=-,height=320mm]{images/back}
\end{document}
