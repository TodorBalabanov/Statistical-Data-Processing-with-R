\documentclass[a4paper,12pt]{minimal}

\usepackage[12pt]{moresize}
\usepackage{hyperref}
\usepackage[utf8x]{inputenc}
\usepackage[english,bulgarian]{babel}

\hypersetup{
  pdfauthor = {Тодор Балабанов},
  pdfkeywords = {},
  pdftitle = {Тест по R}
}

\begin{document}

\begin{center}
{\Huge }
\end{center}

\begin{Form}[action=mailto:todor.balabanov@gmail.com,encoding=text,method=post]

\begin{tabular}{c c c} \\ 
	\TextField[name=first_name]{Име} & \TextField[name=second_name]{Презиме} & \TextField[name=third_name]{Фамилия} \\ \\
	\CheckBox[name=phd_student,width=3mm]{Докторант} & \CheckBox[name=bas_employee,width=3mm]{Служител в БАН} &  \CheckBox[name=other_student,width=3mm]{Друго} \\ \\
	\TextField[name=email]{Имейл} & \TextField[name=phone]{Телефон} & \TextField[name=mobile]{Мобилен}  \\ \\ 
\end{tabular}

1. Функциите в R са?
\ChoiceMenu[radio,name=question_01]{\mbox{}}{\\
Последователност от мнемонични кодове;=a,\\
Последователност от аритметични операции;=b,\\
Предикатна логика от първи ред;=c,\\
Последователност от инструкции;=d}\\%

2. Векторите в R са?
\ChoiceMenu[radio,name=question_02]{\mbox{}}{\\
Колекция от еднотипни елементи;=a,\\%
Дървовидна структура данни;=b,\\
Насочен тегловен граф;=c,\\
Масив с фиксирана дължина;=d}\\

3. Липсваща стойност в R се отбелязва с?
\ChoiceMenu[radio,name=question_03]{\mbox{}}{\\
NULL;=a,\\
NA;=b,\\%
NIL;=c,\\
MISSING;=d}\\

4. Данни в организация тип „рамка“ в R се представят чрез?
\ChoiceMenu[radio,name=question_04]{\mbox{}}{\\
data.frame;=a,\\%
factor;=b,\\
list;=c,\\
vector;=d}\\

5. Списъците в R са?
\ChoiceMenu[radio,name=question_05]{\mbox{}}{\\
Масиви с еднородни елементи;=a,\\
Двоично балансирани дървета;=b,\\
Контейнери с разнородни по тип елементи;=c,\\%
Тегловни графи с цикли;=d}\\

6. Матриците в R наподобяват?
\ChoiceMenu[radio,name=question_06]{\mbox{}}{\\
Динамични структури от данни;=a,\\
Скаларна стойност;=b,\\
Едномерен масив;=c,\\
Двумерен масив;=d}\\%

7. Масивите в R имат?
\ChoiceMenu[radio,name=question_07]{\mbox{}}{\\
Еднотипни елементи;=a,\\%
Разнотипни елементи;=b,\\
Нелинейна структура;=c,\\
Многолинейна структура;=d}\\

8. Множество функции в R?
\ChoiceMenu[radio,name=question_08]{\mbox{}}{\\
Не приемат входни стойности и не връщат стойност;=a,\\
Не допускат използването на вектори и скаларни стойности;=b,\\
Са организирани да изпълняват векторни пресмятания;=c,\\%
Нямат приложение в практиката;=d}\\

9. Много от операциите в R?
\ChoiceMenu[radio,name=question_09]{\mbox{}}{\\
Нямат практическо приложение;=a,\\
Са абсолютно неприложими за пресмятания със скаларни стойности;=b,\\
Са реализирани за векторни пресмятания;=c,\\%
Не могат да участват в сложни изрази;=d}\\

\Submit{Изпрати}
\Reset{Изчисти}

\end{Form}

\end{document}
