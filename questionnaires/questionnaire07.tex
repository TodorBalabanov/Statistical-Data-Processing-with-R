\documentclass[a4paper,12pt]{minimal}

\usepackage[12pt]{moresize}
\usepackage{hyperref}
\usepackage[utf8x]{inputenc}
\usepackage[english,bulgarian]{babel}

\hypersetup{
  pdfauthor = {Тодор Балабанов},
  pdfkeywords = {},
  pdftitle = {Тест по R}
}

\begin{document}

\begin{center}
{\Huge Реорганизация на данните и обработка на символни низове}
\end{center}

\begin{Form}[action=mailto:todor.balabanov@gmail.com,encoding=text,method=post]

\begin{tabular}{c c c} \\ 
	\TextField[name=first_name]{Име} & \TextField[name=second_name]{Презиме} & \TextField[name=third_name]{Фамилия} \\ \\
	\CheckBox[name=phd_student,width=3mm]{Докторант} & \CheckBox[name=bas_employee,width=3mm]{Служител в БАН} &  \CheckBox[name=other_student,width=3mm]{Друго} \\ \\
	\TextField[name=email]{Имейл} & \TextField[name=phone]{Телефон} & \TextField[name=mobile]{Мобилен}  \\ \\ 
\end{tabular}

1. Функцията cbind се използва за?
\ChoiceMenu[radio,name=question_01]{\mbox{}}{\\
Обединяване на редове;=a,\\
Обединяване на колони;=b,\\%
Разединяване на колони;=c,\\
Разединяване на редове;=d}\\

2. Функцията rbind се използва за?
\ChoiceMenu[radio,name=question_02]{\mbox{}}{\\
Разединяване на редове;=a,\\
Разединяване на колони;=b,\\
Обединяване на редове;=c,\\%
Обединяване на колони;=d}\\

3. Функцията merge се използва за?
\ChoiceMenu[radio,name=question_03]{\mbox{}}{\\
Интерполиране на данни;=a,\\
Транспониране на данни;=b,\\
Сливане на данни по ключ;=c,\\%
Разединение на данни;=d}\\

4. Функцията join е по-добра от функцията merge с?
\ChoiceMenu[radio,name=question_04]{\mbox{}}{\\
По-подредения код;=a,\\
Бързодействието си;=b,\\%
Обработката на изключителни ситуации;=c,\\
По-удачно избраното име;=d}\\

5. Транспониране на данните представлява?
\ChoiceMenu[radio,name=question_05]{\mbox{}}{\\
Размяна на редовете с колони;=a,\\%
Размяна само по колони;=b,\\
Размяна само по редове;=c,\\
Изчисляване на допълнителни стойности по редове и колони;=d}\\

6. Конкатенация на символни низове е?
\ChoiceMenu[radio,name=question_06]{\mbox{}}{\\
Циклично отместване на два символни низа;=a,\\
Преобръщане на два символни низа;=b,\\
Разделяне на два символни низа;=c,\\
Слепване на два символни низа;=d}\\%

7. Функцията sprintf се използва за?
\ChoiceMenu[radio,name=question_07]{\mbox{}}{\\
Прочитане на стойности от символен низ;=a,\\
Отпечатване на стойности в символен низ;=b,\\%
Преобръщане на символен низ;=c,\\
Реверсивно отместване в символен низ;=d}\\

8. Функцията str\_split се изпозлва за?
\ChoiceMenu[radio,name=question_08]{\mbox{}}{\\
Сортиране на буквите в символен низ;=a,\\
Разбъркване на буквите в символен низ;=b,\\
Сливане на множество символни низове в един;=c,\\
Раздробяване на символен низ по зададен разделител;=d}\\%

9. Функцията str\_replace\_all служи за?
\ChoiceMenu[radio,name=question_09]{\mbox{}}{\\
Извличане на шаблон при всяко срещане в символен низ;=a,\\
Изтриване на шаблон при всяко срещане в символен низ;=b,\\
Заместване на шаблон при всяко срещане в символен низ;=c,\\%
Разбъркване на шаблон при всяко срещане в символен низ;=d}\\

\Submit{Изпрати}
\Reset{Изчисти}\\ \\

{\ssmall This teaching material is funded by Velbazhd Software LLC and it is partially supported by the Bulgarian Ministry of Education and Science (contract D01–205/23.11.2018) under the National Scientific Program ``Information and Communication Technologies for a Single Digital Market in Science, Education and Security (ICTinSES)'', approved by DCM \# 577/17.08.2018.}

\end{Form}

\end{document}
