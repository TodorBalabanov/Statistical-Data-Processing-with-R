\documentclass[a4paper,12pt]{minimal}

\usepackage[12pt]{moresize}
\usepackage{hyperref}
\usepackage[utf8x]{inputenc}
\usepackage[english,bulgarian]{babel}

\hypersetup{
  pdfauthor = {Тодор Балабанов},
  pdfkeywords = {},
  pdftitle = {Тест по R}
}

\begin{document}

\begin{center}
{\Huge Статистическа обработка на данните}
\end{center}

\begin{Form}[action=mailto:todor.balabanov@gmail.com,encoding=text,method=post]

\begin{tabular}{c c c} \\ 
	\TextField[name=first_name]{Име} & \TextField[name=second_name]{Презиме} & \TextField[name=third_name]{Фамилия} \\ \\
	\CheckBox[name=phd_student,width=3mm]{Докторант} & \CheckBox[name=bas_employee,width=3mm]{Служител в БАН} &  \CheckBox[name=other_student,width=3mm]{Друго} \\ \\
	\TextField[name=email]{Имейл} & \TextField[name=phone]{Телефон} & \TextField[name=mobile]{Мобилен}  \\ \\ 
\end{tabular}

1. Функцията sample служи за генериране на?
\ChoiceMenu[radio,name=question_01]{\mbox{}}{\\
Извадка от случайни числа;=a,\\%
Поредица от нарастващи стойности;=b,\\
Поредица от намаляващи стойности;=c,\\
Поредица от минимално-максимални стойности;=d}\\

2. С функцията mean се изчислява?
\ChoiceMenu[radio,name=question_02]{\mbox{}}{\\
Средна стойност;=a,\\%
Медиана;=b,\\
Мода;=c,\\
Дисперсия;=d}\\

3. С функцията median се изчислява?
\ChoiceMenu[radio,name=question_03]{\mbox{}}{\\
Средна стойност;=a,\\
Медиана;=b,\\%
Мода;=c,\\
Дисперсия;=d}\\

4. Стандартното отклонение е корен квадратен от?
\ChoiceMenu[radio,name=question_04]{\mbox{}}{\\
Дисперсията;=a,\\%
Модата;=b,\\
Медианата;=c,\\
Ковариацията;=d}\\

5. Квантилът определя каква част от измерените стойности попадат в?
\ChoiceMenu[radio,name=question_05]{\mbox{}}{\\
Оптимални стойности от вероятностното разпределение;=a,\\
Минимални стойности от вероятностното разпределение;=b,\\
Максимални стойности от вероятностното разпределение;=c,\\
Определен процент от вероятностното разпределение;=d}\\%

6. Корелационният коефициент е?
\ChoiceMenu[radio,name=question_06]{\mbox{}}{\\
В интервала от -100\% до +100\%;=a,\\
В интервала от минус безкрайност до плюс безкрайност;=b,\\
В интервала от -1 до +1;=c,\\%
Силно независим от двете променливи;=d}\\

7. Стюдънт тестът изследва?
\ChoiceMenu[radio,name=question_07]{\mbox{}}{\\
Формата на вероятностното разпределение за две случайни променливи;=a,\\
Стандартните отклонения на две случайни променливи;=b,\\
Средните на две нормални случайни променливи;=c,\\%
Честотните зависимости между две случайни променливи;=d}\\

8. Функцията aov служи за?
\ChoiceMenu[radio,name=question_08]{\mbox{}}{\\
Спектографски анализ;=a,\\
Дисперсионен анализ;=b,\\%
Графологичен анализ;=c,\\
Семантично-синтактичен анализ;=d}\\

9. Функцията lm служи за анализ с?
\ChoiceMenu[radio,name=question_09]{\mbox{}}{\\
Експоненциална регресия;=a,\\
Линейна регресия;=b,\\%
Нелинейна регресия;=c,\\
Логаритмична регресия;=d}\\

\Submit{Изпрати}
\Reset{Изчисти}\\ \\

{\ssmall This teaching material is funded by Velbazhd Software LLC and it is partially supported by the Bulgarian Ministry of Education and Science (contract D01–205/23.11.2018) under the National Scientific Program ``Information and Communication Technologies for a Single Digital Market in Science, Education and Security (ICTinSES)'', approved by DCM \# 577/17.08.2018.}

\end{Form}

\end{document}
