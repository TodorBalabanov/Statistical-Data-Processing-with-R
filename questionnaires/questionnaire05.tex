\documentclass[a4paper,12pt]{minimal}

\usepackage[12pt]{moresize}
\usepackage{hyperref}
\usepackage[utf8x]{inputenc}
\usepackage[english,bulgarian]{babel}

\hypersetup{
  pdfauthor = {Тодор Балабанов},
  pdfkeywords = {},
  pdftitle = {Тест по R}
}

\begin{document}

\begin{center}
{\Huge Оператори за контрол на изпълнението и потребителски функции}
\end{center}

\begin{Form}[action=mailto:todor.balabanov@gmail.com,encoding=text,method=post]

\begin{tabular}{c c c} \\ 
	\TextField[name=first_name]{Име} & \TextField[name=second_name]{Презиме} & \TextField[name=third_name]{Фамилия} \\ \\
	\CheckBox[name=phd_student,width=3mm]{Докторант} & \CheckBox[name=bas_employee,width=3mm]{Служител в БАН} &  \CheckBox[name=other_student,width=3mm]{Друго} \\ \\
	\TextField[name=email]{Имейл} & \TextField[name=phone]{Телефон} & \TextField[name=mobile]{Мобилен}  \\ \\ 
\end{tabular}

1. Коя от следните ключови думи не се използва в конструкция за преход?
\ChoiceMenu[radio,name=question_01]{\mbox{}}{\\
if;=a,\\
else;=b,\\
switch;=c,\\
volatile;=d}\\%

2. В заглавната част на оператора за условен преход if?
\ChoiceMenu[radio,name=question_02]{\mbox{}}{\\
Могат да се използват сложни логически изрази;=a,\\%
Не могат да се използват логически изрази;=b,\\
Законите на Де Морган са неприложими;=c,\\
Булевата алгебра е неприложима;=d}\\

3. Програмната конструкция else?
\ChoiceMenu[radio,name=question_03]{\mbox{}}{\\
Няма самостоятелна форма на употреба;=a,\\%
Се използва само и единствено самостоятелно;=b,\\
Изпълнява основна роля при реализацията на цикли;=c,\\
Прекъсва аварийно изпълнението на програмния скрипт;=d}\\

4. Цикълът for е?
\ChoiceMenu[radio,name=question_04]{\mbox{}}{\\
Със следусловие;=a,\\
С предусловие;=b,\\%
Конструкция за рекурсия;=c,\\
Конструкция за аварийно прекратяване на програмния скрипт;=d}\\

5. Конструкцията while служи за?
\ChoiceMenu[radio,name=question_05]{\mbox{}}{\\
Рекурсия;=a,\\
Транслитерация;=b,\\
Аварийно прекъсване;=c,\\
Цикъл;=d}\\%

6. Ключовата дума next се използва за?
\ChoiceMenu[radio,name=question_06]{\mbox{}}{\\
Прекратяване на циклите;=a,\\
Прекратяване на текущата итерация в цикли;=b,\\%
Връщане на стойност от функция;=c,\\
Извикване на потребителски дефинирана функция;=d}\\

7. Функциите на програмния език в R?
\ChoiceMenu[radio,name=question_07]{\mbox{}}{\\
Нямат реално практическо приложение;=a,\\
Са ламбда изрази;=b,\\
По своята същност са обекти;=c,\\%
Не позволяват рекурсивни извиквания;=d}\\

8. При извикване аргументите на функциите в R?
\ChoiceMenu[radio,name=question_08]{\mbox{}}{\\
Не могат да се разместват;=a,\\
Могат да се разместват;=b,\\%
Не могат да се подават;=c,\\
Задължително трябва да се използват в тялото на функцията;=d}\\

9. Функциите в R?
\ChoiceMenu[radio,name=question_09]{\mbox{}}{\\
Не могат да получават променлив брой аргументи;=a,\\
Не могат да връщат стойност;=b,\\
Не могат да се подават като параметър на друга функция;=c,\\
Могат да са подават като параметър на друга функция;=d}\\%

\Submit{Изпрати}
\Reset{Изчисти}\\ \\

{\ssmall This teaching materials is funded by Velbazhd Software LLC and it is partially supported by the Bulgarian Ministry of Education and Science (contract D01–205/23.11.2018) under the National Scientific Program ``Information and Communication Technologies for a Single Digital Market in Science, Education and Security (ICTinSES)'', approved by DCM \# 577/17.08.2018.}

\end{Form}

\end{document}
