\documentclass[a4paper,12pt]{minimal}

\usepackage[12pt]{moresize}
\usepackage{hyperref}
\usepackage[utf8x]{inputenc}
\usepackage[english,bulgarian]{babel}

\hypersetup{
  pdfauthor = {Тодор Балабанов},
  pdfkeywords = {},
  pdftitle = {Тест по R}
}

\begin{document}

\begin{center}
{\Huge Оформление на резултатите за печатно и електронно представяне}
\end{center}

\begin{Form}[action=mailto:todor.balabanov@gmail.com,encoding=text,method=post]

\begin{tabular}{c c c} \\ 
	\TextField[name=first_name]{Име} & \TextField[name=second_name]{Презиме} & \TextField[name=third_name]{Фамилия} \\ \\
	\CheckBox[name=phd_student,width=3mm]{Докторант} & \CheckBox[name=bas_employee,width=3mm]{Служител в БАН} &  \CheckBox[name=other_student,width=3mm]{Друго} \\ \\
	\TextField[name=email]{Имейл} & \TextField[name=phone]{Телефон} & \TextField[name=mobile]{Мобилен}  \\ \\ 
\end{tabular}

1. LaTeX е?
\ChoiceMenu[radio,name=question_01]{\mbox{}}{\\
Тагиращ език за оформление на текстови документи;=a,\\%
Програмен език за предикатна логика;=b,\\
Декларативен език за работа с релационни бази данни;=c,\\
Асемблерен език за програмиране на микроконтролери;=d}\\

2. RMarkdown е?
\ChoiceMenu[radio,name=question_02]{\mbox{}}{\\
Език за мрежова комуникация;=a,\\
Език за графично оформление;=b,\\
Опростен аслеблерен език;=c,\\
Опростен език за тагиране;=d}\\%

3. С пакета knitr се изчисляват?
\ChoiceMenu[radio,name=question_03]{\mbox{}}{\\
Логически изрази;=a,\\
Асемблер инструкциите за изпълнение от процесора;=b,\\
R кода за заместване в текстов документ;=c,\\%
Заявки към релационна база данни;=d}\\

4. Софтуерът pandoc се използва за?
\ChoiceMenu[radio,name=question_04]{\mbox{}}{\\
Звуково оформление;=a,\\
Триизмерна визуализация;=b,\\
Трансформиране на файлови формати;=c,\\%
Превод на естествени езици;=d}\\

5. С функцията render се генерира?
\ChoiceMenu[radio,name=question_05]{\mbox{}}{\\
Интерактивни документи от тип PPTX;=a,\\
Статични документи от тип PDF;=b,\\%
Интерактивни документи от тип Java Applet;=c,\\
Интерактивни документи от тип Adobe Shockwave;=d}\\

6. С пакета htmlwidgets могат да се създават?
\ChoiceMenu[radio,name=question_06]{\mbox{}}{\\
Статични документи от тип GCode;=a,\\
Статични документи от тип ODF;=b,\\
Статични документи от тип DOCX;=c,\\
Динамични документи с HTML и JavaScript;=d}\\%

7. С функцията kable данните се оформят?
\ChoiceMenu[radio,name=question_07]{\mbox{}}{\\
Таблично;=a,\\%
Експоненциално;=b,\\
Логаритмично;=c,\\
Диференциално;=d}\\

8. Пакетът DT добавя?
\ChoiceMenu[radio,name=question_08]{\mbox{}}{\\
Дедуктивност в табличното представане;=a,\\
Полиморфичност в табличното представяне;=b,\\
Статичност в табличното представяне;=c,\\
Интерактивност в табличното представяне;=d}\\%

9. Пакетът d3heatmap дава възможност за създаване на?
\ChoiceMenu[radio,name=question_09]{\mbox{}}{\\
Манипулативни рекурсивни връзки;=a,\\
Интуитивни семантични връзки;=b,\\
Интерактивни топлинни карти;=c,\\%
Репродуктивни описания;=d}\\

\Submit{Изпрати}
\Reset{Изчисти}\\ \\

{\ssmall This teaching material is funded by Velbazhd Software LLC and it is partially supported by the Bulgarian Ministry of Education and Science (contract D01–205/23.11.2018) under the National Scientific Program ``Information and Communication Technologies for a Single Digital Market in Science, Education and Security (ICTinSES)'', approved by DCM \# 577/17.08.2018.}

\end{Form}

\end{document}
