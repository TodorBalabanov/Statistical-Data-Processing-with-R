\documentclass[a4paper,12pt]{minimal}

\usepackage[12pt]{moresize}
\usepackage{hyperref}
\usepackage[utf8x]{inputenc}
\usepackage[english,bulgarian]{babel}

\hypersetup{
  pdfauthor = {Тодор Балабанов},
  pdfkeywords = {},
  pdftitle = {Тест по R}
}

\begin{document}

\begin{center}
{\Huge Разширени графични възможности и вероятностни разпределения}
\end{center}

\begin{Form}[action=mailto:todor.balabanov@gmail.com,encoding=text,method=post]

\begin{tabular}{c c c} \\ 
	\TextField[name=first_name]{Име} & \TextField[name=second_name]{Презиме} & \TextField[name=third_name]{Фамилия} \\ \\
	\CheckBox[name=phd_student,width=3mm]{Докторант} & \CheckBox[name=bas_employee,width=3mm]{Служител в БАН} &  \CheckBox[name=other_student,width=3mm]{Друго} \\ \\
	\TextField[name=email]{Имейл} & \TextField[name=phone]{Телефон} & \TextField[name=mobile]{Мобилен}  \\ \\ 
\end{tabular}

1. Хистограмата е графика която?
\ChoiceMenu[radio,name=question_01]{\mbox{}}{\\
Онагледява възможността за ковариационна връзка;=a,\\
Онагледява възможността за корелационна връзка;=b,\\
Показва формата на разпръскване на различни стойности;=c,\\
Показва как се разпределят групи от стойности;=d}\\%

2. Диаграмата на разпръскване показва каква е формата на?
\ChoiceMenu[radio,name=question_02]{\mbox{}}{\\
Непрекъснатото вероятностно разпределение;=a,\\
Облака при една променлива;=b,\\
Облака от точки при две променливи;=c,\\%
Синусуидалното вероятностно разпределение;=d}\\

3. Графиката тип кутия се изчертава на база на?
\ChoiceMenu[radio,name=question_03]{\mbox{}}{\\
Квартилите;=a,\\%
Процентите;=b,\\
Средната стойност;=c,\\
Модата;=d}\\

4. Пакетът ggthemes дава възможности за?
\ChoiceMenu[radio,name=question_04]{\mbox{}}{\\
Семантично оформление на графиките;=a,\\
Тематично оформление на графиките;=b,\\%
Синтактично оформление на графиките;=c,\\
Апостериорно оформление на графиките;=d}\\

5. С пакета Rdice може да се?
\ChoiceMenu[radio,name=question_05]{\mbox{}}{\\
Извършва числено диференциране;=a,\\
Извършва числено интегриране;=b,\\
Генерират експерименти с различни зарове;=c,\\%
Извършва символно пресмятане на математически изрази;=d}\\

6. Функцията rnorm служи за генериране на случайни числа по?
\ChoiceMenu[radio,name=question_06]{\mbox{}}{\\
Равномерно разпределение;=a,\\
Биномно разпределение;=b,\\
Поасоново разпределение;=c,\\
Нормално разпределение;=d}\\%

7. Функцията rbinom служи за генериране на случайни числа по?
\ChoiceMenu[radio,name=question_07]{\mbox{}}{\\
Равномерно разпределение;=a,\\
Биномно разпределение;=b,\\%
Поасоново разпределение;=c,\\
Нормално разпределение;=d}\\

8. Функцията rpois служи за генериране на случайни числа по?
\ChoiceMenu[radio,name=question_08]{\mbox{}}{\\
Равномерно разпределение;=a,\\
Биномно разпределение;=b,\\
Поасоново разпределение;=c,\\%
Нормално разпределение;=d}\\

9. Функцията runif служи за генериране на случайни числа по?
\ChoiceMenu[radio,name=question_09]{\mbox{}}{\\
Равномерно разпределение;=a,\\%
Биномно разпределение;=b,\\
Поасоново разпределение;=c,\\
Нормално разпределение;=d}\\

\Submit{Изпрати}
\Reset{Изчисти}\\ \\

{\ssmall This teaching material is funded by Velbazhd Software LLC and it is partially supported by the Bulgarian Ministry of Education and Science (contract D01–205/23.11.2018) under the National Scientific Program ``Information and Communication Technologies for a Single Digital Market in Science, Education and Security (ICTinSES)'', approved by DCM \# 577/17.08.2018.}

\end{Form}

\end{document}
