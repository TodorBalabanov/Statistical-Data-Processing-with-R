\documentclass[a4paper,12pt]{minimal}

\usepackage[12pt]{moresize}
\usepackage{hyperref}
\usepackage[utf8x]{inputenc}
\usepackage[english,bulgarian]{babel}

\hypersetup{
  pdfauthor = {Тодор Балабанов},
  pdfkeywords = {},
  pdftitle = {Тест по R}
}

\begin{document}

\begin{center}
{\Huge Групиране и обхождане на данни}
\end{center}

\begin{Form}[action=mailto:todor.balabanov@gmail.com,encoding=text,method=post]

\begin{tabular}{c c c} \\ 
	\TextField[name=first_name]{Име} & \TextField[name=second_name]{Презиме} & \TextField[name=third_name]{Фамилия} \\ \\
	\CheckBox[name=phd_student,width=3mm]{Докторант} & \CheckBox[name=bas_employee,width=3mm]{Служител в БАН} &  \CheckBox[name=other_student,width=3mm]{Друго} \\ \\
	\TextField[name=email]{Имейл} & \TextField[name=phone]{Телефон} & \TextField[name=mobile]{Мобилен}  \\ \\ 
\end{tabular}

1. Фамилията функции apply служат за?
\ChoiceMenu[radio,name=question_01]{\mbox{}}{\\
Манипулиране на графики;=a,\\
Манипулиране на единична стойност;=b,\\
Групово манипулиране на данни;=c,\\%
Манипулиране на комуникацията по мрежа;=d}\\

2. Функцията aggregate служи за?
\ChoiceMenu[radio,name=question_02]{\mbox{}}{\\
Изтриване без критерии;=a,\\
Сливане без критерии;=b,\\
Разединяване по критерии;=c,\\
Групиране по критерии;=d}\\%

3. Пакетът plyr предлага функционалност за?
\ChoiceMenu[radio,name=question_03]{\mbox{}}{\\
Разделяне{,} манипулиране и обединение;=a,\\%
Сливане{,} изтриване и съхранение;=b,\\
Изчертаване{,} разпечатване и оформление;=c,\\
Криптиране{,} кодиране и компресиране;=d}\\

4. Пакетът data.table има за цел да?
\ChoiceMenu[radio,name=question_04]{\mbox{}}{\\
Намали възможностите на матриците;=a,\\
Разшири възможностите на data.frame;=b,\\%
Намали възможностите на списъците;=c,\\
Намали възможностите на масивите;=d}\\

5. Потоците в R?
\ChoiceMenu[radio,name=question_05]{\mbox{}}{\\
Налагат да се използват множество междинни променливи;=a,\\
Позволяват да се избегне използването на междинни променливи;=b,\\%
Подобряват комуникацията по мрежа;=c,\\
Работят без извикването на функции за пресмятане;=d}\\

6. Пакетът purrr служи за?
\ChoiceMenu[radio,name=question_06]{\mbox{}}{\\
Стандартизиран начин за обхождане на списъци и вектори;=a,\\%
Нестандартен начин за обхождане на дървовидни структури от данни;=b,\\
Нестандартен начин за обхождане на циклични графи;=c,\\
Стандартизиран начин за ускоряване на изчислителния процес;=d}\\

7. Функцията select служи за?
\ChoiceMenu[radio,name=question_07]{\mbox{}}{\\
Избор на комуникационен порт при TCP/IP комуникация;=a,\\
Избор на реда по който се изпълняват аритметичните операции;=b,\\
Подбор на колони от зададено множество данни;=c,\\
Подбор на редове от зададено множество данни;=d}\\%

8. Функцията filter служи за?
\ChoiceMenu[radio,name=question_08]{\mbox{}}{\\
Семантичен анализ на символни низове;=a,\\
Синтактичен анализ на символни низове;=b,\\
Избиране на колони от данните по зададен критерии;=c,\\
Избиране на редове от данните по зададен критерии;=d}\\%

9. Функцията map се използва за?
\ChoiceMenu[radio,name=question_09]{\mbox{}}{\\
Прилагане на определена функция върху първия елемент на списък;=a,\\
Прилагане на определена функция върху всеки елемент на списък;=b,\\%
Прилагане на определена функция върху последния елемент на списък;=c,\\
Прилагане на определена функция върху корена на балансирано бинарно дърво;=d}\\

\Submit{Изпрати}
\Reset{Изчисти}\\ \\

{\ssmall This teaching material is funded by Velbazhd Software LLC and it is partially supported by the Bulgarian Ministry of Education and Science (contract D01–205/23.11.2018) under the National Scientific Program ``Information and Communication Technologies for a Single Digital Market in Science, Education and Security (ICTinSES)'', approved by DCM \# 577/17.08.2018.}

\end{Form}

\end{document}
