\documentclass[a4paper,12pt]{minimal}

\usepackage[12pt]{moresize}
\usepackage{hyperref}
\usepackage[utf8x]{inputenc}
\usepackage[english,bulgarian]{babel}

\hypersetup{
  pdfauthor = {Тодор Балабанов},
  pdfkeywords = {},
  pdftitle = {Тест по R}
}

\begin{document}

\begin{center}
{\Huge Приближени пресмятания - подходи, методи, алгоритми}
\end{center}

\begin{Form}[action=mailto:todor.balabanov@gmail.com,encoding=text,method=post]

\begin{tabular}{c c c} \\ 
	\TextField[name=first_name]{Име} & \TextField[name=second_name]{Презиме} & \TextField[name=third_name]{Фамилия} \\ \\
	\CheckBox[name=phd_student,width=3mm]{Докторант} & \CheckBox[name=bas_employee,width=3mm]{Служител в БАН} &  \CheckBox[name=other_student,width=3mm]{Друго} \\ \\
	\TextField[name=email]{Имейл} & \TextField[name=phone]{Телефон} & \TextField[name=mobile]{Мобилен}  \\ \\ 
\end{tabular}

1. Монте-Карло методите?
\ChoiceMenu[radio,name=question_01]{\mbox{}}{\\
Приближени пресмятания със случайни числа;=a,\\%
Точни пресмятания по градиент;=b,\\
Точно числено диференциране;=c,\\
Точно числено интегриране;=d}\\

2. С пакета MonteCarlo мога да се извършват?
\ChoiceMenu[radio,name=question_02]{\mbox{}}{\\
Интерполации;=a,\\
Екстраполации;=b,\\
Симулации;=c,\\%
Манипулации;=d}\\

3. Генетичните алгоритми са?
\ChoiceMenu[radio,name=question_03]{\mbox{}}{\\
Глобална оптимизационна евристика;=a,\\%
Точен числен метод за оптимизация по градиент;=b,\\
Метод за точно числено диференциране;=c,\\
Метод за точно числено интегриране;=d}\\

4. С пакета genalg може да се търсят?
\ChoiceMenu[radio,name=question_04]{\mbox{}}{\\
Точни глобални минимуми в едномерно пространство;=a,\\
Точни глобални максимуми в едномерно пространство;=b,\\
Субоптимални решения в многомерно пространство;=c,\\%
Екстремални точки в едномерно пространство;=d}\\

5. Функцията rbga.bin работи основно с?
\ChoiceMenu[radio,name=question_05]{\mbox{}}{\\
Реални и комплексни числа;=a,\\
Булеви стойности;=b,\\
Символни низове;=c,\\
Бинарни вектори;=d}\\%

6. Параметърът evalFunc трябва да е?
\ChoiceMenu[radio,name=question_06]{\mbox{}}{\\
Естествено число;=a,\\
Реално число;=b,\\
Обект на функция за оценка;=c,\\%
Булева стойност;=d}\\

7. За кое от следните не са подходящи изкуствените неверонни мрежи?
\ChoiceMenu[radio,name=question_07]{\mbox{}}{\\
Класифициране;=a,\\
Трансценденциране;=b,\\%
Прогнозиране;=c,\\
Разпознаване на образи;=d}\\

8. С пакета neuralnet може да се ползват?
\ChoiceMenu[radio,name=question_08]{\mbox{}}{\\
Генетични алгоритми;=a,\\
Изкуствени невронни мрежи;=b,\\%
Рояк от частици и колония на мравките;=c,\\
Симулирано закаляване и еволюция на разликите;=d}\\

9. Функцията compute от пакета neuralnet се използва за?
\ChoiceMenu[radio,name=question_09]{\mbox{}}{\\
Проверка с тестовото множество;=a,\\%
За изчисление на теглата в мрежата;=b,\\
За изчисление на обратното разпространение;=c,\\
За определяне на структурата в мрежата;=d}\\

\Submit{Изпрати}
\Reset{Изчисти}\\ \\

{\ssmall This teaching material is funded by Velbazhd Software LLC and it is partially supported by the Bulgarian Ministry of Education and Science (contract D01–205/23.11.2018) under the National Scientific Program ``Information and Communication Technologies for a Single Digital Market in Science, Education and Security (ICTinSES)'', approved by DCM \# 577/17.08.2018.}

\end{Form}

\end{document}
