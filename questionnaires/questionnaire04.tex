\documentclass[a4paper,12pt]{minimal}

\usepackage[12pt]{moresize}
\usepackage{hyperref}
\usepackage[utf8x]{inputenc}
\usepackage[english,bulgarian]{babel}

\hypersetup{
  pdfauthor = {Тодор Балабанов},
  pdfkeywords = {},
  pdftitle = {Тест по R}
}

\begin{document}

\begin{center}
{\Huge Въвеждане на данни и извеждане на графики}
\end{center}

\begin{Form}[action=mailto:todor.balabanov@gmail.com,encoding=text,method=post]

\begin{tabular}{c c c} \\ 
	\TextField[name=first_name]{Име} & \TextField[name=second_name]{Презиме} & \TextField[name=third_name]{Фамилия} \\ \\
	\CheckBox[name=phd_student,width=3mm]{Докторант} & \CheckBox[name=bas_employee,width=3mm]{Служител в БАН} &  \CheckBox[name=other_student,width=3mm]{Друго} \\ \\
	\TextField[name=email]{Имейл} & \TextField[name=phone]{Телефон} & \TextField[name=mobile]{Мобилен}  \\ \\ 
\end{tabular}

1. Сценарии за изчисления върху данни в R се съхранява във?
\ChoiceMenu[radio,name=question_01]{\mbox{}}{\\
Обикновен текстов файл с .r разширение;=a,\\%
В Microsoft Excel файл;=b,\\
В LibreOffice Calc файл;=c,\\
В бинарен файл с машинни инструкции;=d}\\

2. Универсален файлов формат за въвеждане на данни в R e?
\ChoiceMenu[radio,name=question_02]{\mbox{}}{\\
DOCX;=a,\\
CVS;=b,\\%
PPTX;=c,\\
EXE;=d}\\

3. CSV файл с данни в R се чете с функцията?
\ChoiceMenu[radio,name=question_03]{\mbox{}}{\\
read.date;=a,\\
read.table;=b,\\%
read.binary;=c,\\
read.text;=d}\\

4. Пакетът RODBC служи за?
\ChoiceMenu[radio,name=question_04]{\mbox{}}{\\
Манипулация на растерни изображения;=a,\\
Манипулация на векторни изображения;=b,\\
Осъществяване на комуникация по мрежа;=c,\\
Работа с релационни бази данни;=d}\\%

5. Rdata файловете служат за?
\ChoiceMenu[radio,name=question_05]{\mbox{}}{\\
Сериализация на работна променлива по мрежата;=a,\\
Бинарно съхраняване на работна променлива;=b,\\%
Съхраняване на растерни изображения;=c,\\
Съхраняване на векторни изображения;=d}\\

6. Някои пакети на R?
\ChoiceMenu[radio,name=question_06]{\mbox{}}{\\
Няма възможност да бъдат заредени;=a,\\
Не са предназначени за използване в R;=b,\\
Са силно интерферентни;=c,\\
Имат демонстративни множества от данни;=d}\\%

7. Форматът JSON представлява?
\ChoiceMenu[radio,name=question_07]{\mbox{}}{\\
SQL описание на релации;=a,\\
JavaScript описание на обекти;=b,\\%
Описание графични примитиви;=c,\\
Последователност от GCode инструкции;=d}\\

8. Пакетът ggplot2 служи за?
\ChoiceMenu[radio,name=question_08]{\mbox{}}{\\
Визуализация на данни;=a,\\%
Сериализация на данни;=b,\\
Интерполация на данни;=c,\\
Екстраполация на данни;=d}\\

9. Кое от изброените не е инструмент за визуализация?
\ChoiceMenu[radio,name=question_09]{\mbox{}}{\\
Хистограма;=a,\\
Диаграма на разпръскване;=b,\\
Референциален интегритет;=c,\\%
Графика тип кутии;=d}\\

\Submit{Изпрати}
\Reset{Изчисти}

\end{Form}

\end{document}
