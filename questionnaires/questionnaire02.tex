\documentclass[a4paper,12pt]{minimal}

\usepackage[12pt]{moresize}
\usepackage{hyperref}
\usepackage[utf8x]{inputenc}
\usepackage[english,bulgarian]{babel}

\hypersetup{
  pdfauthor = {Тодор Балабанов},
  pdfkeywords = {},
  pdftitle = {Тест по R}
}

\begin{document}

\begin{center}
{\Huge Пакетна организация, променливи, основни математически операции в R и типове данни}
\end{center}

\begin{Form}[action=mailto:todor.balabanov@gmail.com,encoding=text,method=post]

\begin{tabular}{c c c} \\ 
	\TextField[name=first_name]{Име} & \TextField[name=second_name]{Презиме} & \TextField[name=third_name]{Фамилия} \\ \\
	\CheckBox[name=phd_student,width=3mm]{Докторант} & \CheckBox[name=bas_employee,width=3mm]{Служител в БАН} &  \CheckBox[name=other_student,width=3mm]{Друго} \\ \\
	\TextField[name=email]{Имейл} & \TextField[name=phone]{Телефон} & \TextField[name=mobile]{Мобилен}  \\ \\ 
\end{tabular}

1. Функционалността в програмния продукт R е организирана в?
\ChoiceMenu[radio,name=question_01]{\mbox{}}{\\
Макроси;=a,\\
Пакети;=b,\\%
Ламбда изрази;=c,\\
SQL синтаксис;=d}\\

2. За да се ползва определена функционалност от пакет трябва пакетът да се включи с командите?
\ChoiceMenu[radio,name=question_02]{\mbox{}}{\\
include или embed;=a,\\
include или require;=b,\\
library или require;=c,\\%
library или include;=d}\\

3. Кое от изброените не е характерно за операциите за пресмятане в R?
\ChoiceMenu[radio,name=question_03]{\mbox{}}{\\
Трансфузионност;=a,\\%
Асоциативност;=b,\\
Приоритет;=c,\\
Кардиналност;=d}\\

4. В програмния език на софтуерния продукт R?
\ChoiceMenu[radio,name=question_04]{\mbox{}}{\\
Не се използват променливи;=a,\\
Не съществуват типове на променливите;=b,\\
Типовете на променливите са явни;=c,\\
Типовете на променливите са неявни;=d}\\%

5. В програмния език на софтуерния продукт R?
\ChoiceMenu[radio,name=question_05]{\mbox{}}{\\
Променливите се съхраняват единствено на твърдия диск;=a,\\
Променливите се съхраняват единствено в стека;=b,\\
Няма нужда от съхраняването на променливи;=c,\\
Променливите се помещават в обща памет;=d}\\%

6. С функцията class се проверява?
\ChoiceMenu[radio,name=question_06]{\mbox{}}{\\
Неявния тип на променливата;=a,\\%
Диапазона на действие на променливата;=b,\\
Обсегът на видимост на променливата;=c,\\
Разположението в паметта на променливата;=d}\\

7. Логическите стойности се използват за?
\ChoiceMenu[radio,name=question_07]{\mbox{}}{\\
Числено пресмятане на интеграли;=a,\\
Аритметични пресмятания с реални числа;=b,\\
Изрази от булевата алгебра;=c,\\%
Обработка на текстова информация;=d}\\

8. Типът factor в R представлява?
\ChoiceMenu[radio,name=question_08]{\mbox{}}{\\
Специален тип за извършване на умножение;=a,\\
Специален тип за извършване на матрични пресмятания;=b,\\
Индексирани символни низове;=c,\\%
Специален тип за извършване на побитови операции;=d}\\

9. Типът POSIXct се използва за представянето на?
\ChoiceMenu[radio,name=question_09]{\mbox{}}{\\
Само на час;=a,\\
Само на дата;=b,\\
Дата и час;=c,\\%
Не се ползва за дата и час;=d}\\

\Submit{Изпрати}
\Reset{Изчисти}\\ \\

{\ssmall This teaching material is funded by Velbazhd Software LLC and it is partially supported by the Bulgarian Ministry of Education and Science (contract D01–205/23.11.2018) under the National Scientific Program ``Information and Communication Technologies for a Single Digital Market in Science, Education and Security (ICTinSES)'', approved by DCM \# 577/17.08.2018.}

\end{Form}

\end{document}
