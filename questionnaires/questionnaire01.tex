\documentclass[a4paper,12pt]{minimal}

\usepackage[12pt]{moresize}
\usepackage{hyperref}
\usepackage[utf8x]{inputenc}
\usepackage[english,bulgarian]{babel}

\hypersetup{
  pdfauthor = {Тодор Балабанов},
  pdfkeywords = {},
  pdftitle = {Тест по R}
}

\begin{document}

\begin{center}
{\Huge Инсталация и стартиране}
\end{center}

\begin{Form}[action=mailto:todor.balabanov@gmail.com,encoding=text,method=post]

\begin{tabular}{c c c} \\ 
	\TextField[name=first_name]{Име} & \TextField[name=second_name]{Презиме} & \TextField[name=third_name]{Фамилия} \\ \\
	\CheckBox[name=phd_student,width=3mm]{Докторант} & \CheckBox[name=bas_employee,width=3mm]{Служител в БАН} &  \CheckBox[name=other_student,width=3mm]{Друго} \\ \\
	\TextField[name=email]{Имейл} & \TextField[name=phone]{Телефон} & \TextField[name=mobile]{Мобилен}  \\ \\ 
\end{tabular}

1. Програмният продукт R е?
\ChoiceMenu[radio,name=question_01]{\mbox{}}{\\
Със затворен код;=a,\\
С отворен код;=b,\\%
Без програмен код;=c,\\
С минималистичен програмен код;=d}\\

2. Програмният продукт R е?
\ChoiceMenu[radio,name=question_02]{\mbox{}}{\\
Лошо организиран;=a,\\
Добре организиран;=b,\\
Организиран на модулен принцип;=c,\\%
Организиран на монолитен принцип;=d}\\

3. Програмният продукт R е не може да се използва на?
\ChoiceMenu[radio,name=question_03]{\mbox{}}{\\
Обикновен калкулатор;=a,\\%
Операционна система Windows;=b,\\
Операционна система Linux;=c,\\
Операционна система MacOS;=d}\\

4. Програмният продукт R се разпространява във?
\ChoiceMenu[radio,name=question_04]{\mbox{}}{\\
Версия с 20 и 40 бита;=a,\\
Версия с 16 и 32 бита;=b,\\
Версия с 16 и 64 бита;=c,\\
Версия с 32 и 64 бита;=d}\\%

5. Програмният продукт R се разпространява под?
\ChoiceMenu[radio,name=question_05]{\mbox{}}{\\
GPL2 лиценз;=a,\\%
MIT лиценз;=b,\\
BSD2 лиценз;=c,\\
CDDL лиценз;=d}\\

6. Програмният продукт R се използва?
\ChoiceMenu[radio,name=question_06]{\mbox{}}{\\
В уеб браузър;=a,\\
С индустриален контролер;=b,\\
В режим на команден интерпретатор;=c,\\%
С помощта на аудио интерфейс ;=d}\\

7. Програмният продукт R е базиран на?
\ChoiceMenu[radio,name=question_07]{\mbox{}}{\\
Декларативен програмен език;=a,\\
Интерпретивен програмен език;=b,\\%
Компилация от полуинтерпретивен тип;=c,\\
Компилатор и линкер;=d}\\

8. Програмният продукт R?
\ChoiceMenu[radio,name=question_08]{\mbox{}}{\\
Използва библиотеки;=a,\\%
Не използва библиотеки;=b,\\
Репродуцира библиотеки;=c,\\
Интерполира библиотеки;=d}\\

9. Програмният продукт R е предназначен за?
\ChoiceMenu[radio,name=question_09]{\mbox{}}{\\
Манипулативна обработка на данни;=a,\\
Интуитивна обработка на данни;=b,\\
Репродуктивна обработка на данни;=c,\\
Статистическа обработка на данни;=d}\\%

\Submit{Изпрати}
\Reset{Изчисти} \\ \\

{\ssmall This teaching material is funded by Velbazhd Software LLC and it is partially supported by the Bulgarian Ministry of Education and Science (contract D01–205/23.11.2018) under the National Scientific Program ``Information and Communication Technologies for a Single Digital Market in Science, Education and Security (ICTinSES)'', approved by DCM \# 577/17.08.2018.}

\end{Form}

\end{document}
