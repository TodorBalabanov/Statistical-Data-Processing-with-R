\newpage
\addcontentsline{toc}{chapter}{Предговор}
\chapter*{Предговор}
\pagenumbering{arabic}
\setcounter{page}{1}
\pagestyle{fancyplain}

Това учебно помагало е предвидено за студенти и докторанти, които в своите магистърски или докторски тези се сблъскват с потребността да извършат определени експерименти, а след това да обработят статистически получените резултати. 

В съвременния живот нуждата от обработка на информация единствено нараства. В множество ситуации от ежедневния ни живот се налага да бъдат вземани решения. От своя страна, всяко решение е толкова по-успешно, колкото по-информирано е взето то. Статистическата обработка на събраната информация е една от основните за вземането на информирани решения. В областта на статистическата обработка съществуват множество софтуерни решения, като се започне от по-достъпните за хора без опит, като Microsoft Excel и се стигне до професионалните пакети, като SPSS, Matlab и Mathematica. 

Това учебно помагало представя програмния продукт R, който първоначално се разработва от Robert Gentleman и Ross Ihaka в University of Auckland през 1993 година. R е замислен като алтернатива на програмния продукт S, създаден от John Chambers, служител в Bell Labs. Първоначалният замисъл за R е инструмент, който да бъде използван в интерактивен режим, през командния ред. В последствие тази идея прераства в самостоятелен програмен език. Основното предназначение на R е обработка на данни, което включва въвеждане, пресмятане, визуализация на графики и отчети. 

Езикът получава значително по голяма популярност след 2000 година, като излиза от рамките на академичните среди и навлиза в финансовите среди, маркетинга, фармацията, социологията, психологията и в много други области. Най-често потребителите на R са хора с опит в програмни езици, като C/C++, Java, C\# или пък преди това са използвали други статистически пакети, като SAS, SPSS и дори Excel. Тези потребители дават значителен тласък в развитието на пакета R, добавяйки множество софтуерни приставки (add-ons). 

Въпреки че в някои случаи R се оказва стряскащ и дори смущаващ, особено за начинаещите потребители, с времето и с процеса по навлизане в материята овладяването му се улеснява и ускорява. Това учебно помагало представя информацията по един достъпен и олекотен начин за възприемане. Изложени са предимно най-важните аспекти от използването на пакета R, което от своя страна дава стабилна основа за бъдещо самостоятелно развитие на читателя. Материалът е съобразен със съдържанието на курса „Анализ на данни с R“, провеждан в „Център за обучение“ към „Българска академия на науките. Учебното помагало е организирано в следните глави.

Глава 1 - \nameref{chapter01}: Представя процеса по инсталиране и стартиране на програмния пакет.