\newpage
\chapter{Оформление на резултатите за печатно и електронно представяне}
\label{chapter11}
\thispagestyle{empty}

Финалното оформление\index{визуално оформление} на получените от анализа резултати е не по-малко важно от самото им пресмятане. Спрямо аудиторията пред която резултатите ще бъдат представяни оформлението им може да бъде в различни варианти, като писмен доклад, уеб страница\index{уеб страници} или презентация\index{мултимедийни презентации} със слайдове. За тези нужди програмният продукт $R$ предлага група от пакети, спомагащи за постигането на максимална експресивност в представянето. Пакетът $knitr$ спомага оформянето на отчети и доклади. Пакетът дава възможност за работа с тагиращите езици $LaTeX$ и $Markdown$, като резултата от компилацията може да бъдат PDF документи, HTML страници, презентации и дори Microsoft Word документи. 

\section{Работа с LaTeX}

\section{Работа с RMarkdown}

\section*{Заключение}


