\newpage
\chapter{Оформление на резултатите за печатно и електронно представяне}
\label{chapter11}
\thispagestyle{empty}

Финалното оформление\index{визуално оформление} на получените от анализа резултати е не по-малко важно от самото им пресмятане. Спрямо аудиторията пред която резултатите ще бъдат представяни оформлението им може да бъде в различни варианти, като писмен доклад, уеб страница\index{уеб страници} или презентация\index{мултимедийни презентации} със слайдове. За тези нужди програмният продукт $R$ предлага група от пакети, спомагащи за постигането на максимална експресивност в представянето. Пакетът $knitr$ спомага оформянето на отчети и доклади. Пакетът дава възможност за работа с тагиращите езици $LaTeX$ и $Markdown$, като резултата от компилацията може да бъдат PDF документи, HTML страници, презентации и дори Microsoft Word документи. 

\section{Работа с LaTeX}

LaTeX\index{LaTeX} е тагиращ език с широко приложение в писането на научни статии, тези, книги, постери и презентации. За да се използват възможностите на LaTeX е необходимо инсталирането на допълнителен софтуер за съответната операционна система. За трите най-популярни операционни системи LaTeX се поддържа от различни пакети, както следва: Windows - MiKTex, MacOS - MacTex и Linux - TeX Live.

За работа с LaTeX се създават обикновено текстови файлове, чието разширение е „.tex“ и може да се създават с всеки съвременен текстов редактор. Tex документите са йерахични документи с ясно дефинирана структура. На първия ред се записва инструкция за вида на документа с командата \textbackslash documentclass\{...\}. Най-популярните видове документ са $report$, $beamer$, $memoir$, $letter$ и други. След типа на документа следва служебна секция за зареждане на нужни за компилацията пакети и/или индекси. За включването на изображения е необходимо използването на пакета $graphicx$. В същата секция се определя авторът (\textbackslash author), заглавието (\textbackslash title) и датата (\textbackslash date) на документа. Същинското съдържание на документа се разполага между инструкциите \textbackslash begin\{document\} и \textbackslash end\{document\}.

Изложението на документа може да бъде разделено на отделни секции с инструкцията \textbackslash section\{Название на секция\}. Всичко написано след тази инструкция става част на съответната секция докато не бъде достигната следващата инструкция на нова секция. Номерирането на секциите и подсекциите се извършва автоматично от текстовия процесор на LaTeX. Когато са поставени етикети с инструкцията \textbackslash label\{етикет\}, те могат да бъдат позовавани в други части на документа с инструкцията \textbackslash ref\{етикет\}. Съдържанието на документа се генерира автоматично с помощта на инструкцията \textbackslash tableofcontents.

Изброените инструкции са напълно достатъчни за създаване на базови документи, но далеч не покриват пълните възможности на LaTeX. Тъй като компилатора използван в текстовия процесор е еднопасов, то за да се направи правилно индексиране на препратките и таблицата за съдържанието често се налага компилацията да бъде стартирана два пъти последователно. 

\section{Работа с RMarkdown}

\section*{Заключение}


