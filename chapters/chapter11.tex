\newpage
\chapter{Оформление на резултатите за печатно и електронно представяне}
\label{chapter11}
\thispagestyle{empty}

Финалното оформление\index{визуално оформление} на получените от анализа резултати е не по-малко важно от самото им пресмятане. Спрямо аудиторията пред която резултатите ще бъдат представяни оформлението им може да бъде в различни варианти, като писмен доклад, уеб страница\index{уеб страници} или презентация\index{мултимедийни презентации} със слайдове. За тези нужди програмният продукт $R$ предлага група от пакети, спомагащи за постигането на максимална експресивност в представянето. Пакетът $knitr$ спомага оформянето на отчети и доклади. Пакетът дава възможност за работа с тагиращите езици $LaTeX$ и $Markdown$, като резултата от компилацията може да бъдат $PDF$ документи, $HTML$ страници, презентации и дори $Microsoft Word$ документи. 

\section{Работа с LaTeX}

$LaTeX$\index{LaTeX} е тагиращ език с широко приложение в писането на научни статии, тези, книги, постери и презентации. За да се използват възможностите на $LaTeX$ е необходимо инсталирането на допълнителен софтуер за съответната операционна система. За трите най-популярни операционни системи $LaTeX$ се поддържа от различни пакети, както следва: $Windows$ - $MiKTex$, $MacOS$ - $MacTex$ и $Linux$ - $TeX Live$.

За работа с $LaTeX$ се създават обикновено текстови файлове, чието разширение е „.tex“ и може да се създават с всеки съвременен текстов редактор. Tex документите са йерахични документи с ясно дефинирана структура. На първия ред се записва инструкция за вида на документа с командата \textbackslash documentclass\{...\}. Най-популярните видове документ са $report$, $beamer$, $memoir$, $letter$ и други. След типа на документа следва служебна секция за зареждане на нужни за компилацията пакети и/или индекси. За включването на изображения е необходимо използването на пакета $graphicx$. В същата секция се определя авторът (\textbackslash author), заглавието (\textbackslash title) и датата (\textbackslash date) на документа. Същинското съдържание на документа се разполага между инструкциите \textbackslash begin\{document\} и \textbackslash end\{document\}.

Изложението на документа може да бъде разделено на отделни секции с инструкцията \textbackslash section\{Название на секция\}. Всичко написано след тази инструкция става част на съответната секция докато не бъде достигната следващата инструкция на нова секция. Номерирането на секциите и подсекциите се извършва автоматично от текстовия процесор на $LaTeX$. Когато са поставени етикети с инструкцията \textbackslash label\{етикет\}, те могат да бъдат позовавани в други части на документа с инструкцията \textbackslash ref\{етикет\}. Съдържанието на документа се генерира автоматично с помощта на инструкцията \textbackslash tableofcontents.

Изброените инструкции са напълно достатъчни за създаване на базови документи, но далеч не покриват пълните възможности на $LaTeX$. Тъй като компилатора използван в текстовия процесор е еднопасов, то за да се направи правилно индексиране на препратките и таблицата за съдържанието често се налага компилацията да бъде стартирана два пъти последователно. 

Създаването на $LaTeX$ документи с интегриране на $R$ инструкции в него става чрез изготвянето на стандартен $LaTeX$ документ в който се добавят фрагменти (chunks) на $R$ програмен код. Тези фрагменти се предхождат от инструкция за начало на фрагмента по зададен шаблон (Листинг \ref{listing0179}), а края на фрагмента се обозначава със символа маймунско $а$.

\begin{lstlisting}[caption=Инструкция за R фрагмент в LaTeX документ, label=listing0179]
<<label-value,option1=value1,option2=value2>>=
@
\end{lstlisting}

Текстовият документ се запазва с разширение „.Rnw“. Трансформацията на комбинирания код ($LaTeX$ и $R$) се транслира до $Tex$ с командата $Sweave$ в командния интерпретатор на $R$ (Листинг \ref{listing0180}). Важно е $Rnw$ файлът да се намира в същата директория в която се изпълнява командата за транслиране. 

\begin{lstlisting}[caption=Транслиране от Rnw до Tex, label=listing0180]
library( knitr )

setwd( "~/Desktop" )

Sweave( "./example0002.Rnw" )
\end{lstlisting}

В резултат на транслацията, в съответната директория се генерира Tex файл, който от своя страна се подава на транслатор, изпълняван в конзолата на операционната система, $Tex$ към $PDF$ (Листинг \ref{listing0181}).

\begin{lstlisting}[caption=Транслиране от Tex до PDF, label=listing0181]
pdflatex ./example0002.tex
\end{lstlisting}

За илюстриране на възможностите при генерирането на $PDF$ документи са предложени примерните файлове достъпни на следните електронни адреси:

\begin{lstlisting}[caption=Адрес на примерени $LaTeX$ документи, label=listing0182]
https://raw.githubusercontent.com/TodorBalabanov/Statistical-Data-Processing-with-R/master/code/Sweave.sty

https://raw.githubusercontent.com/TodorBalabanov/Statistical-Data-Processing-with-R/master/code/example0002.Rnw

https://raw.githubusercontent.com/TodorBalabanov/Statistical-Data-Processing-with-R/master/code/example0003.Rnw
\end{lstlisting}

За да бъде успешно транслирането от $Tex$ към $PDF$ е нужно в директорията с $Tex$ файла да се помести и Sweave.sty файлът, който съдържа дефиниции необходими за извършването на процеса по транслация. 

\begin{lstlisting}[caption=Линейна регресия на разходи спрямо спестявания, label=listing0183]
library( knitr )

setwd( "~/Desktop" )

Sweave( "./example0003.Rnw" )

system("pdflatex ./example0003.tex", intern=TRUE)
\end{lstlisting}

При избора на имена за етикетите в $R$ фрагментите трябва да се избягва използването на специални символи, като празни интервал и точка. Добър вариант е използването на латинските букви и символа за тире. Когато се използва флага $echo$ със стойност $FALSE$ в $Tex$ файла не се включва $R$ програмния код довел до визуализацията на съответните резултати. 

За визуализация на изображения (Листинг \ref{listing0183}) най-удачно е междинната графика да бъде генериране в отделен графичен файл (в случая Fig01.png), който в последствие да бъде добавен като графичен компонент в Rnw документа.

\section{Работа с RMarkdown}

$RMarkdown$\index{RMarkdown} става все по-използван през годините, тъй като е значително по-опростен от $LaTeX$ и това създава допълнителен комфорт при оформянето на информацията. $RMarkdown$ дава възможности за генериране на множество различни файлови формати, а също така позволява разширяване на шаблонен принцип (templates). Процесът по създаване на документи с $R$ и $RMarkdown$ е сходен с процеса ползван за $LaTeX$. В чист текст се изписва $RMarkdown$ структурата, а на подходящи места се вмъкват фрагменти $R$ код (chunks). Текстовите файлове се съхраняват с разширение „.Rmd“.

R пакетът $knitr$ се използва за изчислението на $R$ кода в $RMarkdown$, а $pandoc$ софтуерния пакет за транслирането в различните файлови формати. Транслирането на текстовия файл става с $R$ функцията $render$, която използва софтуерния пакет $pandoc$. Група функции от пакета $rmarkdown$ - $html\_document$, $pdf\_document$, $word\_document$ и $ioslides\_presentation$ се използват за най-популярните файлови формати. Допълнително пакетите $rticles$, $tufte$ и $resumer$ дават още възможности.

Всеки $RMarkdown$ документ започва със заглавна част в YAML формат, която дава подробна информация за документа. Заглавната част започва с три последователни тирета и завършва с три последователни тирета, на отделен ред. Всеки ред в заглавната част е ключ-стойност двойка, която определя параметри, като – заглавие, автор, дата и целеви формат на документа. Типовете възможни документи за генериране са описани в пакетите, които отговарят за тях и в общия случай съвпадат с названието на функцията, която извършва транслацията. За пакети различни от $rmarkdown$, типът на документа се предхожда от името на пакета, който го поддържа. Ако типът на документа изисква допълнителни опции, то те се задават като под етикети с поне два спейса подравняване. 

$Markdown$ тагиращият език е създаден с цел максимално опростяване на синтаксиса. Не разполага с възможностите и гъвкавостта на $LaTeX$ или $HTML$, но пък значително ускорява писането и оформлението. Поради максималната си опростеност $Markdown$ е много бърз за научаване, което винаги е положително. 

В $Markdown$ нов ред се получава с оставянето на прасен ред между текстовете и с два или повече празни интервала в края на реда. За наклонен текст се поставя подчертаваща черта от двете страни на текста. За удебелен текст се поставят по две подчертаващи черти от двете страни на текста. За наклонен и удебелен шрифт се поставят по три подчертаващи черти от двете страни на текста. Символът по-голямо в началото на всеки ред служи за блоков цитат. Неномериран списък се създава с тире в началото на всеки нов ред или със звезда. Номериран списък се създава с цифра или буква, последвани с точка в началото на всеки ред. Списъците могат да бъдат влагани един в друг. Заглавията се обозначават със символа диез в началото на реда. Броят на диезите определя нивото на заглавие, като са възможни стойности от едно до шест. При генерирането на $PDF$ файл заглавията от ранг едно са за секции, а от ранг две за подсекции. Хипервръзки се създават с текст в квадратни скоби, последвани от URL адрес обграден в кръгли скоби. Включването на изображения става с алтернативен текст в квадратни скоби и адрес до изображението в кръгли скоби, но пред квадратните скоби се изписва удивителен знак. Изписването на уравненията става с ограждане на уравнението от двете страни с два знака за долар. За целта двойните долар знаци е важно да са на отделни редове, а като синтаксис за формулите се използва $LaTeX$. 

\subsection{Статични документи}

Когато финалните документи не съдържат възможности за манипулирането на компонентите в тях, то те съдържат статична информация. Най-често това са $PDF$ файлове и $HTML$ документи без JavaScript функционалност.

\begin{lstlisting}[caption=Адрес на примерни RMarkdown документ, label=listing0184]
https://raw.githubusercontent.com/TodorBalabanov/Statistical-Data-Processing-with-R/master/code/example0004.Rmd

https://raw.githubusercontent.com/TodorBalabanov/Statistical-Data-Processing-with-R/master/code/example0005.Rmd

https://raw.githubusercontent.com/TodorBalabanov/Statistical-Data-Processing-with-R/master/code/example0007.Rmd
\end{lstlisting}

Примерен $RMarkdown$ е представен на адресът от Листинг \ref{listing0184}, а транслирането до $HTML$ документ става с командите представени в Листинг \ref{listing0185}. За целта $Rmd$ файлът трябва да се намира в Desktop директорията на локалния компютър. 

\begin{lstlisting}[caption=Транслиране от RMarkdown в HTML и PDF, label=listing0185]
library( knitr )
library( ggplot2 )
library( rmarkdown )

setwd( "~/Desktop" )

render( "./example0004.Rmd" )
render( "./example0005.Rmd" )
render( "./example0007.Rmd" )
\end{lstlisting}

$Markdown$ документи, които съдържат фрагменти с $R$ програмен код се наричат $RMarkdown$. Поведението на тези фрагменти е като на $knitr$, но се етикетират по различен начин и имат малко повече възможности. Фрагментът започва с ти апострофа, отваряща фигурна скоба, буквата $r$, етикет на фрагмента, опции разделени със запетая и затваряща фигурна скоба. Фрагмента се затваря с три последователни апострофа. Всичко между отварянето и затварянето на фрагмента се третира като $R$ програмен код. 

Възможност за създаване на презентации с $R$ има през $LaTeX$ и шаблонът $Beamer$, където всяка страница се явява отделен слайд. Сложността на $LaTeX$ синтаксиса може да бъде избегната чрез използването на $RMarkdown$, който да се транслира до $HTML5$ презентация (example0007.Rmd). Заглавието на всеки слайд се маркира с два диеза, което е еквивалентно на заглавие от втори ред. Графично оформление на слайда може да се укаже с class, идентификатор на слайда и стандартни $CSS$ инструкции от вида: \{.vcenter .flexbox \#SlideID\}

\subsection{Динамични документи}

С помощта на пакета $htmlwidgets$ към финалния документ може да се добави JavaScript функционалност, така че отделните компоненти да дадат определена степенен на интерактивност. 

Въпреки че графичното представяне на информацията е значително по-информативно, в някои случай е нужно резултатите да бъдат представени в таблична форма. С помощта на функцията $kable$ от пакета $knitr$ могат да се създават таблици със статично съдържание (Листинг \ref{listing0187}).

\begin{lstlisting}[caption=Адрес на примерни интерактивни документ, label=listing0186]
https://raw.githubusercontent.com/TodorBalabanov/Statistical-Data-Processing-with-R/master/code/example0006.Rmd
\end{lstlisting}

За да се добави интерактивност в табличното представяне\index{таблично представяне} е предложен пакетът $DT$. С помощта на функцията $datatable$ се генерира таблица, даваща множество допълнителни възможности. Имената на колоните могат да се премахнат с опцията $rownames$. С помощта на опцията $filter$ се дава възможност за филтриране на редовете. С добавяне на разширението $Scroller$ се дава възможност за вертикално прелистване на редовете. Опцията $scrollX$ дава възможност за хоризонтално прелистване на колоните. Опцията $dom$ указва елементът за визуализация да е самата таблица (t от буквеното съчетание tiS). Допълнителна информация за таблицата се задава с буквата i в съчетанието tiS. Възможностите за прелистване се задават с буквата S в съчетанието tiS. Част от параметрите се задават на самата функция $datatable$, а друга част като списък от опции, демонстрирани в примера example0006.Rmd (Листинг \ref{listing0186}).

\begin{lstlisting}[caption=Създаване на интерактивни документи, label=listing0187]
library( DT )
library( dplyr )
library( knitr )
library( ggplot2 )
library( rmarkdown )
library( d3heatmap )

setwd( "~/Desktop" )

render( "./example0006.Rmd" )
\end{lstlisting}

Използването на топлинни карти\index{топлинна карта} е значително по-информативно, когато се пресмятат корелационни матрици\index{корелационна матрица}. С помощта на пакета $d3heatmap$ могат да се създават интерактивни топлинни карти. За нуждите на примера се демонстрира икономическото множество данни. С параметъра $colors$ на функцията $d3heatmap$ може да се контролира цветната палитра за изобразяване на стойностите (Листинг \ref{listing0187}). Примерът е поместен в работен файл с адрес посочен в Листинг \ref{listing0186}.

\section*{Заключение}

Благодарение на възможностите за интегриране на $R$ в $LaTeX$ и $RМarkdown$ документи може да се постигне бързо и стилно представяне на статистическите резултати получени сред извършване на съответни пресмятания. Изложените примери представят само най-основното, но пакетите разработени за $R$ дават значително повече възможности. С малко повече желание и малко повече старание всеки потребител на пакета $R$ може да постигне висока степен на експресивност в начина по който представя резултатите от своята работа.

