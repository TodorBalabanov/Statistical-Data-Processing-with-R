\newpage
\chapter{Оформление на резултатите за печатно и електронно представяне}
\label{chapter11}
\thispagestyle{empty}

Финалното оформление\index{визуално оформление} на получените от анализа резултати е не по-малко важно от самото им пресмятане. Спрямо аудиторията пред която резултатите ще бъдат представяни оформлението им може да бъде в различни варианти, като писмен доклад, уеб страница\index{уеб страници} или презентация\index{мултимедийни презентации} със слайдове. За тези нужди програмният продукт $R$ предлага група от пакети, спомагащи за постигането на максимална експресивност в представянето. Пакетът $knitr$ спомага оформянето на отчети и доклади. Пакетът дава възможност за работа с тагиращите езици $LaTeX$ и $Markdown$, като резултата от компилацията може да бъдат PDF документи, HTML страници, презентации и дори Microsoft Word документи. 

\section{Работа с LaTeX}

LaTeX\index{LaTeX} е тагиращ език с широко приложение в писането на научни статии, тези, книги, постери и презентации. За да се използват възможностите на LaTeX е необходимо инсталирането на допълнителен софтуер за съответната операционна система. За трите най-популярни операционни системи LaTeX се поддържа от различни пакети, както следва: Windows - MiKTex, MacOS - MacTex и Linux - TeX Live.

За работа с LaTeX се създават обикновено текстови файлове, чието разширение е „.tex“ и може да се създават с всеки съвременен текстов редактор. Tex документите са йерахични документи с ясно дефинирана структура. На първия ред се записва инструкция за вида на документа с командата \textbackslash documentclass\{...\}. Най-популярните видове документ са $report$, $beamer$, $memoir$, $letter$ и други. След типа на документа следва служебна секция за зареждане на нужни за компилацията пакети и/или индекси. За включването на изображения е необходимо използването на пакета $graphicx$. В същата секция се определя авторът (\textbackslash author), заглавието (\textbackslash title) и датата (\textbackslash date) на документа. Същинското съдържание на документа се разполага между инструкциите \textbackslash begin\{document\} и \textbackslash end\{document\}.

Изложението на документа може да бъде разделено на отделни секции с инструкцията \textbackslash section\{Название на секция\}. Всичко написано след тази инструкция става част на съответната секция докато не бъде достигната следващата инструкция на нова секция. Номерирането на секциите и подсекциите се извършва автоматично от текстовия процесор на LaTeX. Когато са поставени етикети с инструкцията \textbackslash label\{етикет\}, те могат да бъдат позовавани в други части на документа с инструкцията \textbackslash ref\{етикет\}. Съдържанието на документа се генерира автоматично с помощта на инструкцията \textbackslash tableofcontents.

Изброените инструкции са напълно достатъчни за създаване на базови документи, но далеч не покриват пълните възможности на LaTeX. Тъй като компилатора използван в текстовия процесор е еднопасов, то за да се направи правилно индексиране на препратките и таблицата за съдържанието често се налага компилацията да бъде стартирана два пъти последователно. 

Създаването на LaTeX документи с интегриране на R инструкции в него става чрез изготвянето на стандартен LaTeX документ в който се добавят фрагменти (chunks) на R програмен код. Тези фрагменти се предхождат от инструкция за начало на фрагмента по зададен шаблон (Листинг \ref{listing0179}), а края на фрагмента се обозначава със символа маймунско а.

\begin{lstlisting}[caption=Инструкция за R фрагмент в LaTeX документ, label=listing0179]
<<label-value,option1=value1,option2=value2>>=
@
\end{lstlisting}

Текстовият документ се запазва с разширение „.Rnw“. Трансформацията на комбинира ния код (LaTeX и R) се транслира до Tex с командата $Sweave$ в командния интерпретатор на R (Листинг \ref{listing0180}). Важно е Rnw файлът да се намира в същата директория в която се изпълнява командата за транслиране. 

\begin{lstlisting}[caption=Транслиране от Rnw до Tex, label=listing0180]
library( knitr )

setwd( "~/Desktop" )

Sweave( "./example0002.Rnw" )
\end{lstlisting}

В резултат на транслацията, в съответната директория се генерира Tex файл, който от своя страна се подава на транслатор, изпълняван в конзолата на операционната система, Tex към PDF (Листинг \ref{listing0181}).

\begin{lstlisting}[caption=Транслиране от Tex до PDF, label=listing0181]
pdflatex ./example0002.tex
\end{lstlisting}

За илюстриране на възможностите при генерирането на PDF документи са предложени примерните файлове достъпни на следните електронни адреси:

\begin{lstlisting}[caption=Адрес на примерен R скрипт, label=listing0182]
https://raw.githubusercontent.com/TodorBalabanov/Statistical-Data-Processing-with-R/master/code/Sweave.sty

https://raw.githubusercontent.com/TodorBalabanov/Statistical-Data-Processing-with-R/master/code/example0002.Rnw

https://raw.githubusercontent.com/TodorBalabanov/Statistical-Data-Processing-with-R/master/code/example0003.Rnw
\end{lstlisting}

За да бъде успешно транслирането от Tex към PDF е нужно в директорията с Tex файла да се помести и Sweave.sty файлът, който съдържа дефиниции необходими за извършването на процеса по транслация. 

\begin{lstlisting}[caption=Линейна регресия на разходи спрямо спестявания, label=listing0183]
library( knitr )

setwd( "~/Desktop" )

Sweave( "./example0003.Rnw" )

system("pdflatex ./example0003.tex", intern=TRUE)
\end{lstlisting}

При избора на имена за етикетите в R фрагментите трябва да се избягва използването на специални символи, като празни интервал и точка. Добър вариант е използването на латинските букви и символа за тире. Когато се използва флага $echo$ със стойност $FALSE$ в Tex файла не се включва R програмния код довел до визуализацията на съответните резултати. 

За визуализация на изображения (Листинг \ref{listing0183}) най-удачно е междинната графика да бъде генериране в отделен графичен файл (в случая Fig01.png), който в последствие да бъде добавен като графичен компонент в Rnw документа.

\section{Работа с RMarkdown}

\section*{Заключение}


