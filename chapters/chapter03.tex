\newpage
\chapter{Сложни структури от данни и викане на функции}
\label{chapter03}

\section{Извикване на функции}

Функциите са последователност от инструкции, обособени като едно цяло, така че да са подходящи за многократно извикване. Функциите приемат входящи параметри, могат да имат върната стойност, а символът диес (\#) се използва в началото на ред за коментар. Организацията на програмния текст във функции позволява лесна четимост и по-лесно откриване на програмни дефекти (бъгове). По своята същност командите в конзолата на R са функции, които потребителят извиква. Поради този факт е важно да се знаят възможностите за работа с функции\index{извикване на функции}. За разлика от масово наложените езици за обектно-ориентирано програмиране, в R функциите са по-съществени от обектите.

\begin{lstlisting}[caption=Извикване на функции, label=listing0029]
x <- c(1, 2, 3, 5, 6, 7, 8, 9)
x
[1] 1 2 3 5 6 7 8 9
mean( x )
[1] 5.125
median( x )
[1] 5.5
sd( x )
[1] 2.900123
\end{lstlisting}

Листинг \ref{listing0029} демонстрира извикването на три функции, които получават като единствен аргумент вектор от числени стойности. По-сложните функции може да имат повече аргументи\index{аргументи на функции} и те да се подават по различен начин. 

Всяка функция, която е достъпна в R, има съпровождаща документация\index{документация на функции}, но качеството на тази документация може да варира според уменията на автора ѝ. Най-бързият начин за достъп до информацията е въпросителен (?) пред името на функцията (Листинг \ref{listing0030}).

\begin{lstlisting}[caption=Документация за функциите, label=listing0030]
? mean
?? mean
? median
?? median
? sd
?? sd
\end{lstlisting}

Един въпросителен отваря информацията в локален прозорец, а два въпросителни отварят уеб страницата, съдържаща документацията на функцията.

\begin{lstlisting}[caption=Документация за операции, label=listing0031]
? `+`
? `-`
? `*`
? `/`
\end{lstlisting}

За голяма част от операциите също може да се получи информация по аналогичен начин (Листинг \ref{listing0031}), но операцията\index{документация на операции} трябва да бъде оградена със символа апостроф (\`).

\begin{lstlisting}[caption=Частично търсене, label=listing0032]
apropos( "med" )
[1] "elNamed"        "elNamed<-"      "median"         "median.default"
[5] "medpolish"      "runmed"
\end{lstlisting}

Често потребителите имат идея каква функция търсят, но не се досещат за точното изписване на името ѝ. В такива ситуации е полезна възможността за частично търсене\index{частично търсене}, която предоставя функцията apropos (Листинг \ref{listing0032}).

\section{Вектори}

Векторът\index{вектори} е колекция от елементи, които са от един и същи тип (Листинг \ref{listing0021}). 

\begin{lstlisting}[caption=Вектори от числа и символни низове, label=listing0021]
v1 <- c(1, 3, 2, 1, 5)
v2 <- c("Peter", "Ivan", "Geroge")
\end{lstlisting}

Векторите в R имат значителна роля за езика, тъй като R е вкторизиран език, което го прави различен от конвенционалните програмни езици, като C/C++, C\# или Java. Това означава, че всяка математическа операция се изпълнява върху целия вектор и не е нужно да се обикалят отделните елементи един по един (Листинг \ref{listing0022}). За разлика от математическата концепция в R векторите не се делят на вектор-стълб или вектор-ред. При нужда от вектор-ред или вектор-стълб може да се използват матрици с единична стойност на един от размерите.

\begin{lstlisting}[caption=Базови операции над вектори, label=listing0022]
x <- c(1, 2, 3, 4, 5, 6, 7, 8, 9, 10)
x
[1]  1  2  3  4  5  6  7  8  9 10
x * 5
[1]  5 10 15 20 25 30 35 40 45 50
x + 3
[1]  4  5  6  7  8  9 10 11 12 13
x - 4
[1] -3 -2 -1  0  1  2  3  4  5  6
x / 5
[1] 0.2 0.4 0.6 0.8 1.0 1.2 1.4 1.6 1.8 2.0
x ^ 3
[1]    1    8   27   64  125  216  343  512  729 1000
sqrt( x )
[1] 1.000000 1.414214 1.732051 2.000000 2.236068 2.449490 2.645751 2.828427
[9] 3.000000 3.162278
\end{lstlisting}

Основният начин за създаване на вектор е чрез функцията c, като названието и идва от combine (комбиниране на елементи), но също е възможно да се използва и алтернативен запис (Листинг \ref{listing0023}). 

\begin{lstlisting}[caption=Алтернативен синтаксис за създаване на вектори, label=listing0023]
1:10
[1] 1 2 3 4 5 6 7 8 9 10
10:1
[1] 10 9 8 7 6 5 4 3 2 1
-2:3
[1] -2 -1 0 1 2 3
5:-7
[1] 5 4 3 2 1 0 -1 -2 -3 -4 -5 -6 -7
\end{lstlisting}

Когато двата операнда на операцията са вектори\index{операции с вектори} с еднакви дължини, то операцията се прилага на всичките елементи по двойки (Листинг \ref{listing0024}).

\begin{lstlisting}[caption=Операции между вектори с еднаква дължина, label=listing0024]
x <- 1:10
y <- -10:-1
nchar( x )
[1] 1 1 1 1 1 1 1 1 1 2
nchar( y )
[1] 3 2 2 2 2 2 2 2 2 2
x + y
[1] -9 -7 -5 -3 -1  1  3  5  7  9
x - y
[1] 11 11 11 11 11 11 11 11 11 11
x * y
[1] -10 -18 -24 -28 -30 -30 -28 -24 -18 -10
x / y
[1]  -0.1000000  -0.2222222  -0.3750000  -0.5714286  -0.8333333  -1.2000000
[7]  -1.7500000  -2.6666667  -4.5000000 -10.0000000
x ^ y
[1] 1.000000e+00 1.953125e-03 1.524158e-04 6.103516e-05 6.400000e-05
[6] 1.286008e-04 4.164931e-04 1.953125e-03 1.234568e-02 1.000000e-01
x > y
[1] TRUE TRUE TRUE TRUE TRUE TRUE TRUE TRUE TRUE TRUE
\end{lstlisting}

Когато векторите са с различна дължина, по-късият вектор се превърта и се започва от началото му. Функцията nchar показва колко символа са необходими за изписването на всеки от елементите. 

\begin{lstlisting}[caption=Проверка дали някоя или всички стойности от вектора отговарят на определено условие, label=listing0025]
any(x+y < 0)
[1] TRUE
all(x+y < 0)
[1] FALSE
\end{lstlisting}

При група от проверки може да се установи дали всички елементи на вектора изпълняват определено условие или поне някои елементи го изпълняват (Листинг \ref{listing0025}).

\begin{lstlisting}[caption=Достъп до отделни елементи във вектор, label=listing0026]
x[ 1 ]
[1] 1
x[ 2:3 ]
[1] 2 3
x[ c(2,5,7) ]
[1] 2 5 7
\end{lstlisting}

Достъп до отделни елементи във вектор може да се осъществи по индекс или множество от индекси (Листинг \ref{listing0026}).

\begin{lstlisting}[caption=Имена на елементите във вектора, label=listing0027]
z <- c(One=1, Two=2, Three=3)
z
  One   Two Three 
    1     2     3 
names( z )
[1] "One"   "Two"   "Three"
\end{lstlisting}


R позволява на елементите във вектора да се поставят имена (Листинг \ref{listing0027}).

\begin{lstlisting}[caption=Трансформация на вектор във фактор, label=listing0028]
e <- c("High School", "College", "Masters", "Doctorate")
e

[1] "High School" "College"     "Masters"     "Doctorate"  
f1 <- as.factor( e )
f1
[1] High School College     Masters     Doctorate  
Levels: College Doctorate High School Masters
as.numeric( f1 )
[1] 3 1 4 2

f2 <- factor(c("High School", "College", "Masters", "Doctorate"), 
		 levels=c("High School", "College", "Masters", "Doctorate"),
		 ordered=TRUE)
f2
[1] High School College     Masters     Doctorate  
Levels: High School < College < Masters < Doctorate
as.numeric( f2 )
[1] 1 2 3 4
\end{lstlisting}

Вектор може да бъде трансформиран във фактор\index{фактори} с помощта на функции за трансформация (Листинг \ref{listing0028}). Факторът е тип данни, при който всяка стойност се среща само по един път. Някои множества са неподредени (както е f1) и при тях няма значение редът на елементите, но някои множества са подредени (както е f2) и при тях е от значение редът на елементите. Функцията factor позволява изрично да се зададе какъв е редът на елементите. Факторът е значително по-икономичен на памет от вектора, тъй като се запазват единствено числените стойности на отделните елементи, но пък използването му може да доведе до трудни за откриване логически грешки.

\section{Липсващи стойности}

Липсващи стойности\index{липсващи стойности} в данните е ежедневен проблем за хората обработващи статистическа информация. Причините за липсващите данни могат да идват от различни обстоятелства, примерно пропуснато измерване или дефектирал датчик. R дава две възможности за обозначаване на липсващи стойности в данните (NA и NULL). Макар и да имат сходно значение, тези две стойности водят до различни резултати при различните пресмятания. 

Когато има липсващи стойности в данните съществуват множество начини този факт да бъде отразен. В някои комплекти данни се записва недопустима числена стойност или се използва някаква символна комбинация. В езика R е възприето липсващите стойности да се обозначават с NA (Листинг \ref{listing0033}).

\begin{lstlisting}[caption=Липсващи стойности, label=listing0033]
x <- c(1, NA, 3, NA, 5)
x
[1]  1 NA  3 NA  5
is.na( x )
[1] FALSE  TRUE FALSE  TRUE FALSE
mean( x )
[1] NA
median( x )
[1] NA
sd( x )
[1] NA
\end{lstlisting}

Значението на NULL е лиса, а не изпусната стойност, поради тази причина векторът се редуцира с толкова елементи, колкото NULL стойности има в него (Листинг \ref{listing0034}).

\begin{lstlisting}[caption=Липсващи стойности, label=listing0034]
y <- c(1, NULL, 3, NULL, 5)
y
[1] 1 3 5
is.na( y )
[1] FALSE FALSE FALSE
mean( y )
[1] 3
median( y )
[1] 3
sd( y )
[1] 2
is.null( y )
[1] FALSE
\end{lstlisting}

Тъй като на практика векторът се редуцира с броя на NULL стойностите си, то функцията is.null не е векторизирана, а се отнася за целия обект. 

\section*{Заключение}

В настоящата глава са представени възможностите за извикване на функции в R. Разгледани са начините за извикване на документация за функциите. Представени са начини за работа с липсващи данни и са демонстрирани някои от по-сложните типове данни. 

