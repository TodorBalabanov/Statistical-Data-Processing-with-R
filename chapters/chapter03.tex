\newpage
\chapter{Сложни структури от данни и викане на функции}
\label{chapter03}

\section{Извикване на функции}

Функциите са последователност от инструкции, обособени като едно цяло, така че да са подходящи за многократно извикване. Функциите приемат входящи параметри, могат да имат върната стойност, а символът диес (\#) се използва в началото на ред за коментар. Организацията на програмния текст във функции позволява лесна четимост и по-лесно откриване на програмни дефекти (бъгове). По своята същност командите в конзолата на R са функции, които потребителят извиква. Поради този факт е важно да се знаят възможностите за работа с функции\index{извикване на функции}. За разлика от масово наложените езици за обектно-ориентирано програмиране, в R функциите са по-съществени от обектите.

\begin{lstlisting}[caption=Извикване на функции, label=listing0029]
x <- c(1, 2, 3, 5, 6, 7, 8, 9)
x
[1] 1 2 3 5 6 7 8 9

mean( x )
[1] 5.125

median( x )
[1] 5.5

sd( x )
[1] 2.900123
\end{lstlisting}

Листинг \ref{listing0029} демонстрира извикването на три функции, които получават като единствен аргумент вектор от числени стойности. По-сложните функции може да имат повече аргументи\index{аргументи на функции} и те да се подават по различен начин. 

Всяка функция, която е достъпна в R, има съпровождаща документация\index{документация на функции}, но качеството на тази документация може да варира според уменията на автора ѝ. Най-бързият начин за достъп до информацията е въпросителен (?) пред името на функцията (Листинг \ref{listing0030}).

\begin{lstlisting}[caption=Документация за функциите, label=listing0030]
? mean
?? mean
? median
?? median
? sd
?? sd
\end{lstlisting}

Един въпросителен отваря информацията в локален прозорец, а два въпросителни отварят уеб страницата, съдържаща документацията на функцията.

\begin{lstlisting}[caption=Документация за операции, label=listing0031]
? `+`
? `-`
? `*`
? `/`
\end{lstlisting}

За голяма част от операциите също може да се получи информация по аналогичен начин (Листинг \ref{listing0031}), но операцията\index{документация на операции} трябва да бъде оградена със символа апостроф (\`).

\begin{lstlisting}[caption=Частично търсене, label=listing0032]
apropos( "med" )
[1] "elNamed"        "elNamed<-"      "median"         "median.default"
[5] "medpolish"      "runmed"
\end{lstlisting}

Често потребителите имат идея каква функция търсят, но не се досещат за точното изписване на името ѝ. В такива ситуации е полезна възможността за частично търсене\index{частично търсене}, която предоставя функцията apropos (Листинг \ref{listing0032}).

\section{Вектори}

Векторът\index{вектори} е колекция от елементи, които са от един и същи тип (Листинг \ref{listing0021}). 

\begin{lstlisting}[caption=Вектори от числа и символни низове, label=listing0021]
v1 <- c(1, 3, 2, 1, 5)
v2 <- c("Peter", "Ivan", "Geroge")
\end{lstlisting}

Векторите в R имат значителна роля за езика, тъй като R е вкторизиран език, което го прави различен от конвенционалните програмни езици, като C/C++, C\# или Java. Това означава, че всяка математическа операция се изпълнява върху целия вектор и не е нужно да се обикалят отделните елементи един по един (Листинг \ref{listing0022}). За разлика от математическата концепция в R векторите не се делят на вектор-стълб или вектор-ред. При нужда от вектор-ред или вектор-стълб може да се използват матрици с единична стойност на един от размерите.

\begin{lstlisting}[caption=Базови операции над вектори, label=listing0022]
x <- c(1, 2, 3, 4, 5, 6, 7, 8, 9, 10)
x
[1]  1  2  3  4  5  6  7  8  9 10

x * 5
[1]  5 10 15 20 25 30 35 40 45 50

x + 3
[1]  4  5  6  7  8  9 10 11 12 13

x - 4
[1] -3 -2 -1  0  1  2  3  4  5  6

x / 5
[1] 0.2 0.4 0.6 0.8 1.0 1.2 1.4 1.6 1.8 2.0

x ^ 3
[1]    1    8   27   64  125  216  343  512  729 1000

sqrt( x )
[1] 1.000000 1.414214 1.732051 2.000000 2.236068 2.449490 2.645751 2.828427
[9] 3.000000 3.162278
\end{lstlisting}

Основният начин за създаване на вектор е чрез функцията c, като названието и идва от combine (комбиниране на елементи), но също е възможно да се използва и алтернативен запис (Листинг \ref{listing0023}). 

\begin{lstlisting}[caption=Алтернативен синтаксис за създаване на вектори, label=listing0023]
1:10
[1] 1 2 3 4 5 6 7 8 9 10

10:1
[1] 10 9 8 7 6 5 4 3 2 1

-2:3
[1] -2 -1 0 1 2 3

5:-7
[1] 5 4 3 2 1 0 -1 -2 -3 -4 -5 -6 -7
\end{lstlisting}

Когато двата операнда на операцията са вектори\index{операции с вектори} с еднакви дължини, то операцията се прилага на всичките елементи по двойки (Листинг \ref{listing0024}).

\begin{lstlisting}[caption=Операции между вектори с еднаква дължина, label=listing0024]
x <- 1:10
y <- -10:-1

nchar( x )
[1] 1 1 1 1 1 1 1 1 1 2

nchar( y )
[1] 3 2 2 2 2 2 2 2 2 2

x + y
[1] -9 -7 -5 -3 -1  1  3  5  7  9

x - y
[1] 11 11 11 11 11 11 11 11 11 11

x * y
[1] -10 -18 -24 -28 -30 -30 -28 -24 -18 -10

x / y
[1]  -0.1000000  -0.2222222  -0.3750000  -0.5714286  -0.8333333  -1.2000000
[7]  -1.7500000  -2.6666667  -4.5000000 -10.0000000

x ^ y
[1] 1.000000e+00 1.953125e-03 1.524158e-04 6.103516e-05 6.400000e-05
[6] 1.286008e-04 4.164931e-04 1.953125e-03 1.234568e-02 1.000000e-01

x > y
[1] TRUE TRUE TRUE TRUE TRUE TRUE TRUE TRUE TRUE TRUE
\end{lstlisting}

Когато векторите са с различна дължина, по-късият вектор се превърта и се започва от началото му. Функцията nchar показва колко символа са необходими за изписването на всеки от елементите. 

\begin{lstlisting}[caption=Проверка дали някоя или всички стойности от вектора отговарят на определено условие, label=listing0025]
any(x+y < 0)
[1] TRUE

all(x+y < 0)
[1] FALSE
\end{lstlisting}

При група от проверки може да се установи дали всички елементи на вектора изпълняват определено условие или поне някои елементи го изпълняват (Листинг \ref{listing0025}).

\begin{lstlisting}[caption=Достъп до отделни елементи във вектор, label=listing0026]
x[ 1 ]
[1] 1

x[ 2:3 ]
[1] 2 3

x[ c(2,5,7) ]
[1] 2 5 7
\end{lstlisting}

Достъп до отделни елементи във вектор може да се осъществи по индекс или множество от индекси (Листинг \ref{listing0026}).

\begin{lstlisting}[caption=Имена на елементите във вектора, label=listing0027]
z <- c(One=1, Two=2, Three=3)
z
  One   Two Three 
    1     2     3 

names( z )
[1] "One"   "Two"   "Three"
\end{lstlisting}

R позволява на елементите във вектора да се поставят имена (Листинг \ref{listing0027}).

\begin{lstlisting}[caption=Трансформация на вектор във фактор, label=listing0028]
e <- c("High School", "College", "Masters", "Doctorate")
e

[1] "High School" "College"     "Masters"     "Doctorate"  
f1 <- as.factor( e )
f1
[1] High School College     Masters     Doctorate  
Levels: College Doctorate High School Masters

as.numeric( f1 )
[1] 3 1 4 2

f2 <- factor(c("High School", "College", "Masters", "Doctorate"), 
		 levels=c("High School", "College", "Masters", "Doctorate"),
		 ordered=TRUE)
f2
[1] High School College     Masters     Doctorate  
Levels: High School < College < Masters < Doctorate

as.numeric( f2 )
[1] 1 2 3 4
\end{lstlisting}

Вектор може да бъде трансформиран във фактор\index{фактори} с помощта на функции за трансформация (Листинг \ref{listing0028}). Факторът е тип данни, при който всяка стойност се среща само по един път. Някои множества са неподредени (както е f1) и при тях няма значение редът на елементите, но някои множества са подредени (както е f2) и при тях е от значение редът на елементите. Функцията factor позволява изрично да се зададе какъв е редът на елементите. Факторът е значително по-икономичен на памет от вектора, тъй като се запазват единствено числените стойности на отделните елементи, но пък използването му може да доведе до трудни за откриване логически грешки.

\begin{lstlisting}[caption=Вектори с латинските букви, label=listing0051]
letters
 [1] "a" "b" "c" "d" "e" "f" "g" "h" "i" "j" "k" "l" "m" "n" "o" "p" "q" "r"
[19] "s" "t" "u" "v" "w" "x" "y" "z"

LETTERS
 [1] "A" "B" "C" "D" "E" "F" "G" "H" "I" "J" "K" "L" "M" "N" "O" "P" "Q" "R"
[19] "S" "T" "U" "V" "W" "X" "Y" "Z"
\end{lstlisting}

В R има два специално предефинирани вектора, които съдържат буквите от латинската азбука (Листинг \ref{listing0051}).

\section{Липсващи стойности}

Липсващи стойности\index{липсващи стойности} в данните е ежедневен проблем за хората обработващи статистическа информация. Причините за липсващите данни могат да идват от различни обстоятелства, примерно пропуснато измерване или дефектирал датчик. R дава две възможности за обозначаване на липсващи стойности в данните (NA и NULL). Макар и да имат сходно значение, тези две стойности водят до различни резултати при различните пресмятания. 

Когато има липсващи стойности в данните съществуват множество начини този факт да бъде отразен. В някои комплекти данни се записва недопустима числена стойност или се използва някаква символна комбинация. В езика R е възприето липсващите стойности да се обозначават с NA (Листинг \ref{listing0033}).

\begin{lstlisting}[caption=Липсващи стойности, label=listing0033]
x <- c(1, NA, 3, NA, 5)
x
[1]  1 NA  3 NA  5

is.na( x )
[1] FALSE  TRUE FALSE  TRUE FALSE

mean( x )
[1] NA

median( x )
[1] NA

sd( x )
[1] NA
\end{lstlisting}

Значението на NULL е лиса, а не изпусната стойност, поради тази причина векторът се редуцира с толкова елементи, колкото NULL стойности има в него (Листинг \ref{listing0034}).

\begin{lstlisting}[caption=Липсващи стойности, label=listing0034]
y <- c(1, NULL, 3, NULL, 5)
y
[1] 1 3 5

is.na( y )
[1] FALSE FALSE FALSE

mean( y )
[1] 3

median( y )
[1] 3

sd( y )
[1] 2

is.null( y )
[1] FALSE
\end{lstlisting}

Тъй като на практика векторът се редуцира с броя на NULL стойностите си, то функцията is.null не е векторизирана, а се отнася за целия обект. 

\section{Рамкирани данни}

Рамкираните данни\index{рамкирани данни} (data.frame) са една от най-полезните структури от данни в езика R. Най-интуитивната аналогия за рамкирани данни е един лист (data sheet) в Microsft Excel, състоящ се от колони и редове. В термините на статистиката, всяка колона е наблюдавана променлива, а всеки ред е едно конкретно наблюдение (измерване). В термините на R, всяка колона е вектор, а дължината на всичките вектори е една и съща. По този начин всяка колона може да съдържа различни типове данни. Също така, в рамките на една колона всички елементи са от един и същи тип. 

Съществуват множество начини да се създадат рамкирани данни, но най-лесният е с функцията data.frame.

\begin{lstlisting}[caption=Създаване на рамкирани данни, label=listing0035]
x <- sample(1:5)
x
[1] 4 1 3 2 5

y <- sample(-2:2)
y
[1]  0 -2  1 -1  2

q <- c("Football", "Basketball", "Volleyball", "Handball", "Rugby")
q
[1] "Football"   "Basketball" "Volleyball" "Handball"   "Rugby"

df1 <- data.frame(x, y, q)
df1
  x  y          q
1 4  0   Football
2 1 -2 Basketball
3 3  1 Volleyball
4 2 -1   Handball
5 5  2      Rugby
\end{lstlisting}

Листинг \ref{listing0035} демонстрира създаването на рамкирани данни от два вектора с числа (sample служи за разбъркване на стойностите по случаен начин) и един вектор със символни низове. Така получената структура е с размери 5x3 и се състои от три вектора. Имената на колоните се вземат служебно, но е възможно те да бъдат определени при създаването на самата структура (Листинг \ref{listing0036}).

\begin{lstlisting}[caption=Създаване на рамкирани данни с имена на колоните, label=listing0036]
df2 <- data.frame(First=x, Second=y, Sport=q)
rownames( df2 ) <- c("One", "Two", "Three", "Four", "Five")
df2
      First Second      Sport
One       4      0   Football
Two       1     -2 Basketball
Three     3      1 Volleyball
Four      2     -1   Handball
Five      5      2      Rugby
\end{lstlisting}

Рамкираните данни имат множество атрибути, като най-съществените са броя редове и броя колони (Листинг \ref{listing0037}). Атрибутите имат съществено значение при прилагането на различните алгоритми за статистически анализ. 

\begin{lstlisting}[caption=Атрибути на рамкираните данни, label=listing0037]
nrow( df1 )
[1] 5

ncol( df1 )
[1] 3

dim( df1 )
[1] 5 3

names( df2 )
[1] "First"  "Second" "Sport"

rownames( df2 )
[1] "One"   "Two"   "Three" "Four"  "Five" 

head(df1, n=3)
  x  y          q
1 4  0   Football
2 1 -2 Basketball
3 3  1 Volleyball

tail(df1, n=3)
  x  y          q
3 3  1 Volleyball
4 2 -1   Handball
5 5  2      Rugby

class( df1 )
[1] "data.frame"
\end{lstlisting}

Също така, може да се проверят имената на колоните, имената на редовете, с функцията head първите няколко реда, а с функцията tail последните няколко реда.

\begin{lstlisting}[caption=Фактори в рамковите данни, label=listing0038]
df2[1, 2]
[1] 0

df2[3, 2:3]
      Second      Sport
Three      1 Volleyball

df2$Sport
[1] Football   Basketball Volleyball Handball   Rugby     
Levels: Basketball Football Handball Rugby Volleyball

class( df2$Sport )
[1] "factor"

df2$Sport[1:2]
[1] Football   Basketball
Levels: Basketball Football Handball Rugby Volleyball

df2[3, ]
      First Second      Sport
Three     3      1 Volleyball

df2[, c("First", "Sport")]
      First      Sport
One       4   Football
Two       1 Basketball
Three     3 Volleyball
Four      2   Handball
Five      5      Rugby
\end{lstlisting}

Рамкираните данни позволяват достъп до елементите като индекси\index{достъп по индекс} на двумерен масив или директно с адресиране на конкретна колона (Листинг \ref{listing0038}). Достъпът до цял ред става без указване на колона. За достъп до колоните по име се съставя вектор с имената на колоните. 

\begin{lstlisting}[caption=Вътрешно представяне на факторите, label=listing0039]
f1 <- factor( c("Sofia", "Plovdiv", "Varna", "Burgas", "Ruse") )
f1
[1] Sofia   Plovdiv Varna   Burgas  Ruse   
Levels: Burgas Plovdiv Ruse Sofia Varna

model.matrix(~f1 - 1)
  f1Burgas f1Plovdiv f1Ruse f1Sofia f1Varna
1        0         0      0       1       0
2        0         1      0       0       0
3        0         0      0       0       1
4        1         0      0       0       0
5        0         0      1       0       0
attr(,"assign")
[1] 1 1 1 1 1
attr(,"contrasts")
attr(,"contrasts")$f1
[1] "contr.treatment"
\end{lstlisting}

Тъй като факторите са малко по-различни от векторите, как се представят вътрешно в рамкираните данни може да се проследи с функцията model.matrix (Листинг \ref{listing0039}).

\section{Списъци}

В някои ситуации е нужно да се ползва контейнер, който да съдържа обекти от различни типове. В R това се постига със списъчните структури. Този тип структури могат да съдържат голям брой и различни по типове елементи. Списъците се създават с функцията list (Листинг \ref{listing0040}).

\begin{lstlisting}[caption=Създаване на списък, label=listing0040]
l1 <- list(1, 2, 3, 4, 5)
l1
[[1]]
[1] 1

[[2]]
[1] 2

[[3]]
[1] 3

[[4]]
[1] 4

[[5]]
[1] 5
\end{lstlisting}

Всеки елемент в списъка е самостоятелен (Листинг \ref{listing0040}), но е възможно да има и списък с единствен елемент, който е вектор (Листинг \ref{listing0041}).

\begin{lstlisting}[caption=Вектор в списък, label=listing0041]
l2 <- list( c(1, 2, 3, 4, 5) )
l2
[[1]]
[1] 1 2 3 4 5
\end{lstlisting}

Разнородни елементи на списък са показани в листинг \ref{listing0042}.

\begin{lstlisting}[caption=Списък с разнородни данни, label=listing0042]
l3 <- list( df2, 1:5, l1 )
l3
[[1]]
      First Second      Sport
One       4      0   Football
Two       1     -2 Basketball
Three     3      1 Volleyball
Four      2     -1   Handball
Five      5      2      Rugby

[[2]]
[1] 1 2 3 4 5

[[3]]
[[3]][[1]]
[1] 1

[[3]][[2]]
[1] 2

[[3]][[3]]
[1] 3

[[3]][[4]]
[1] 4

[[3]][[5]]
[1] 5
\end{lstlisting}

По аналогия с рамкираните данни, списъците също могат да съдържат наименования на елементите си (Листинг \ref{listing0043}).

\begin{lstlisting}[caption=Названия на елементите в списъка, label=listing0043]
names( l3 ) <- c("Frame", "Vector", "Element")
l3
$Frame
      First Second      Sport
One       4      0   Football
Two       1     -2 Basketball
Three     3      1 Volleyball
Four      2     -1   Handball
Five      5      2      Rugby

$Vector
[1] 1 2 3 4 5

$Element
$Element[[1]]
[1] 1

$Element[[2]]
[1] 2

$Element[[3]]
[1] 3

$Element[[4]]
[1] 4

$Element[[5]]
[1] 5
\end{lstlisting}

Достъпът до елементите на списъка може да стане по индекс\index{достъп по индекс} или по название на елемента (Листинг \ref{listing0044}).

\begin{lstlisting}[caption=Достъп до елементите на списъка, label=listing0044]
l3[ 1 ]
$Frame
      First Second      Sport
One       4      0   Football
Two       1     -2 Basketball
Three     3      1 Volleyball
Four      2     -1   Handball
Five      5      2      Rugby

l3[ "Frame" ]
$Frame
      First Second      Sport
One       4      0   Football
Two       1     -2 Basketball
Three     3      1 Volleyball
Four      2     -1   Handball
Five      5      2      Rugby
\end{lstlisting}

Чрез вложено позоваване може да се достъпи конкретен елемент (Листинг \ref{listing0045}). Тъй като сложните структури от данни могат да съдържат на свой ред сложни структури от данни, вложеното позоваване може да добие доста неприветлив вид. 

\begin{lstlisting}[caption=Вложено позоваване, label=listing0045]
l3[[1]]
      First Second      Sport
One       4      0   Football
Two       1     -2 Basketball
Three     3      1 Volleyball
Four      2     -1   Handball
Five      5      2      Rugby

l3[["Frame"]]$Sport
[1] Football   Basketball Volleyball Handball   Rugby     
Levels: Basketball Football Handball Rugby Volleyball
\end{lstlisting}

Добавянето на елемент към списъка става с директно позоваване към елемента на който индекс трябва да попадне новият елемент, дори и това място да не е предварително предвидено (Листинг \ref{listing0046}).

\begin{lstlisting}[caption=Добавяне на елемент, label=listing0046]
l3[ 4 ] <- "Games"
l3
$Frame
      First Second      Sport
One       4      0   Football
Two       1     -2 Basketball
Three     3      1 Volleyball
Four      2     -1   Handball
Five      5      2      Rugby

$Vector
[1] 1 2 3 4 5

$Element
$Element[[1]]
[1] 1

$Element[[2]]
[1] 2

$Element[[3]]
[1] 3

$Element[[4]]
[1] 4

$Element[[5]]
[1] 5

[[4]]
[1] "Games"

length( l3 )
[1] 4
\end{lstlisting}

\section{Матрици}

Една от най-важните структури в математиката и статистиката е матрицата. Матриците\index{матрици} в R много наподобяват рамкираните данни, тъй като се състоят от колони и редове, с разликата че всички елементи на матрицата са еднотипни. По аналогия с векторите, матриците също се обработват с матрична аритметика, а не с обикаляне на елементите един по един. 

\begin{lstlisting}[caption=Създаване на матрици, label=listing0047]
m1 <- matrix(1:6, nrow=3)
m1
     [,1] [,2]
[1,]    1    4
[2,]    2    5
[3,]    3    6

m2 <- matrix(7:12, nrow=3)
m2
     [,1] [,2]
[1,]    7   10
[2,]    8   11
[3,]    9   12

m3 <- matrix(7:18, nrow=2)
m3
     [,1] [,2] [,3] [,4] [,5] [,6]
[1,]    7    9   11   13   15   17
[2,]    8   10   12   14   16   18
\end{lstlisting}

Създаването на матрици става с функцията matrix (Листинг \ref{listing0047}). От съществено значение е размерът на матрицата, както и попълнването на елементите, което се случва колона по колона. 

\begin{lstlisting}[caption=Операции с матрици, label=listing0048]
nrow( m1 )
[1] 3

ncol( m1 )
[1] 2

dim( m1 )
[1] 3 2

m1 + m2
     [,1] [,2]
[1,]    8   14
[2,]   10   16
[3,]   12   18
 
m1 * m2
     [,1] [,2]
[1,]    7   40
[2,]   16   55
[3,]   27   72

m1 == m2
      [,1]  [,2]
[1,] FALSE FALSE
[2,] FALSE FALSE
[3,] FALSE FALSE
\end{lstlisting}

Повечето матрични операции се изпълняват елемент за елемент (Листинг \ref{listing0048}), но матричното умножение е малко по-особено тъй като изисква съчетаване на размерите по колони и редове (Листинг \ref{listing0049}).

\begin{lstlisting}[caption=Матрично умножение, label=listing0049]
m1 %*% t(m2)
     [,1] [,2] [,3]
[1,]   47   52   57
[2,]   64   71   78
[3,]   81   90   99
\end{lstlisting}

Както при рамкираните данни, така и при матриците може да има имена на колоните и редовете (Листинг \ref{listing0050}). Тази възможност значително подобрява визуализирането на данните след извършването на математическите пресмятания. 

\begin{lstlisting}[caption=Имена на колоните и редовете, label=listing0050]
colnames( m1 ) <- c("First", "Second")
rownames( m1 ) <- c("One", "Two", "Three")
m1
      First Second
One       1      4
Two       2      5
Three     3      6

colnames( m2 ) <- c("Left", "Right")
rownames( m2 ) <- c("1st", "2nd", "3rd")
m2
    Left Right
1st    7    10
2nd    8    11
3rd    9    12
\end{lstlisting}

Функцията t служи за транспониране на матрица. Транспонирането най-често се налага при матричното умножение (Листинг \ref{listing0049}). За да бъде транспонирана една матрица, елементите й с разменят симетрично, спрямо главния диагонал. Не е нужно матрицата да бъде квадратна за да бъде транспонирана. Транспонирането е валидно и за правоъгълни матрици. 

\section{Масиви}

Масивът\index{масиви} по своята същност е многомерен вектор. Елементите на масива са еднотипни и достъпът до тях също се осъществява по индекс с квадратни скоби. Първият индекс е за ред, а вторият за колона и така нататък за по-високите размерности.

\begin{lstlisting}[caption=Работа с масиви, label=listing0052]
a1 <- array(1:12, dim = c(2, 3, 2))
a1
, , 1

     [,1] [,2] [,3]
[1,]    1    3    5
[2,]    2    4    6

, , 2

     [,1] [,2] [,3]
[1,]    7    9   11
[2,]    8   10   12

a1[1, , ]
     [,1] [,2]
[1,]    1    7
[2,]    3    9
[3,]    5   11
 
a1[1, , 1]
[1] 1 3 5
 
a1[, , 1]
     [,1] [,2] [,3]
[1,]    1    3    5
[2,]    2    4    6
\end{lstlisting}

Основната разлика между матриците и масивите е, че матриците са ограничено до две размерности, докато масивите могат да имат много измерения.

\section*{Заключение}

В настоящата глава са представени възможностите за извикване на функции в R. Разгледани са начините за извикване на документация за функциите. Представени са начини за работа с липсващи данни и са демонстрирани някои от по-сложните типове данни. 

