Настоящото учебно помагало съдържа серия от примери по дисциплините „Анализ на Данни с R“, която се изучава в докторантската програма на „Център за обучение“ към „Българска академия на науките“. Изложението е насочено към аудитория със сериозни познания в областта на математиката и статистиката. 

Помагалото е съставено от практически примери, обхващащи цялостния цикъл за статистическа обработка на информацията, което включва събиране на информация, предварителна обработка на суровите данни, прилагане на методи за статистическа обработка, визуализация на резултатите и оформление на финалните документи. Структурата на учебното помагало позволява отделни части от него да послужат при разработването на курсови задачи и/или дипломни работи.

Свободният достъп до помагалото дава възможност то да бъде използвано в учебния процес на курсове с подобно съдържание и на други учебни заведения. Изложеният материал дава възможност за допълване на съществуващите магистърски програми, примерно в областта на приложната математика и статистиката.

Посочените в библиографията литературни източници са предимно със справочен характер, но дават възможност на заинтересуваните читатели да разширят познанията си в засегнатите области. 

This handbook contains a set of examples on the Statistical Data Analysis with R course, part of the doctoral program in Bulgarian Academy of Sciences Training Department. It targets an audience with a strong knowledge in the field of the mathematics and statistics.

The handbook is made up of practical examples covering the entire cycle of statistical data analysis, which includes – data collection, raw data manipulation, statistical calculations, results visualization and results preprinting. The structure of the handbook is selected in such way that parts of it can be used to develop course assignments and/or diploma theses.

Free access to the handbook enables it to be taken up in other courses present in the curricula of other educational institutions. The presented material allows the extension of the existing master programs, for example in the field of applied mathematics and statistics.

The information sources used are mostly of a reference nature but allow interested readers to expand their knowledge in the areas concerned.

