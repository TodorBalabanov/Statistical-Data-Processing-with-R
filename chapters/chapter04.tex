\newpage
\chapter{Въвеждане на данни и извеждане на графики}
\label{chapter04}

Статистическата обработка на данни в R започва с въвеждането на събраната информация\index{въвеждане на информация} и завършва с визуализация на резултатите\index{визуализация на резултатите} от анализа. Тези две фази от етапа на статистическата обработка имат своята важност, тъй като входящите данни силно определят надеждността на извършвания анализ, а правилно визуализираните резултати определят степента на разбиране, която ще постигне аудиторията пред която анализът се представя. 

\section{Въвеждане на данни от външни източници}

Данните в примерите до тази глава бяха фиксирани и се въвеждаха ръчно от конзолата, в интерактивен режим. Този начин на работа не е най-рационалния, когато се правят модели и с данните за модела се провеждат многократни експерименти. Обичайната практика е командите за съставянето на модела да бъдат написани в обикновен текстов файл, с разширение „.r“, а данните да бъдат зареждани от външен файл\index{четене от файл}. Този начин на работа позволява да бъдат създадени множество модели, които често се различават по нещо дребно, и да бъдат зареждани различни входни данни, примерно за различни периоди на измерване. 

Продуктът R позволява множество различни начини за въвеждане на данни в системата, но най-достъпният начин е през CSV (Comma Separated Values) файлове. CSV файловият формат е текстов файлов формат, който позволява таблично представяне на данни (колони и редове). CSV може да бъде четен и редактиран с обикновен текстов редактор, като Notepad под Microsoft Windows, TextEdit под Mac OS X или Nano под Linux. CSV комфортно се визуализира и обработва от продуктите Microsoft Excel, OpenOffice Calc и Libre Calc. 

\begin{lstlisting}[caption=Зареждане на данни от CSV файл, label=listing0053]
data <- read.table(file="http://raw.githubusercontent.com/TodorBalabanov/Statistical-Data-Processing-with-R/master/data/tomato.csv", header=TRUE, sep=",")

head( data )
  Round             Tomato Price      Source Sweet Acid Color Texture Overall
1     1         Simpson SM  3.99 Whole Foods   2.8  2.8   3.7     3.4     3.4
2     1  Tuttorosso (blue)  2.99     Pioneer   3.3  2.8   3.4     3.0     2.9
3     1 Tuttorosso (green)  0.99     Pioneer   2.8  2.6   3.3     2.8     2.9
4     1     La Fede SM DOP  3.99   Shop Rite   2.6  2.8   3.0     2.3     2.8
5     2       Cento SM DOP  5.49  D Agostino   3.3  3.1   2.9     2.8     3.1
6     2      Cento Organic  4.99  D Agostino   3.2  2.9   2.9     3.1     2.9
  Avg.of.Totals Total.of.Avg
1          16.1         16.1
2          15.3         15.3
3          14.3         14.3
4          13.4         13.4
5          14.4         15.2
6          15.5         15.1

tail( data )
   Round                   Tomato Price      Source Sweet Acid Color Texture
11     3       Scotts Backyard SM  0.00  Home Grown   1.6  2.9   3.1     2.4
12     3 Di Casa Barone (organic) 12.80      Eataly   1.7  3.6   3.8     2.3
13     4         Trader Joes Plum  1.49 Trader Joes   3.4  3.3   4.0     3.6
14     4          365 Whole Foods  1.49 Whole Foods   2.8  2.7   3.4     3.1
15     4        Muir Glen Organic  3.19 Whole Foods   2.9  2.8   2.7     3.2
16     4        Bionature Organic  3.39 Whole Foods   2.4  3.3   3.4     3.2
   Overall Avg.of.Totals Total.of.Avg
11     1.9          11.9         11.9
12     1.4          12.7         12.7
13     3.9          17.8         18.2
14     3.1          14.8         15.2
15     3.1          14.8         14.7
16     2.8          15.1         15.2
\end{lstlisting}

Зареждането на CSV в R най-ефективно се постига с функцията read.table (Листинг \ref{listing0053}). Резултатът от четенето е обект от тип рамкирани данни. При викането на функцията параметрите се подават с явно изписване на имената им. Точният адрес на файла се подава в кавички, а когато първият ред от данните е заглавен ред се подава флаг за заглавен ред. Третият аргумент е за указване на разделителя в редовете, тъй като не винаги този разделител е запетая. Често софтуерните продукти за електронни таблици поставят символа за табулация, като разделител между данните на един ред. Символът табулация попада в групата на белите символи (white spaces), тъй като не се изобразява видимо, а с празно пространство. Когато трябва да бъде подаден като аргумент за разделител се използва комбинацията от обратна наклонена черта и буквата t (\textbackslash t).

\section*{Заключение}

Коректното въвеждане на данните в системата е от изключителна важност за осъществяването на коректен анализ и постигането на приемливи статистически резултати. В другия край на процеса е самото визуализиране на получените резултати и максималната експресивност, която може да се постигне за представянето пред широка аудитория. Тези две стъпки от процеса по статистически анализ са свързани с въвеждането на информацията и графичното визуализиране на получените резултати. 

