\documentclass{beamer}

\usepackage[T2A]{fontenc}
\usepackage[utf8x]{inputenc}
\usepackage[english,bulgarian]{babel}
\usepackage{multirow}
\usepackage{ragged2e}

\mode<presentation> {
	\usetheme{Berlin}
}

%\usebackgroundtemplate {
%	\includegraphics[width=370px, height=270px, trim=0 0 0 -80px]{background}
%}

\graphicspath{{../images/}}

\title{Оформление на резултатите за печатно и електронно представяне}
\subtitle{Статистическа обработка на данни с R}

\author{Пламен Петров и Тодор Балабанов}

\date{7.VI.2020}

\institute[ЦО и ИИКТ към БАН] {
	Център за обучение \\
	Институт по информационни и комуникационни технологии \\ 
	Българската академия на науките \\
	\medskip
	\textit{p.petrov@iit.bas.bg todorb@iinf.bas.bg}
}

\addtobeamertemplate{navigation symbols}{}{
	\usebeamerfont{footline}
	\usebeamercolor[fg]{footline}
	\hspace{1em}
	\insertframenumber/\inserttotalframenumber
}

\begin{document}

\begin{frame}
	\titlepage
\end{frame}

\begin{frame}
\begin{exampleblock}{Acknowledgments}
\justify These teaching materials are funded by Velbazhd Software LLC and it is partially supported by the Bulgarian Ministry of Education and Science (contract D01–205/23.11.2018) under the National Scientific Program ``Information and Communication Technologies for a Single Digital Market in Science, Education and Security (ICTinSES)'', approved by DCM \# 577/17.08.2018.
\end{exampleblock}
\end{frame}

\section*{Теми}
\begin{frame}[shrink]
	\frametitle{Съдържание}
	\tableofcontents
\end{frame}

\section{Работа с LaTeX}

\begin{frame}
\center \huge{Работа с LaTeX}
\end{frame}

\subsection{Добра интеграция с R}

\begin{frame}
\frametitle{Код компилиран до текст}
\begin{block}{Инструкция за R фрагмент в LaTeX документ}
<<label-value, option1=value1, option2=value2>>=

\texttt{@}
\end{block}

\begin{block}{Транслиране от Rnw до Tex}
library( knitr )

setwd( $"$\textasciitilde /Desktop$"$ )

Sweave( $"$./example0002.Rnw$"$ )
\end{block}

\begin{block}{Транслиране от Tex до PDF}
pdflatex ./example0002.tex
\end{block}
\end{frame}

\begin{frame}
\frametitle{Примерни документи}
\begin{block}{URL адреси}
https://raw.githubusercontent.com/TodorBalabanov/Statistical-Data-Processing-with-R/master/code/Sweave.sty

https://raw.githubusercontent.com/TodorBalabanov/Statistical-Data-Processing-with-R/master/code/example0002.Rnw

https://raw.githubusercontent.com/TodorBalabanov/Statistical-Data-Processing-with-R/master/code/example0003.Rnw
\end{block}
\end{frame}

\begin{frame}
\frametitle{Примерно представяне}
\begin{block}{Линейна регресия на разходи спрямо спестявания}
library( knitr )

setwd( $"$\textasciitilde /Desktop$"$ )

Sweave( $"$./example0003.Rnw$"$ )

system($"$pdflatex ./example0003.tex$"$, intern=TRUE)
\end{block}
\end{frame}

\section{Работа с RMarkdown}

\begin{frame}
\center \huge{Работа с RMarkdown}
\end{frame}

\subsection{Статични документи}

\begin{frame}
\frametitle{Опростен синтаксис за тагиране}
\begin{block}{Адрес на примерни RMarkdown документ}
https://raw.githubusercontent.com/TodorBalabanov/Statistical-Data-Processing-with-R/master/code/example0004.Rmd

https://raw.githubusercontent.com/TodorBalabanov/Statistical-Data-Processing-with-R/master/code/example0005.Rmd

https://raw.githubusercontent.com/TodorBalabanov/Statistical-Data-Processing-with-R/master/code/example0007.Rmd
\end{block}
\end{frame}

\begin{frame}
\frametitle{Преход между файловите формати}
\begin{block}{Транслиране от RMarkdown в HTML и PDF}
library( knitr )

library( ggplot2 )

library( rmarkdown )

setwd( $"$\textasciitilde /Desktop$"$ )

render( $"$./example0004.Rmd$"$ )

render( $"$./example0005.Rmd$"$ )

render( $"$./example0007.Rmd$"$ )
\end{block}
\end{frame}

\subsection{Динамични документи}

\begin{frame}
\frametitle{Динамично съдържание}
\begin{block}{Адрес на примерни интерактивни документ}
https://raw.githubusercontent.com/TodorBalabanov/Statistical-Data-Processing-with-R/master/code/example0006.Rmd
\end{block}
\end{frame}

\begin{frame}
\frametitle{Взаимодействие с потребителя}
\begin{block}{Създаване на интерактивни документи}
library( DT )

library( dplyr )

library( knitr )

library( ggplot2 )

library( rmarkdown )

library( d3heatmap )

setwd( $"$\textasciitilde /Desktop$"$ )

render( $"$./example0006.Rmd$"$ )
\end{block}
\end{frame}

\begin{frame}
\frametitle{}
\begin{block}{}
\end{block}
\end{frame}

\section{Заключение}

\begin{frame}
\center \huge{Заключение}
\end{frame}

\subsection{Дискусия}

\begin{frame}
\frametitle{Въпроси и отговори}
\center \huge{Благодаря за вниманието!}
\end{frame}

\end{document}
