\documentclass{beamer}

\usepackage[T2A]{fontenc}
\usepackage[utf8x]{inputenc}
\usepackage[english,bulgarian]{babel}

\mode<presentation> {
	\usetheme{Berlin}
}

%\usebackgroundtemplate {
%	\includegraphics[width=370px, height=270px, trim=0 0 0 -80px]{background}
%}

\graphicspath{{../images/}}

\title{Сложни структури от данни и извикване на функции}
\subtitle{Статистическа обработка на данни с R}

\author{Пламен Петров и Тодор Балабанов}

\date{05.V.2020}

\institute[ЦО и ИИКТ към БАН] {
	Център за обучение \\
	Институт по информационни и комуникационни технологии \\ 
	Българската академия на науките \\
	\medskip
	\textit{p.petrov@iit.bas.bg todorb@iinf.bas.bg}
}

\addtobeamertemplate{navigation symbols}{}{
	\usebeamerfont{footline}
	\usebeamercolor[fg]{footline}
	\hspace{1em}
	\insertframenumber/\inserttotalframenumber
}

\begin{document}

\begin{frame}
	\titlepage
\end{frame}

\section*{Теми}
\begin{frame}
	\frametitle{Съдържание}
	\tableofcontents
\end{frame}

\section{Извикване на функции}

\begin{frame}
\center \huge{Извикване на функции}
\end{frame}

\subsection{Обща информация за функциите}

\begin{frame}
\frametitle{Групи от команди}
\begin{block}{Извикване на функции}
x <- c(1, 2, 3, 5, 6, 7, 8, 9)

[1] 1 2 3 5 6 7 8 9

mean( x )

[1] 5.125

median( x )

[1] 5.5

sd( x )

[1] 2.900123
\end{block}
\end{frame}

\begin{frame}
\frametitle{Помощна информация}
\begin{block}{Документация за функциите}
? mean

?? mean

? median

?? median

? sd

?? sd
\end{block}
\end{frame}

\begin{frame}
\frametitle{Помощна информация}
\begin{block}{Документация за операции}
? `+`

? `-`

? `*`

? `/`
\end{block}
\end{frame}

\subsection{Търсене на функции}

\begin{frame}
\frametitle{Възможности за намиране на функция}
\begin{block}{Частично търсене}
apropos( "med" )

[1] "elNamed"        "elNamed<-"      "median"         "median.default"

[5] "medpolish"      "runmed"
\end{block}
\end{frame}

\section{Базови контейнери за данни}

\begin{frame}
\center \huge{Базови контейнери за данни}
\end{frame}

\subsection{Вектори}

\subsection{Списъци}

\subsection{Матрици}

\subsection{Масиви}

\subsection{Липсващи стойности}

\section{Структури подходящи за статистическа обработка}

\begin{frame}
\center \huge{Структури подходящи за статистическа обработка}
\end{frame}

\subsection{Рамкирани данни}

\section{Заключение}

\begin{frame}
\center \huge{Заключение}
\end{frame}

\subsection{Дискусия}

\begin{frame}
\frametitle{Въпроси и отговори}
\center \huge{Благодаря за вниманието!}
\end{frame}

\end{document}
