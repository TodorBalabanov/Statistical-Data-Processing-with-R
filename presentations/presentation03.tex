\documentclass{beamer}

\usepackage[T2A]{fontenc}
\usepackage[utf8x]{inputenc}
\usepackage[english,bulgarian]{babel}

\mode<presentation> {
	\usetheme{Berlin}
}

%\usebackgroundtemplate {
%	\includegraphics[width=370px, height=270px, trim=0 0 0 -80px]{background}
%}

\graphicspath{{../images/}}

\title{Сложни структури от данни и извикване на функции}
\subtitle{Статистическа обработка на данни с R}

\author{Пламен Петров и Тодор Балабанов}

\date{05.V.2020}

\institute[ЦО и ИИКТ към БАН] {
	Център за обучение \\
	Институт по информационни и комуникационни технологии \\ 
	Българската академия на науките \\
	\medskip
	\textit{p.petrov@iit.bas.bg todorb@iinf.bas.bg}
}

\addtobeamertemplate{navigation symbols}{}{
	\usebeamerfont{footline}
	\usebeamercolor[fg]{footline}
	\hspace{1em}
	\insertframenumber/\inserttotalframenumber
}

\begin{document}

\begin{frame}
	\titlepage
\end{frame}

\section*{Теми}
\begin{frame}
	\frametitle{Съдържание}
	\tableofcontents
\end{frame}

\section{Извикване на функции}

\begin{frame}
\center \huge{Извикване на функции}
\end{frame}

\subsection{Обща информация за функциите}

\begin{frame}
\frametitle{Групи от команди}
\begin{block}{Извикване на функции}
x <- c(1, 2, 3, 5, 6, 7, 8, 9)

[1] 1 2 3 5 6 7 8 9

mean( x )

[1] 5.125

median( x )

[1] 5.5

sd( x )

[1] 2.900123
\end{block}
\end{frame}

\begin{frame}
\frametitle{Помощна информация}
\begin{block}{Документация за функциите}
? mean

?? mean

? median

?? median

? sd

?? sd
\end{block}
\end{frame}

\begin{frame}
\frametitle{Помощна информация}
\begin{block}{Документация за операции}
? `+`

? `-`

? `*`

? `/`
\end{block}
\end{frame}

\subsection{Търсене на функции}

\begin{frame}
\frametitle{Възможности за намиране на функция}
\begin{block}{Частично търсене}
apropos( "med" )

[1] "elNamed"        "elNamed<-"      "median"         "median.default"

[5] "medpolish"      "runmed"
\end{block}
\end{frame}

\section{Базови контейнери за данни}

\begin{frame}
\center \huge{Базови контейнери за данни}
\end{frame}

\subsection{Вектори}

\begin{frame}
\frametitle{Подредено множество от стойности}
\begin{block}{Вектор от числа и вектор от символни низове}
v1 <- c(1, 3, 2, 1, 5)

v2 <- c("Peter", "Ivan", "Geroge")
\end{block}
\end{frame}

\begin{frame}
\frametitle{Векторни пресмятания}
\begin{block}{Аритметични операции над вектори}
x <- c(1, 2, 3, 4, 5, 6, 7, 8, 9, 10)

[1]  1  2  3  4  5  6  7  8  9 10

x * 5

[1]  5 10 15 20 25 30 35 40 45 50

x + 3

[1]  4  5  6  7  8  9 10 11 12 13

x - 4

[1] -3 -2 -1  0  1  2  3  4  5  6

x / 5

[1] 0.2 0.4 0.6 0.8 1.0 1.2 1.4 1.6 1.8 2.0
\end{block}
\end{frame}

\begin{frame}
\frametitle{Векторни пресмятания}
\begin{block}{Степенуване на вектори}
x\textasciicircum 3

[1]    1    8   27   64  125  216  343  512  729 1000

sqrt( x )

[1] 1.000000 1.414214 1.732051 2.000000 2.236068 2.449490 2.645751 2.828427

[9] 3.000000 3.162278
\end{block}
\end{frame}

\begin{frame}
\frametitle{Задаване на стойности}
\begin{block}{Алтернативен синтаксис за създаване на вектори}
1:10

[1] 1 2 3 4 5 6 7 8 9 10

10:1

[1] 10 9 8 7 6 5 4 3 2 1

-2:3

[1] -2 -1 0 1 2 3

5:-7

[1] 5 4 3 2 1 0 -1 -2 -3 -4 -5 -6 -7
\end{block}
\end{frame}

\begin{frame}
\frametitle{При вектори с еднаква дължина}
\begin{block}{Събиране и изваждане}
x <- 1:10

y <- -10:-1

nchar( x )

[1] 1 1 1 1 1 1 1 1 1 2

nchar( y )

[1] 3 2 2 2 2 2 2 2 2 2

x + y

[1] -9 -7 -5 -3 -1  1  3  5  7  9

x - y

[1] 11 11 11 11 11 11 11 11 11 11
\end{block}
\end{frame}

\begin{frame}
\frametitle{При вектори с еднаква дължина}
\begin{block}{Умножение и деление}
x * y

[1] -10 -18 -24 -28 -30 -30 -28 -24 -18 -10

x / y

[1]  -0.1000000  -0.2222222  -0.3750000  -0.5714286  -0.8333333  -1.2000000

[7]  -1.7500000  -2.6666667  -4.5000000 -10.0000000
\end{block}
\end{frame}

\begin{frame}
\frametitle{При вектори с еднаква дължина}
\begin{block}{Степенуване и сравнение}
x\textasciicircum y

[1] 1.000000e+00 1.953125e-03 1.524158e-04 6.103516e-05 6.400000e-05

[6] 1.286008e-04 4.164931e-04 1.953125e-03 1.234568e-02 1.000000e-01

x > y

[1] TRUE TRUE TRUE TRUE TRUE TRUE TRUE TRUE TRUE TRUE
\end{block}
\end{frame}

\subsection{Списъци}

\subsection{Матрици}

\subsection{Масиви}

\subsection{Липсващи стойности}

\section{Структури подходящи за статистическа обработка}

\begin{frame}
\center \huge{Структури подходящи за статистическа обработка}
\end{frame}

\subsection{Рамкирани данни}

\section{Заключение}

\begin{frame}
\center \huge{Заключение}
\end{frame}

\subsection{Дискусия}

\begin{frame}
\frametitle{Въпроси и отговори}
\center \huge{Благодаря за вниманието!}
\end{frame}

\end{document}
