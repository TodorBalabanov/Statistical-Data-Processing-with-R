\documentclass{beamer}

\usepackage[T2A]{fontenc}
\usepackage[utf8x]{inputenc}
\usepackage[english,bulgarian]{babel}

\mode<presentation> {
	\usetheme{Berlin}
}

%\usebackgroundtemplate {
%	\includegraphics[width=370px, height=270px, trim=0 0 0 -80px]{background}
%}

\graphicspath{{../images/}}

\title{Пакетна организация, променливи, основни математически операции в R и типове данни}
\subtitle{Статистическа обработка на данни с R}

\author{Пламен Петров и Тодор Балабанов}

\date{03.V.2020}

\institute[ЦО и ИИКТ към БАН] {
	Център за обучение \\
	Институт по информационни и комуникационни технологии \\ 
	Българската академия на науките \\
	\medskip
	\textit{p.petrov@iit.bas.bg todorb@iinf.bas.bg}
}

\addtobeamertemplate{navigation symbols}{}{
	\usebeamerfont{footline}
	\usebeamercolor[fg]{footline}
	\hspace{1em}
	\insertframenumber/\inserttotalframenumber
}

\begin{document}

\begin{frame}
	\titlepage
\end{frame}

\section*{Теми}
\begin{frame}
	\frametitle{Съдържание}
	\tableofcontents
\end{frame}

\section{Инсталиране на пакети}

\begin{frame}
\center \huge{Инсталиране на пакети}
\end{frame}

\subsection{Пример с пакета coefplot}

\begin{frame}
\frametitle{Команда за инсталиране на пакета coefplot}
\begin{figure}[]\includegraphics[width=\textwidth,height=0.75\textheight]{pic0014}\end{figure}
\end{frame}

\begin{frame}
\frametitle{Избор на сървър за изтегляне на пакета}
\begin{figure}[]\includegraphics[width=\textwidth,height=0.75\textheight]{pic0015}\end{figure}
\end{frame}

\subsection{Връзки между пакетите}

\begin{frame}
\frametitle{Зависимости между пакетите}
\begin{figure}[]\includegraphics[width=\textwidth,height=0.75\textheight]{pic0016}\end{figure}
\end{frame}

\begin{frame}
\frametitle{Резултат от инсталацията на пакета}
\begin{figure}[]\includegraphics[width=\textwidth,height=0.75\textheight]{pic0017}\end{figure}
\end{frame}

\subsection{Зареждане и премахване на пакети}

\begin{frame}
\frametitle{Зареждане на пакета coefplot}
\begin{figure}[]\includegraphics[width=\textwidth,height=0.75\textheight]{pic0018}\end{figure}
\end{frame}

\begin{frame}
\frametitle{Премахване на пакета coefplot от общата памет}
\begin{figure}[]\includegraphics[width=\textwidth,height=0.75\textheight]{pic0019}\end{figure}
\end{frame}

\section{Основни математически операции}

\begin{frame}
\center \huge{Основни математически операции}
\end{frame}

\subsection{Изпълнение на команди с операции за пресмятане}

\begin{frame}
\frametitle{Примерни аритметични операции}
\begin{figure}[]\includegraphics[width=\textwidth,height=0.75\textheight]{pic0020}\end{figure}
\end{frame}

\subsection{Свойства на операциите}

\begin{frame}
\frametitle{Кардиналност и асоциативност}
\begin{block}{Събиране - бинарна операция}
1 + 1
\end{block}
\begin{block}{Минус - унарна операция}
-5
\end{block}
\begin{block}{Лява асоциативност}
2 + 2 + 2
\end{block}
\begin{block}{Дясна асоциативност}
a = b = 2
\end{block}
\end{frame}

\begin{frame}
\frametitle{Контекстна зависимост на операциите}
\begin{block}{Контекстна зависимост при събиране}
"abc" + "def"

2 + 2
\end{block}
\begin{block}{Контекстна зависимост при делене}
5 / 3

5.0 / 3.0
\end{block}
\end{frame}

\begin{frame}
\frametitle{Приоритет на операциите}
\begin{block}{Аритметични операции}
2 + 2 * 2
\end{block}
\begin{block}{Смяна на приоритета}
(2 + 2) * 2
\end{block}
\end{frame}

\section{Типове и променливи}

\begin{frame}
\center \huge{Типове и променливи}
\end{frame}

\subsection{}

\section{Заключение}

\begin{frame}
\center \huge{Заключение}
\end{frame}

\subsection{Дискусия}

\begin{frame}
\frametitle{Въпроси и отговори}
\center \huge{Благодаря за вниманието!}
\end{frame}

\end{document}
