\documentclass{beamer}

\usepackage[T2A]{fontenc}
\usepackage[utf8x]{inputenc}
\usepackage[english,bulgarian]{babel}
\usepackage{multirow}

\mode<presentation> {
	\usetheme{Berlin}
}

%\usebackgroundtemplate {
%	\includegraphics[width=370px, height=270px, trim=0 0 0 -80px]{background}
%}

\graphicspath{{../images/}}

\title{Групиране и обхождане на данни}
\subtitle{Статистическа обработка на данни с R}

\author{Пламен Петров и Тодор Балабанов}

\date{25.V.2020}

\institute[ЦО и ИИКТ към БАН] {
	Център за обучение \\
	Институт по информационни и комуникационни технологии \\ 
	Българската академия на науките \\
	\medskip
	\textit{p.petrov@iit.bas.bg todorb@iinf.bas.bg}
}

\addtobeamertemplate{navigation symbols}{}{
	\usebeamerfont{footline}
	\usebeamercolor[fg]{footline}
	\hspace{1em}
	\insertframenumber/\inserttotalframenumber
}

\begin{document}

\begin{frame}
	\titlepage
\end{frame}

\section*{Теми}
\begin{frame}[shrink]
	\frametitle{Съдържание}
	\tableofcontents
\end{frame}

\section{Фамилията функции apply}

\begin{frame}
\center \huge{Фамилията функции apply}
\end{frame}

\subsection{apply}

\begin{frame}
\frametitle{Обработка по редове или колеони}
\begin{block}{Сума по редове и колони}
m1 <- matrix(11:19, nrow=3)
 
apply(m1, 1, sum)
 
apply(m1, 2, sum)

m1[2,2] <- NA

apply(m1, 1, sum)

apply(m1, 2, sum)

apply(m1, 1, sum, na.rm=TRUE)

apply(m1, 2, sum, na.rm=TRUE)
\end{block}
\end{frame}

\subsection{lapply и sapply}

\begin{frame}
\frametitle{Вектори и списъци}
\begin{block}{Сума на обекти в списък}
l1 <- list(m2=matrix(1:9,3), l2=1:5, m3=matrix(1:4,2), n1=2)

lapply(l1, sum)

sapply(l1, sum)
\end{block}
\end{frame}

\subsection{mapply}

\begin{frame}
\frametitle{Множество от списъци}
\begin{block}{Проверка за идентичност на елементите}
l3 <- list(m4=matrix(1:25,5), m5=matrix(1:16,2), l4=1:5)

l5 <- list(m6=matrix(1:25,5), m7=matrix(1:16,8), l6=15:1)

mapply(identical, l3, l5)

mapply(f1<-function(x,y)\{NROW(x)+NROW(y)\}, l3, l5)
\end{block}
\end{frame}

\subsection{Агрегация}

\begin{frame}
\frametitle{Агрегация}
\begin{block}{Групиране на данни}
data(diamonds, package=$"$ggplot2$"$)

aggregate(price\textasciitilde cut, diamonds, mean)

aggregate(price\textasciitilde cut+color, diamonds, mean)

aggregate(cbind(price,carat)\textasciitilde cut, diamonds, mean)

aggregate(cbind(price,carat)\textasciitilde cut+color, diamonds, mean)
\end{block}
\end{frame}

\section{Пакетът plyr}

\begin{frame}
\center \huge{Пакетът plyr}
\end{frame}

\subsection{ddply}

\begin{frame}
\frametitle{Подготовка на данните}
\begin{block}{Бейзболна статистика}
library( plyr )

baseball\$sf[baseball\$year < 1954] <- 0

baseball\$hbp[ is.na(baseball\$hbp) ] <- 0

baseball <- baseball[baseball\$ab>=50,]

baseball\$OBP <- with(baseball, (h+bb+hbp)/(ab+bb+hbp+sf))
\end{block}
\end{frame}

\begin{frame}
\frametitle{Обработка на числител и делител}
\begin{block}{Пресмятане на OBP за цялата кариера на играча}
career <- ddply(baseball, .variables=$"$id$"$, .fun=function(data){c(OBP=with(data,sum(h+bb+hbp) / sum(ab+bb+hbp+sf)))})

career <- career[ order(career\$OBP, decreasing=TRUE), ]

head(career, n=3)
\end{block}
\end{frame}

\subsection{ddply}

\begin{frame}
\frametitle{Групова обработка}
\begin{block}{Сума на всеки елемент в списък}
l1 <- list(m2=matrix(1:9,3), l2=1:5, m3=matrix(1:4,2), n1=2)

llply(l1, sum)

identical(lapply(l1,sum), llply(l1,sum))

laply(l1, sum)
\end{block}
\end{frame}

\subsection{Помощни функции и бързодействие}

\begin{frame}
\frametitle{Усложняване на агрегацията}
\begin{block}{Повече от една агрегатна функция}
library(ggplot2)

aggregate(price\textasciitilde cut, diamonds, each(mean, median))
\end{block}
\end{frame}

\begin{frame}
\frametitle{Намалена консумация на памет}
\begin{block}{Бързодействие при използване на референции}
system.time(dlply(baseball, $"$id$"$, nrow))

reference <- idata.frame( baseball )

system.time(dlply(reference, $"$id$"$, nrow))
\end{block}
\end{frame}

\section{Пакетът data.table}

\begin{frame}
\center \huge{Пакетът data.table}
\end{frame}

\subsection{Разширяване на възможностите}

\begin{frame}
\frametitle{Реализация с вътрешно индексиране}
\begin{block}{Създаване на data.table}
df <- data.frame(x1=10:1, x2=letters[11:20], x3=LETTERS[1:10], x4=rep(c($"$One$"$, $"$Two$"$, $"$Three$"$), length.out=10))

dt <- data.table(x1=10:1, x2=letters[11:20], x3=LETTERS[1:10], x4=rep(c($"$One$"$, $"$Two$"$, $"$Three$"$), length.out=10))

diamonds <- data.table( diamonds )
\end{block}
\end{frame}

\begin{frame}
\frametitle{Достъп до информацията}
\begin{block}{По колони и редове}
dt[1:2, ]

dt[dt\$x1>=7, ]

dt[x1>=7, ]

dt[,list(x3,x4)]

dt[, x1]

dt[,list(x2)]

dt[, $"$x4$"$, with=FALSE]


td[, c($"$x2$"$, $"$x3$"$), with=FALSE]
\end{block}
\end{frame}

\subsection{Ключове}

\begin{frame}
\frametitle{Индексиране}
\begin{block}{Операции с таблици}
tables()

setkey(dt, x4)

key( dt )

setkey(diamonds, cut, color)
\end{block}
\end{frame}

\subsection{Агрегация}

\begin{frame}
\frametitle{}
\begin{block}{}
\end{block}
\end{frame}

\section{Заключение}

\begin{frame}
\center \huge{Заключение}
\end{frame}

\subsection{Дискусия}

\begin{frame}
\frametitle{Въпроси и отговори}
\center \huge{Благодаря за вниманието!}
\end{frame}

\end{document}
