\documentclass{beamer}

\usepackage[T2A]{fontenc}
\usepackage[utf8x]{inputenc}
\usepackage[english,bulgarian]{babel}
\usepackage{multirow}

\mode<presentation> {
	\usetheme{Berlin}
}

%\usebackgroundtemplate {
%	\includegraphics[width=370px, height=270px, trim=0 0 0 -80px]{background}
%}

\graphicspath{{../images/}}

\title{Групиране и обхождане на данни}
\subtitle{Статистическа обработка на данни с R}

\author{Пламен Петров и Тодор Балабанов}

\date{25.V.2020}

\institute[ЦО и ИИКТ към БАН] {
	Център за обучение \\
	Институт по информационни и комуникационни технологии \\ 
	Българската академия на науките \\
	\medskip
	\textit{p.petrov@iit.bas.bg todorb@iinf.bas.bg}
}

\addtobeamertemplate{navigation symbols}{}{
	\usebeamerfont{footline}
	\usebeamercolor[fg]{footline}
	\hspace{1em}
	\insertframenumber/\inserttotalframenumber
}

\begin{document}

\begin{frame}
	\titlepage
\end{frame}

\section*{Теми}
\begin{frame}[shrink]
	\frametitle{Съдържание}
	\tableofcontents
\end{frame}

\section{Фамилията функции apply}

\begin{frame}
\center \huge{Фамилията функции apply}
\end{frame}

\subsection{apply}

\begin{frame}
\frametitle{Обработка по редове или колеони}
\begin{block}{Сума по редове и колони}
m1 <- matrix(11:19, nrow=3)
 
apply(m1, 1, sum)
 
apply(m1, 2, sum)

m1[2,2] <- NA

apply(m1, 1, sum)

apply(m1, 2, sum)

apply(m1, 1, sum, na.rm=TRUE)

apply(m1, 2, sum, na.rm=TRUE)
\end{block}
\end{frame}

\subsection{lapply и sapply}

\begin{frame}
\frametitle{Вектори и списъци}
\begin{block}{Сума на обекти в списък}
l1 <- list(m2=matrix(1:9,3), l2=1:5, m3=matrix(1:4,2), n1=2)

lapply(l1, sum)

sapply(l1, sum)
\end{block}
\end{frame}

\subsection{mapply}

\begin{frame}
\frametitle{Множество от списъци}
\begin{block}{Проверка за идентичност на елементите}
l3 <- list(m4=matrix(1:25,5), m5=matrix(1:16,2), l4=1:5)

l5 <- list(m6=matrix(1:25,5), m7=matrix(1:16,8), l6=15:1)

mapply(identical, l3, l5)

mapply(f1<-function(x,y)\{NROW(x)+NROW(y)\}, l3, l5)
\end{block}
\end{frame}

\subsection{Агрегация}

\begin{frame}
\frametitle{Агрегация}
\begin{block}{Групиране на данни}
data(diamonds, package=$"$ggplot2$"$)

aggregate(price\textasciitilde cut, diamonds, mean)

aggregate(price\textasciitilde cut+color, diamonds, mean)

aggregate(cbind(price,carat)\textasciitilde cut, diamonds, mean)

aggregate(cbind(price,carat)\textasciitilde cut+color, diamonds, mean)
\end{block}
\end{frame}

\section{Пакетът plyr}

\begin{frame}
\center \huge{Пакетът plyr}
\end{frame}

\section{Заключение}

\begin{frame}
\center \huge{Заключение}
\end{frame}

\subsection{Дискусия}

\begin{frame}
\frametitle{Въпроси и отговори}
\center \huge{Благодаря за вниманието!}
\end{frame}

\end{document}
