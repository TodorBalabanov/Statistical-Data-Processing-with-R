\documentclass{beamer}

\usepackage[T2A]{fontenc}
\usepackage[utf8x]{inputenc}
\usepackage[english,bulgarian]{babel}
\usepackage{multirow}

\mode<presentation> {
	\usetheme{Berlin}
}

%\usebackgroundtemplate {
%	\includegraphics[width=370px, height=270px, trim=0 0 0 -80px]{background}
%}

\graphicspath{{../images/}}

\title{Групиране и обхождане на данни}
\subtitle{Статистическа обработка на данни с R}

\author{Пламен Петров и Тодор Балабанов}

\date{25.V.2020}

\institute[ЦО и ИИКТ към БАН] {
	Център за обучение \\
	Институт по информационни и комуникационни технологии \\ 
	Българската академия на науките \\
	\medskip
	\textit{p.petrov@iit.bas.bg todorb@iinf.bas.bg}
}

\addtobeamertemplate{navigation symbols}{}{
	\usebeamerfont{footline}
	\usebeamercolor[fg]{footline}
	\hspace{1em}
	\insertframenumber/\inserttotalframenumber
}

\begin{document}

\begin{frame}
	\titlepage
\end{frame}

\section*{Теми}
\begin{frame}[shrink]
	\frametitle{Съдържание}
	\tableofcontents
\end{frame}

\section{Фамилията функции apply}

\begin{frame}
\center \huge{Фамилията функции apply}
\end{frame}

\subsection{apply}

\begin{frame}
\frametitle{Обработка по редове или колеони}
\begin{block}{Сума по редове и колони}
m1 <- matrix(11:19, nrow=3)
 
apply(m1, 1, sum)
 
apply(m1, 2, sum)

m1[2,2] <- NA

apply(m1, 1, sum)

apply(m1, 2, sum)

apply(m1, 1, sum, na.rm=TRUE)

apply(m1, 2, sum, na.rm=TRUE)
\end{block}
\end{frame}

\subsection{lapply и sapply}

\begin{frame}
\frametitle{Вектори и списъци}
\begin{block}{Сума на обекти в списък}
l1 <- list(m2=matrix(1:9,3), l2=1:5, m3=matrix(1:4,2), n1=2)

lapply(l1, sum)

sapply(l1, sum)
\end{block}
\end{frame}

\subsection{mapply}

\begin{frame}
\frametitle{Множество от списъци}
\begin{block}{Проверка за идентичност на елементите}
l3 <- list(m4=matrix(1:25,5), m5=matrix(1:16,2), l4=1:5)

l5 <- list(m6=matrix(1:25,5), m7=matrix(1:16,8), l6=15:1)

mapply(identical, l3, l5)

mapply(f1<-function(x,y)\{NROW(x)+NROW(y)\}, l3, l5)
\end{block}
\end{frame}

\subsection{Агрегация}

\begin{frame}
\frametitle{Агрегация}
\begin{block}{Групиране на данни}
data(diamonds, package=$"$ggplot2$"$)

aggregate(price\textasciitilde cut, diamonds, mean)

aggregate(price\textasciitilde cut+color, diamonds, mean)

aggregate(cbind(price,carat)\textasciitilde cut, diamonds, mean)

aggregate(cbind(price,carat)\textasciitilde cut+color, diamonds, mean)
\end{block}
\end{frame}

\section{Пакетът plyr}

\begin{frame}
\center \huge{Пакетът plyr}
\end{frame}

\subsection{ddply}

\begin{frame}
\frametitle{Подготовка на данните}
\begin{block}{Бейзболна статистика}
library( plyr )

baseball\$sf[baseball\$year < 1954] <- 0

baseball\$hbp[ is.na(baseball\$hbp) ] <- 0

baseball <- baseball[baseball\$ab>=50,]

baseball\$OBP <- with(baseball, (h+bb+hbp)/(ab+bb+hbp+sf))
\end{block}
\end{frame}

\begin{frame}
\frametitle{Обработка на числител и делител}
\begin{block}{Пресмятане на OBP за цялата кариера на играча}
career <- ddply(baseball, .variables=$"$id$"$, .fun=function(data){c(OBP=with(data,sum(h+bb+hbp) / sum(ab+bb+hbp+sf)))})

career <- career[ order(career\$OBP, decreasing=TRUE), ]

head(career, n=3)
\end{block}
\end{frame}

\subsection{ddply}

\begin{frame}
\frametitle{Групова обработка}
\begin{block}{Сума на всеки елемент в списък}
l1 <- list(m2=matrix(1:9,3), l2=1:5, m3=matrix(1:4,2), n1=2)

llply(l1, sum)

identical(lapply(l1,sum), llply(l1,sum))

laply(l1, sum)
\end{block}
\end{frame}

\subsection{Помощни функции и бързодействие}

\begin{frame}
\frametitle{Усложняване на агрегацията}
\begin{block}{Повече от една агрегатна функция}
library(ggplot2)

aggregate(price\textasciitilde cut, diamonds, each(mean, median))
\end{block}
\end{frame}

\begin{frame}
\frametitle{Намалена консумация на памет}
\begin{block}{Бързодействие при използване на референции}
system.time(dlply(baseball, $"$id$"$, nrow))

reference <- idata.frame( baseball )

system.time(dlply(reference, $"$id$"$, nrow))
\end{block}
\end{frame}

\section{Пакетът data.table}

\begin{frame}
\center \huge{Пакетът data.table}
\end{frame}

\subsection{Разширяване на възможностите}

\begin{frame}
\frametitle{Реализация с вътрешно индексиране}
\begin{block}{Създаване на data.table}
df <- data.frame(x1=10:1, x2=letters[11:20], x3=LETTERS[1:10], x4=rep(c($"$One$"$, $"$Two$"$, $"$Three$"$), length.out=10))

dt <- data.table(x1=10:1, x2=letters[11:20], x3=LETTERS[1:10], x4=rep(c($"$One$"$, $"$Two$"$, $"$Three$"$), length.out=10))

diamonds <- data.table( diamonds )
\end{block}
\end{frame}

\begin{frame}
\frametitle{Достъп до информацията}
\begin{block}{По колони и редове}
dt[1:2, ]

dt[dt\$x1>=7, ]

dt[x1>=7, ]

dt[,list(x3,x4)]

dt[, x1]

dt[,list(x2)]

dt[, $"$x4$"$, with=FALSE]


td[, c($"$x2$"$, $"$x3$"$), with=FALSE]
\end{block}
\end{frame}

\subsection{Ключове}

\begin{frame}
\frametitle{Индексиране}
\begin{block}{Операции с таблици}
tables()

setkey(dt, x4)

key( dt )

setkey(diamonds, cut, color)
\end{block}
\end{frame}

\subsection{Агрегация}

\begin{frame}
\frametitle{Бързодействие заради индексирането}
\begin{block}{Агрегатни функции}
aggregate(price\textasciitilde cut, diamonds, mean)

diamonds[, list(price=mean(price)), by=cut]

diamonds[, list(price=mean(price)), by=list(cut,color)]

diamonds[, list(price=mean(price),carat=sum(carat)), by=list(cut,color)]
\end{block}
\end{frame}

\section{Бързи операции с пакета dplyr}

\begin{frame}
\center \huge{Бързи операции с пакета dplyr}
\end{frame}

\subsection{Потоци и таблици}

\begin{frame}
\frametitle{Верига от изчисления}
\begin{block}{Поточни операции}
library( ggplot2 )

library( magrittr )
 
dim( head(diamonds,n=4) )
 
diamonds \%>\% head(4) \%>\% dim
\end{block}
\end{frame}

\subsection{Извличане по колони}

\begin{frame}
\frametitle{Избор по редове по аналогия с релационните бази данни}
\begin{block}{Избор на редове}
diamonds \%>\% select(carat, price)

diamonds \%>\% select(c(carat, price))

diamonds \%>\% select\_('carat', 'price')

names <- c('carat', 'price')

diamonds \%>\% select\_(.dots=names)

diamonds \%>\% select( one\_of('carat', 'price') )

names <- c('carat', 'price')

diamonds \%>\% select( one\_of(names) )

select(diamonds, 1, 7)

diamonds \%>\% select(1, 7)
\end{block}
\end{frame}

\begin{frame}
\frametitle{Допълнителни възможности при търсене}
\begin{block}{Търсене по частично съвпадение}
diamonds \%>\% select( starts\_with('c') )

diamonds \%>\% select( ends\_with('e') )

diamonds \%>\% select( contains('l') )

diamonds \%>\% select( matches('r.+t') )
\end{block}
\end{frame}

\begin{frame}
\frametitle{Отсяване на информация}
\begin{block}{Изключване на колони}
diamonds \%>\% select(-carat, -price)

diamonds \%>\% select( -c(carat, price) )

diamonds \%>\% select(-1, -7)

diamonds \%>\% select( -c(1,7) )

diamonds \%>\% select\_(.dots=c('-carat', '-price'))

diamonds \%>\% select( -one\_of('carat','price') )
\end{block}
\end{frame}

\subsection{Филтриране}

\begin{frame}
\frametitle{Отсяване на информация}
\begin{block}{Филтриране на редове}
diamonds \%>\% filter(cut == 'Ideal')

diamonds[diamonds\$cut == 'Ideal', ]

diamonds \%>\% filter(cut \%in\% c('Ideal', 'Good'))

diamonds \%>\% filter(carat > 2, price < 14000)

diamonds \%>\% filter(carat > 2 \& price < 14000)

diamonds \%>\% filter(carat < 1 | carat > 5)
\end{block}
\end{frame}

\begin{frame}
\frametitle{Отсяване на информация}
\begin{block}{Филтриране на редове}
diamonds \%>\% filter\_($"$cut == 'Ideal'$"$)

diamonds \%>\% filter\_(\textasciitilde cut == 'Ideal')

cut <- 'Ideal'

diamonds \%>\% filter\_(\textasciitilde cut == cut)

col <- 'cut'

cut <- 'Ideal'

diamonds \%>\% filter\_(sprintf($"$\%s == '\%s'$"$, col, cut))
\end{block}
\end{frame}

\begin{frame}
\frametitle{Използване на индексирането}
\begin{block}{Избор по индекси}
diamonds \%>\% slice(1:5)

diamonds \%>\% slice(c(1:5, 8, 15:20))

diamonds \%>\% slice(-1)
\end{block}
\end{frame}

\subsection{Модификация, обобщение, групиране и подреждане}

\begin{frame}
\frametitle{Модификация на колони}
\begin{block}{Допълнителна колона}
diamonds \%>\% mutate(price/carat)

diamonds \%>\% select(carat, price) \%>\% mutate(price/carat)

diamonds \%>\% select(carat, price) \%>\% mutate(ratio=price/carat)

diamonds \%>\% select(carat, price) \%>\% mutate(ratio=price/carat, square=ratio*ratio)
\end{block}
\end{frame}

\begin{frame}
\frametitle{Двупосочна поточна операции}
\begin{block}{Отразяване на модификациите}
diamonds2 <- diamonds

diamonds2 \%<>\% select(carat, price) \%>\% mutate(ratio=price/carat, square=ratio*ratio)

head(diamonds2, n=3)
\end{block}
\end{frame}

\begin{frame}
\frametitle{Изчисления по няколко агрегатни фунции}
\begin{block}{Обобщаваща информация}
summarize(diamonds, sd(price))

diamonds \%>\% summarize(sd(price))

diamonds \%>\% summarize(AveragePrice=mean(price), MedianPrice=median(price), AverageCarat=mean(carat))
\end{block}
\end{frame}

\begin{frame}
\frametitle{Информативно представяне на данните}
\begin{block}{Групиране при обобщение}
diamonds \%>\% group\_by(cut, color) \%>\% summarize(AveragePrice=mean(price), SumCarat=sum(carat))
\end{block}

\begin{block}{Подредба на резултатите}
diamonds \%>\% group\_by(cut) \%>\% summarize(AveragePrice=mean(price), SumCarat=sum(carat)) \%>\% arrange(desc(SumCarat))
\end{block}
\end{frame}

\subsection{Специфични изчисления и връзка с база данни}

\begin{frame}
\frametitle{Потребителски функции върху данни}
\begin{block}{Специфични изчисления}
bottom <- function(x, N=5) \{ x \%>\% arrange(carat) \%>\% head(N) \}

diamonds \%>\% group\_by(cut) \%>\% do(bottom(., N=3))
\end{block}
\end{frame}

\begin{frame}
\frametitle{Системи за управление на бази от данни}
\begin{block}{Работа с база данни}
setwd( $"$\textasciitilde /Desktop$"$ )

download.file($"$https://github.com/TodorBalabanov/Statistical-Data-Processing-with-R/blob/master/data/diamonds.db? raw=true$"$, destfile=$"$./diamonds.db$"$, mode=$"$wb$"$)

source <- src\_sqlite($"$diamonds.db$"$)

table <- tbl(source, $"$diamonds$"$)
\end{block}
\end{frame}

\begin{frame}
\frametitle{Интерфейс към данни в СУБД}
\begin{block}{Пресмятания с данни в база данни}
table \%>\% group\_by(cut) \%>\% dplyr::summarize(AveragePrice=mean(price), AverageCarat=mean(carat))
\end{block}
\end{frame}

\section{Пакетът purrr}

\begin{frame}
\center \huge{Пакетът purrr}
\end{frame}

\subsection{Фамилията функции map}

\begin{frame}
\frametitle{Обработка елемент по елемент}
\begin{block}{Прилагане на функцията map}
library(purrr)

l1 <- list(m2=matrix(1:9,3), l2=1:5, m3=matrix(1:4,2), n1=2)

l1 \%>\% map( sum )

l1 \%>\% map(function(x) sum(x, na.rm=TRUE))

l1 \%>\% map(sum, na.rm=TRUE)
\end{block}
\end{frame}

\begin{frame}
\frametitle{Фамилия функции map}
\begin{table}[ht]
\centering
\begin{tabular}{|l|r|} 
  \hline
  Функция & Тип на върнатата стойност \\ [0.1ex] 
  \hline\hline
  map & list \\
  \hline
  map\_int & integer \\
  \hline
  map\_dbl & numeric \\
  \hline
  map\_chr & character \\
  \hline
  map\_lgl & logical \\
  \hline
  map\_df & data.frame \\
  \hline
\end{tabular}
\end{table}
\end{frame}

\begin{frame}
\frametitle{Контекстна зависимост}
\begin{block}{Извиквания на map\, според типа на върнатата стойност}
l1 \%>\% map\_int(NROW)

l1 \%>\% map\_dbl(mean)

l1 \%>\% map\_chr(class)

l1 \%>\% map\_lgl(function(x) NROW(x) < 3)

list(3,4,1,5) \%>\% map( function(x)\{ data.frame(A=1:x,B=x:1) \} )

list(3,4,1,5) \%>\% map\_df( function(x)\{ data.frame(A=1:x,B=x:1) \} )

l1 \%>\% map\_if(is.matrix, function(x) x*2)

l1 \%>\% map\_if(is.matrix, \textasciitilde  .x*2)
\end{block}
\end{frame}

\begin{frame}
\frametitle{Група от списъци}
\begin{block}{Обхождане на data.frame}
data(diamonds, package='ggplot2')

diamonds \%>\% map\_dbl(mean)
\end{block}

\begin{block}{Едновременно обхождане на два списъка}
l3 <- list(m4=matrix(1:25,5), m5=matrix(1:16,2), l4=1:5)

l5 <- list(m6=matrix(1:25,5), m7=matrix(1:16,8), l6=15:1)

map2(l3, l5, function(x, y)\{NROW(x) + NROW(y)\})

map2\_int(l3, l5, function(x, y)\{NROW(x) + NROW(y)\})

pmap(list(l3, l5), function(x, y)\{NROW(x) + NROW(y)\})
\end{block}
\end{frame}

\section{Заключение}

\begin{frame}
\center \huge{Заключение}
\end{frame}

\subsection{Дискусия}

\begin{frame}
\frametitle{Въпроси и отговори}
\center \huge{Благодаря за вниманието!}
\end{frame}

\end{document}
