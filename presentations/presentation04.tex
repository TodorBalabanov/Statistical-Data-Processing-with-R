\documentclass{beamer}

\usepackage[T2A]{fontenc}
\usepackage[utf8x]{inputenc}
\usepackage[english,bulgarian]{babel}

\mode<presentation> {
	\usetheme{Berlin}
}

%\usebackgroundtemplate {
%	\includegraphics[width=370px, height=270px, trim=0 0 0 -80px]{background}
%}

\graphicspath{{../images/}}

\title{Въвеждане на данни и извеждане на графики}
\subtitle{Статистическа обработка на данни с R}

\author{Пламен Петров и Тодор Балабанов}

\date{12.V.2020}

\institute[ЦО и ИИКТ към БАН] {
	Център за обучение \\
	Институт по информационни и комуникационни технологии \\ 
	Българската академия на науките \\
	\medskip
	\textit{p.petrov@iit.bas.bg todorb@iinf.bas.bg}
}

\addtobeamertemplate{navigation symbols}{}{
	\usebeamerfont{footline}
	\usebeamercolor[fg]{footline}
	\hspace{1em}
	\insertframenumber/\inserttotalframenumber
}

\begin{document}

\begin{frame}
	\titlepage
\end{frame}

\section*{Теми}
\begin{frame}[shrink]
	\frametitle{Съдържание}
	\tableofcontents
\end{frame}

\section{Структурирани източници на данни}

\begin{frame}
\center \huge{Структурирани източници на данни}
\end{frame}

\subsection{CSV файлове}

\subsection{Excel файлове}

\subsection{SQL бази данни}

\subsection{Четене на данни от JSON формат}

\section{Бинарни източници на данни}

\begin{frame}
\center \huge{Бинарни източници на данни}
\end{frame}

\subsection{Други статистически програми като източници на данни}

\subsection{Бинарни файлове на R}

\subsection{Данни достъпни директно от R}

\section{Визуализация на данни от статистически анализ}

\begin{frame}
\center \huge{Визуализация на данни от статистически анализ}
\end{frame}

\subsection{Хистограма}

\subsection{Диаграма на разсейване}

\subsection{Графики тип кутия}

\section{Заключение}

\begin{frame}
\center \huge{Заключение}
\end{frame}

\subsection{Дискусия}

\begin{frame}
\frametitle{Въпроси и отговори}
\center \huge{Благодаря за вниманието!}
\end{frame}

\end{document}
